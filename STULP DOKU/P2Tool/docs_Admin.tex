\documentclass{article}
\usepackage[ngerman]{babel}
\usepackage{graphicx}
\usepackage{geometry}
\usepackage{xcolor}

\definecolor{warningbackground}{RGB}{252,226,158}
%\definecolor{infoforeground}{RGB}{58,135,173}
\definecolor{warningborder}{RGB}{219,194,129}
\definecolor{link}{RGB}{51,102,204}

\usepackage{environ}
\usepackage{tikz}
\usetikzlibrary{fit,backgrounds,calc}

\NewEnviron{warning}
{
    \vskip \baselineskip
    \begin{tikzpicture}
        \node[inner sep=1pt, draw=warningborder, rounded corners=0.1cm, fill=warningbackground] (box) {
            \parbox[t]{\textwidth}
            {%
                \begin{minipage}{.1\textwidth}
                    \vskip 4pt
                    \centering\tikz[scale=1]
                    \node[scale=1]
                    {
                        \makebox[0pt][c]{%
                        \makebox[0pt][c]{\raisebox{.2em}{\small!}}%
                        \makebox[0pt][c]{\LARGE$\bigtriangleup$}}
                    };
                \end{minipage}%
                \begin{minipage}{.85\textwidth}
                    \vspace{5pt}
                    \BODY
                    \vspace{5pt}
                \end{minipage}\hfill
            }%
        };
    \end{tikzpicture}
}

\usepackage[colorlinks=true, linkcolor=link, urlcolor=link, citecolor=link, anchorcolor=link]{hyperref}
%\usepackage{color}
%\renewcommand\UrlFont{\color{blue}\rmfamily}
\newcommand{\secref}[1]{\hyperref[#1]{Abschnitt~\ref{#1}}}
\newcommand{\figref}[1]{\hyperref[#1]{Abbildung~\ref{#1}}}
\newcommand{\tabref}[1]{\hyperref[#1]{Tabelle~\ref{#1}}}


\title{P2Tool}
\author{Projektgruppe STULP \\ Alessio, Alexander, Daniel, Daniel, Niklas, Paul}
\date{\today}

\geometry{
  a4paper,
  top=3cm,
  bottom=3cm,
  left=2.5cm,
  right=2.5cm,
}

% Increase spacing between lines
\linespread{1.2}

% Remove indentations but increase line spacing between paragraphs
\setlength\parindent{0pt}
%\setlength\parskip{0.6 \baselineskip}


\begin{document}

\maketitle

\tableofcontents
\section{Systemadministration}
Dieses Kapitel enthält die Informationen, die der Administrator benötigt, um das Tool in Betrieb zu nehmen und zu warten.
\paragraph{Grundlagen}
Das Tool wurde auf einer vom IRB zur Verfügung gestellten Debian VM getestet und die folgende Anleitung ist für den Betrieb auf einer solchen konzipiert.  
Die Verwendung einer solchen wird empfohlen, ist aber nicht zwingend erforderlich. 
Die Kommunikation mit der VM erfolgt über SSH und ist nur aus dem Netzwerk der Universität oder über ein entsprechendes VPN \footnote{https://hilfe.uni-paderborn.de/VPN\_unter\_Windows} möglich. Für den Aufbau einer SSH-Verbindung zur VM wird der Single-Sign-On-Dienst der Universität Kerberos\footnote{https://cs.uni-paderborn.de/rechnerbetrieb-irb/old/dienste/login-per-kerberos} benötigt. Alternativ ist auch eine direkte Kommunikation über das Virtual Service Center\footnote{https://vmc.cs.uni-paderborn.de/} möglich, wird aber nicht empfohlen, da wichtige Quality of Live Features wie Copy and Paste nicht oder nur unzureichend unterstützt werden.
\paragraph{VM Konfigurieren}
Sollte es nötig werden beim IRB eine neue VM zu beantragen, ist es ausreichend, eine Debian VM in der Standard Konfiguration mit Backups zu beantragen. Eine VM mit Backup wird vom IRB alle 24 Stunden gespeichert, die täglichen Backups werden zu wöchentlichen Backups aggregiert und die wöchentlichen Backups zu monatlichen. Im Schadensfall durch bespielsweise fehlerhafte Bedienung, kann das IRB die VM in den Zustand eines geigneten Backups zurücksetzten.

\paragraph{Dateistruktur}
Das git repository\footnote{\url{https://git.cs.uni-paderborn.de/stulp/stulp_vm_setup.git}} enthält alle Dateien, die für den Betrieb des Servers notwendig sind. Einige besonders wichtige Dateien und ihre Funktion sind hier zusammengefasst. Die Datei server.xml beschreibt, auf welchen Ports der Server lauscht und welches Zertifikat er verwendet. Der Ordner \textbf{Scipts} enthält verschiedene Skripte, die Aufgaben wie das Erstellen eines Backups oder das Einspielen eines Updates automatisieren. Der Ordner \textbf{webapps} enthält die ROOT.war, aus der der Server beim Start alle p2tool-spezifischen Dateien generiert. Über die Datei startup.sh ist ein manueller Start des Servers ohne Systemdienst möglich.



\paragraph{Verwalten von Ports}
Die Ports, auf denen der Tomcat-Server lauscht, sind in der Datei \textbf{Server.xml} definiert. Hierbei ist zu beachten, dass es für Prozesse ohne sudo-Rechte nicht möglich ist, auf den Standard-http- und https-Ports 80/443 zu lauschen. Ein dauerhafter Betrieb des Tools mit sudo-Rechten ist aus Sicherheitsgründen problematisch. Es wird empfohlen, eine Weiterleitung von diesen Ports auf andere Ports wie z.B. 8080/8443 einzurichten. Eine solche permanente Weiterleitung kann z.B. mit dem Programm iptables wie folgt eingerichtet werden
\begin{verbatim}
sudo apt-get install iptables-persistent 
sudo iptables -t nat -A PREROUTING -p tcp --dport 443 -j REDIRECT --to-port 8443 
sudo iptables-save > /etc/iptables/rules.v4
\end{verbatim}


\paragraph{Einrichten von https}
Damit der Server eine sichere Verbindung über https anbieten kann, benötigt er ein Zertifikat mit zugehörigem geheimen Schlüssel. Die zur Erzeugung eines solchen Zertifikats notwendigen Dateien befinden sich standardmäßig im Verzeichnis \textbf{/etc/ssl/private}. Ein passendes Zertifikat kann mit openssl erzeugt werden. Beispielsweise mit dem Befehl
\textbf{openssl pkcs12 -export -in p2tool-buster.cs.uni-paderborn.de.pem -inkey p2tool-buster.cs.uni-paderborn.de.key -name server \textgreater server.p12 }
In der Datei \textbf{server.xml} muss das bei der Zertifikatserstellung verwendete Passwort und ein Pfad zum Zertifikat hinterlegt werden.

\paragraph{Systemctl und der Tomcat\_user}
Damit Tomcat den Server beim Start der VM automatisch startet, empfiehlt es sich, einen Systemdienst zu verwenden. Dies ist notwendig, da der IRB den Server periodisch neu startet, um Updates einzuspielen. Dazu sollte ein dedizierter Benutzer tomcat\_user angelegt werden.
Des Weiteren muss eine Servicekonfiguration im Pfad \textbf{/etc/systemd/system/tomcat.service} erstellt werden. Ein Beispiel für eine funktionierende Datei wäre wie folgt. Passende Werte für die Umgebungsvariablen können der Ausgabe des Servers, entnommen werden, die man erhält, wenn man den Server, mit der \textbf{startup.sh} manuell startet. Eine Anpassung der Pfade kann erforderlich sein, wenn die absoluten Pfade zum Server abweichen.
Eine Änderung in der Datei \textbf{/etc/systemd/system/tomcat.service} wird mit \textbf{ sudo systemctl daemon-reload} geladen.
Der Befehl \textbf{ systemctl status tomcat.service} erzeugt einen Statusreport des Servers.  
Der Befehl \textbf{systemctl status tomcat.service} startet den Server bei Bedarf manuell.
\begin{verbatim}
[Unit]
Description=Apache Tomcat Web Application Container
After=network.target

[Service]
Type=forking

# Environment setup
Environment="JRE_HOME=/usr"  # Specify your Java installation path
Environment="CATALINA_TMPDIR=/stulp_vm_setup/vlpt-tomcat/temp"
Environment="CATALINA_HOME=/stulp_vm_setup/vlpt-tomcat"
Environment="CATALINA_BASE=/stulp_vm_setup/vlpt-tomcat"
Environment="CLASSPATH=/stulp_vm_setup/vlpt-tomcat/bin/bootstrap.jar: ←
/stulp_vm_setup/vlpt-tomcat/bin/tomcat-juli.jar" 
WorkingDirectory=/stulp_vm_setup/vlpt-tomcat/bin

# Define the Tomcat startup and shutdown scripts
ExecStart=/stulp_vm_setup/vlpt-tomcat/bin/startup.sh
ExecStop=/stulp_vm_setup/vlpt-tomcat/shutdown.sh

# Run Tomcat as the specified user
User=tomcat_user
Group=tomcat_user

# Restart Tomcat if it stops unexpectedly
Restart=on-failure

[Install]
WantedBy=multi-user.target
\end{verbatim}




\paragraph{Nutzer-Rechte}
Da der tomcat\_user keine sudo Rechte hat, ist es notwendig, dass ihm alle Dateien auf die der Server zugreifen muss gehören oder seiner Gruppe zugeordnet sind. Es ist zu beachten, dass Operationen mit git oder das Einspielen von Patches die Ownership der Datein teilweise zurücksetzten, was zu einen Serverfehler mit einer FileNotFoundException führt. Eine korrekte und sichere Ownership kann beispielsweise mit den folgenden Befehlen sichergestellt werden. 

\begin{verbatim}
sudo chgrp -R tomcat_user stulp_vm_setup/
sudo chmod -R 770 stulp_vm_setup/

\end{verbatim}





\section{Profil}
\label{sec:profil}

Über das Menü, das per Klick auf den Namen des eingeloggten Benutzers in der oberen rechten Ecke der Anwendung erreichbar ist, können zwei Aktionen erfolgen: das Ändern des Passworts sowie das Ausloggen aus der aktuellen Sitzung.

\begin{figure}[h!]
    \centering
    \includegraphics[width=0.75\linewidth]{./img/Profil/profile_menu.png}
    \caption{Profilmenü}
    \label{fig:profile_menu}
\end{figure}



\subsection{Passwort ändern}

Auf dieser Seite können Benutzer ihr zum Einloggen verwendetes Passwort ändern.
Um das bestehende Passwort zu ändern, muss ebendieses zuerst bestätigt werden.
Anschließend werden das neue Passwort sowie eine zusätzliche Überprüfung von diesem eingetragen.
Das Passwort wird mit einem Klick auf Speichern geändert.

\begin{figure}[h!]
    \centering
    \includegraphics[width=0.75\linewidth]{./img/Profil/profile_password.png}
    \caption{Passwort ändern}
    \label{fig:profile_password}
\end{figure}


\section{Zeitfenster}
\label{sec:timeslots}
Die Seite \textit{Zeitfenster} ermöglicht das Festlegen von Veranstaltungszeitfenstern für ein Semester.
Dies beeinflusst die dargestellten Auswahlmöglichkeiten für Wunschzeiten der Veranstalter.
Die Zeitfenster können für unterschiedliche Tage unterschiedlich gewählt werden.

\begin{figure}[h!]
    \centering
    \includegraphics[width=0.75\linewidth]{./img/timeslots_overview.png}
    \caption{Übersicht der Zeitfenster}
    \label{fig:timeslots_overview}
\end{figure}


\subsection{Hinzufügen}

Um ein neues Zeitfenster hinzuzufügen, muss zunächst eine Zelle in der Tabelle ausgewählt werden.
Anschließend muss die Dauer des Zeitfensters eingegeben und die Eingabe mit \texttt{Enter} bestätigt werden (siehe \figref{fig:timeslots_add}).
Eine Erfolgsmeldung bestätigt das erfolgreiche Hinzufügen des Zeitfensters.

\begin{warning}
    Zeitfenster dürfen sich nicht überschneiden.
\end{warning}

\begin{figure}[h!]
    \centering
    \includegraphics[width=0.75\linewidth]{img/timeslots_add.png}
    \caption{Zeitfenster hinzufügen}
    \label{fig:timeslots_add}
\end{figure}


\subsection{Entfernen}

Um ein bestehendes Zeitfenster zu entfernen, muss das Papierkorbsymbol im gewünschten Zeitfenster geklickt werden (siehe \figref{fig:timeslots_delete}).
Eine Erfolgsmeldung bestätigt das erfolgreiche Entfernen des Zeitfensters.

\begin{figure}[h!]
    \centering
    \includegraphics[width=0.75\linewidth]{img/timeslots_delete.png}
    \caption{Zeitfenster entfernen}
    \label{fig:timeslots_delete}
\end{figure}

\subsection{Nebenwirkungen}
Wenn in einem Semester die Zeitfenster geändert werden, ergeben sich einige Nebenwirkungen.
Innerhalb des Semesters werden alle angebotenen Lehrveranstaltungen so geändert, dass das gelöschte Zeitfenster nicht mehr als Wunschzeit auswählbar ist und auch in der Datenbank nicht mehr als Wunschzeit gespeichert wird.
Das führt dazu, dass wenn ein Zeitfenster gelöscht wird und später ein Zeitfenster an der gleichen Stelle hinzugefügt wird, Wunschzeiten nicht wiederhergestellt werden können und diese Zeitfenster für alle angebotenen Lehrveranstaltungen zunächst nicht als Wunschzeit angegeben sind.
Konkret bedeutet das, dass falls eine Wunschzeit einer angebotenen Lehrveranstaltung gelöscht wird, diese (und nur diese) Wunschzeit verschwindet.

Außerhalb des Semesters hat das Ändern von Zeitfenstern die Auswirkung, dass die Wunschzeiten nicht mehr kopiert werden können, wenn ein Kurs zum Beispiel aus dem letzten Semester kopiert wird.
Für diesen Fall wird wie beim Erstellen von neuen Kursen standardmäßig kein Zeitfenster als Wunschzeit angegeben.
Zu beachten ist, dass auch hier schon das Löschen und anschließende Hinzufügen des selben Zeitfensters als Änderung zählt.


\section{Semesterauswahl (als Admin / Planer)}
\label{sec:semesterauswahl}
Auf allen Seiten des Tools gibt es die Möglichkeit auszuwählen, in welchem Semester Änderungen vorgenommen werden sollen.
Oben links wird angezeigt, welches Semester aktuell ausgewählt ist.
In \figref{fig:semesterauswahl} ist aktuell zum Beispiel das Sommersemester 2025 ausgewählt, und alle Änderungen, die der Benutzer macht, werden in diesem Semester vollzogen.

\begin{figure}[h!]
    \centering
    \includegraphics[width=0.75\linewidth]{img/Semester_auswahl.png}
    \caption{Semesterverwaltung}
    \label{fig:semesterauswahl}
\end{figure}

Lehrende befinden sich immer im neuesten Semester, dass auf dem Server existiert.
Administratoren und Planer können in ein vergangenes Semester wechseln, um noch Änderungen nachzutragen, während nebenbei schon ein neueres Semester geplant wird.
\begin{warning}
    Beim Erstellen eines neuen Semesters wird das aktuell ausgewählte Semester von allen Benutzern auf das neu erstellte Semester geändert.
\end{warning}

Das derzeit ausgewählte Semester kann geändert werden, indem der Benutzer auf das aktuell ausgewählte Semester klickt. Dann öffnet sich eine Liste mit allen aktuellen Semestern, und mit einem Klick auf das gewünschte Semester wird dieses ausgewählt (siehe \figref{fig:semesterauswahldialog}).
\begin{figure}[h!]
    \centering
    \includegraphics[width=0.75\linewidth]{img/Semester_auswahl_dialog.png}
    \caption{Semesterverwaltung}
    \label{fig:semesterauswahldialog}
\end{figure}


\section{Semesterverwaltung}
\label{sec:semesterverwaltung}
Die Seite \textit{Semesterverwaltung} ermöglicht das Erstellen von neuen Semestern.
Generell sind alle Daten innerhalb dieses Tools in zwei Kategorien einzuteilen: semesterabhängige Daten und semesterübergreifende Daten.
Benutzer werden in der USERDB.db gepeichert und bleiben über alle Semester erhalten.
Genauso die Daten über Semester, die jedem Semester eine ID und einen Namen sowie einen Dateipfad zur zugehörigen VLPT.db Datenbank zuordnen.
Alles andere ist für jedes Semester einzeln definiert.
Wunschzeiten, Zeitfenser, Sperrzeiten, ... werden für jedes Semester einzeln gespeichert und können für einzelne Semester geändert werden.

In der Semesterverwaltung werden alle aktuell schon vorhandenen Semester angezeigt und eine Möglichkeit gegeben neue Semester zu erstellen, indem der \textit{Neues Semester starten} Knopf gedrückt wird (siehe \figref{fig:semesterverwaltung}).
\begin{figure}[h!]
    \centering
    \includegraphics[width=0.75\linewidth]{img/Semesterverwaltung.png}
    \caption{Semesterverwaltung}
    \label{fig:semesterverwaltung}
\end{figure}

Beim Erstellen eines neuen Semesters muss angegeben werden, ob es sich um ein Sommersemester oder Wintersemester handelt und das Jahr muss angegeben werden. Standardmäßig wird immer ein Halbjahr und Jahr angegeben, dass dem Semester nach dem aktuell neuesten entspricht (siehe \figref{fig:semesterverwaltung_new}).
\begin{figure}[h!]
    \centering
    \includegraphics[width=0.75\linewidth]{img/Semesterverwaltung_new_semester.png}
    \caption{Semesterverwaltung}
    \label{fig:semesterverwaltung_new}
\end{figure}

\begin{warning}
    Ein neues Semester sollte nur erstellt werden, wenn der Ersteller sich derzeit im aktuellsten Semester befindet. Andernfalls treten Inkonsistenzen auf und nicht alle Kurse, Professoren und Module werden übernommen.
\end{warning}

Beim Erstellen eines neuen Semesters werden die folgenden Daten aus dem aktuell ausgewählten Semester übernommen:
\begin{itemize}
    \item Druckrubriken
    \item Module
    \item Studiengänge
    \item Kurse (der Name, PaulNr, Kürzel, Veranstaltungsteile und ihre Dauer, usw.)
    \item Professoren
    \item Zeitfenster
\end{itemize}

\textbf{Nicht} kopiert werden Eigenschaften von angebotenen Veranstaltungen (alles was bei \textit{Meine Veranstaltungen} eingestellt wird siehe Kapitel \ref{sec:meine_veranstaltungen}), da diese Eingaben jeweils nur für ein Semester gültig sind.
Damit Benutzer nicht immer alles neu eingeben müssen, können sie Eigenschaften von angebotenen Veranstaltungen aber über Semester hinweg kopieren wie in Kapitel \ref{sec:meine_veranstaltungen_copy} beschrieben wird.
Zu diesen Eigenschaften gehören:
\begin{itemize}
    \item Welcher Professor bietet welchen Kurs an
    \item Übungsanzahlen für Übungen
    \item Wunschzeiten
    \item Bemerkungen für die Planung
    \item Die Startwoche von Veranstaltungen
    \item Ob eine LV zweiwöchentlich angeboten wird
    \item Raumwünsche (Campus/FU)
    \item Angaben, ob Beamer, Fenster, Tafel gewünscht sind
    \item Hörerzahlen
\end{itemize}


\section{Meine Veranstaltungen}
\label{sec:meine_veranstaltungen}
Auf der Seite \textit{Meine Veranstaltungen} können Veranstaltungen des Veranstaltungskatalogs angeboten werden.
Sie wird genutzt, um angebotene Veranstaltungen konfigurieren zu können und dafür die Anzahl der maximalen Hörer einzugeben, sowie Wünsche bezüglich des Raumes zu äußern und Wunschzeiten anzugeben.
Am Anfang, wenn Sie noch keine Kurse angeboten haben, sieht die Seite noch leer aus und bietet oben unter dem Punkt \textit{Aktuelle Veranstaltungen} die Möglichkeit, eine Veranstaltung anzubieten (siehe \figref{fig:meine_veranstaltungen_vplan}).
Diese Seite zeigt immer die Veranstaltungen an, die Sie in Ihrem ausgewählten Semester anbieten.
Bevor Sie Veranstaltungen hinzufügen oder löschen, vergewissern Sie sich, dass Sie sich im richtigen Semester befinden.
Lehrende können nur Veranstaltungen im Semester des aktuellen Planungszyklus Änderungen vornehmen.
Falls kurzfristige Änderungen für das vorherige Semester vorgenommen werden sollen, muss dies über den aktuellen Planer geschehen.
\begin{figure}[h!]
    \centering
    \includegraphics[width=0.75\linewidth]{img/MeineVeranstaltungen//MeinVeranstaltungsplan.png}
    \caption{Meine Veranstaltungen}
    \label{fig:meine_veranstaltungen_vplan}
\end{figure}

\subsection{Veranstaltung anbieten}
\label{sec:meine_veranstaltungen_add}

Um eine Veranstaltung für das aktuell ausgwählte Semester anzubieten, muss der \textit{Hinzufügen} Knopf unter \textit{Aktuelle Veranstaltungen} gedrückt werden.
Dadurch wird die Seite \textit{Veranstaltungsplan bearbeiten} geöffnet (siehe \figref{fig:meine_veranstaltungen_vplan_add}).
\begin{figure}[h!]
    \centering
    \includegraphics[width=0.75\linewidth]{img/MeineVeranstaltungen//MeinVeranstaltungsplan_AddVPlan.png}
    \caption{Meine Veranstaltungen - Veranstaltungsplan bearbeiten}
    \label{fig:meine_veranstaltungen_vplan_add}
\end{figure}
Hier muss zuerst der Fachbereich ausgewählt werden, von dem die Veranstaltung angeboten werden soll.
Für Veranstaltungen wie Analysis, Algebra und Stochastik muss hier \textit{Mathematik} ausgewählt werden, für Veranstaltungen wie Grundlagen der Elektrotechnik, Signalverarbeitung und Nachrichtentechnik muss hier \textit{Elektrotechnik} ausgewählt werden.
Danach kann durch einen Klick auf das Dropdown Menu in der Reihe Kurs ein Kurs ausgewählt werden.
Hier werden alle Kurse des vorher ausgewählten Fachbereichs angezeigt und können durch einen Klick ausgewählt werden (siehe \figref{fig:meine_veranstaltungen_vplan_add_kursliste}).
\begin{figure}[h!]
    \centering
    \includegraphics[width=0.75\linewidth]{img/MeineVeranstaltungen//MeinVeranstaltungsplan_AddVPlan_Kursliste.png}
    \caption{Meine Veranstaltungen - Veranstaltungsplan bearbeiten - Kursauswahl}
    \label{fig:meine_veranstaltungen_vplan_add_kursliste}
\end{figure}
Anschließend muss noch die Sprache, in der die Lehrveranstaltung gehalten wird, durch einen Klick auf Deutsch oder Englisch ausgewählt werden.
Danach muss auf den Knopf \textit{Nächster Schritt} gedrückt werden, um die Veranstaltung weiter zu konfigurieren.

Nun öffnet sich die Seite \textit{Veranstaltungsplan bearbeiten - Wunschzeiten auswählen} (siehe \figref{fig:meine_veranstaltungen_vplan_add_wunschzeiten}).
\begin{figure}[h!]
    \centering
    \includegraphics[width=0.75\linewidth]{img/MeineVeranstaltungen//MeinVeranstaltungsplan_Add_Wunschzeiten.png}
    \caption{Meine Veranstaltungen - Wunschzeiten konfigurieren}
    \label{fig:meine_veranstaltungen_vplan_add_wunschzeiten}
\end{figure}
Hier sollte zunächst überprüft werden, dass unter \textit{Ausgewählte Veranstaltung} die Veranstaltung steht, die angeboten werden soll.
Darunter befinden sich die Veranstaltungsteile der Veranstaltung mit weiteren Konfigurationsmöglichkeiten aufgelistet.
Wenn Sie den Namen der Veranstaltung oder die Art oder Dauer der Teile ändern möchten (zum Beispiel statt 3 Stunden Vorlesung nur 2 Stunden Vorlesung anbieten wollen), kann dies wie in Kapitel \ref{sec:meine_veranstaltungen_teile} beschrieben geändert werden.
Im folgenden wird der Konfigurationsprozess am Beispiel von Vorlesungen, Übungen und Zentralübungen beschrieben.

\subsubsection{Vorlesungen konfigurieren}
Für den Veranstaltungsteil \textit{Vorlesung} können Wunschzeiten ausgewählt werden (siehe \figref{fig:meine_veranstaltungen_vplan_add_wunschzeiten_v}).
\begin{figure}[h!]
    \centering
    \includegraphics[width=0.75\linewidth]{img/MeineVeranstaltungen//MeinVeranstaltungsplan_Add_Wunschzeiten_Vorlesung.png}
    \caption{Vorlesungen konfigurieren}
    \label{fig:meine_veranstaltungen_vplan_add_wunschzeiten_v}
\end{figure}
Dabei können die vom Planer definierten Zeitblöcke mit einem Klick ausgewählt und abgewählt werden.
Ausgewählte Wunschzeiten werden durch einen blauen Haken in der entsprechenden Checkbox angezeigt.
Es können beliebig viele Wunschzeiten ausgewählt werden.
Diese werden bei der Planung beachtet, wenn dies möglich ist.
Es kann jedoch vorkommen, dass es nicht möglich ist, den Wunschzeiten zu entsprechen.
Außerdem muss die \textit{Erwartete Hörerzahl} angegeben werden, damit die Vorlesung einen Raum zugewiesen bekommen kann, der groß genug ist.
Bitte schätzen Sie die Hörerzahl, z.B. anhand der Teilnehmer der letzten Durchführung dieser Veranstaltung.
Darunter können weitere \textit{Raumanforderungen} definiert werden. 
Es kann ausgewählt werden, ob eine Tafel, ein Beamer und Fenster vorhanden sein sollen und ob die Vorlesung in der Fürstenallee oder am Campus stattfinden soll.
Falls weitere Raumanforderungen bestehen, können diese in das Textfeld \textit{Sonstige Raumanforderungen} eingetragen werden.
Dies kann zum Beispiel benutzt werden, um zu sagen, dass der Raum möglichst 6 oder mehr Tafeln haben soll, weil in der Vorlesung viel an die Tafel geschrieben wird.
Danach muss im Feld \textit{Beginnt in Semesterwoche} die Semesterwoche angegeben werden, in der die Vorlesung zum ersten Mal stattfindet.
Im Feld \textit{Sonstige Bemerkung zur Veranstaltungsplanung} können weitere Wüsche und Informationen für die Planung angegeben werden.
Dies kann benutzt werden, falls die Veranstaltung in besonderen Räumen stattfinden soll (zum Beispiel die eigenen Räume einer Fachgruppe) oder wenn die Vorlesung möglichst mindestens 2 Tage vor der Übung stattfinden soll oder falls diese Vorlesung auf keinen Fall parallel zu einer anderen spezifischen Vorlesung liegen soll.

\subsubsection{Übungen konfigurieren}
Übungen haben alle Konfigurationsparameter einer Vorlesung, jedoch zusätzlich zwei weitere Optionen (siehe \figref{fig:meine_veranstaltungen_vplan_add_wunschzeiten_u}).
\begin{figure}[h!]
    \centering
    \includegraphics[width=0.75\linewidth]{img/MeineVeranstaltungen//MeinVeranstaltungsplan_Add_Wunschzeiten_Übung.png}
    \caption{Übungen konfigurieren}
    \label{fig:meine_veranstaltungen_vplan_add_wunschzeiten_u}
\end{figure}
In das Feld \textit{Übungsanzahl insgesamt} muss eingetragen werden, wie viele Übungsgruppen es pro Woche geben soll.
Im Feld \textit{Davon selber halten} soll eingetragen werden, wie viele Übungsgruppen vom Dozenten der Vorlesung gehalten werden und nicht von Mitarbeitern oder Hilfskräften übernommen werden.
Die \textit{Erwartete Hörerzahl} bei Übungen bezieht sich auf die Anzahl der Teilnehmer pro Übungsgruppe.
Wenn es 10 Übungsgruppen bei einer Veranstaltung mit 200 Teilnehmern gibt, sollte hier also 20 angegeben werden.

\subsubsection{Zentralübungen konfigurieren}
Auch eine Zentralübung hat alle Parameter einer Vorlesung und zusätzlich weitere für Zentralübungen spezifische Optionen (siehe \figref{fig:meine_veranstaltungen_vplan_add_wunschzeiten_z}).
\begin{figure}[h!]
    \centering
    \includegraphics[width=0.75\linewidth]{img/MeineVeranstaltungen//MeinVeranstaltungsplan_Add_Wunschzeiten_Zentralübung.png}
    \caption{Zentralübungen konfigurieren}
    \label{fig:meine_veranstaltungen_vplan_add_wunschzeiten_z}
\end{figure}
Es muss ausgewählt werden, ob die Zentralübung vom Dozenten der Vorlesung selbst gehalten wird oder ob sie von einem Mitarbeiter durchgeführt wird oder ob sie in diesem Semester nicht angeboten wird.
Diese Eingabe ist wichtig, damit der Dozent der Vorlesung nicht gleichzeitig für zwei Veranstaltungen eingeplant wird und im Fall einer nicht stattfindenden Zentralübung kein Raum unnötig blockiert wird.
Außerdem kann für Zentralübungen bei \textit{Wochentakt} ausgewählt werden, ob die Zentralübung 14-tägig zweistündig oder wöchentlich einstündig gehalten werden soll.

\subsubsection{Änderungen speichern}
Um die getroffenen Eingaben zu speichern, muss unten der blaue Knopf \textit{Absenden} gedrückt werden.
Danach gelangt man wieder auf die Seite \textit{Meine Veranstaltungen}.
Hier wird jetzt die geplante Veranstaltung unter \textit{Aktuelle Veranstaltungen} angezeigt (siehe \figref{fig:meine_veranstaltungen_vplan2}).
\begin{figure}[h!]
    \centering
    \includegraphics[width=0.75\linewidth]{img/MeineVeranstaltungen//MeinVeranstaltungsplan2.png}
    \caption{Meine Veranstaltungen}
    \label{fig:meine_veranstaltungen_vplan2}
\end{figure}

\subsection{Angebotene Veranstaltung bearbeiten}
\label{sec:meine_veranstaltungen_bearbeiten}
Wenn nach dem ursprünglichen Erstellen einer angebotenen Veranstaltung noch Änderungen an dieser gemacht werden sollen, kann diese bearbeitet werden, indem die gewünschte Veranstaltung unter dem Punkt \textit{Aktuelle Veranstaltungen} angeklickt wird (zum Beispiel auf den Namen der Veranstaltung klicken wie in \figref{fig:meine_veranstaltungen_vplan2}).
Dadurch öffnet sich wieder der Dialog, mit der die Veranstaltung ursprünglich erstellt wurde.
Alle Eingaben die bisher gespeichert wurden, sind hier vorausgewählt, sodass nur Änderungen gemacht werden müssen und die Veranstaltung nicht vollständig neu konfiguriert werden muss.
Ab hier sind die Schritte wieder wie in Kapitel \ref{sec:meine_veranstaltungen_add} beschrieben durchzuführen.
Wenn Sie den Namen der Veranstaltung oder die Art oder Dauer der Teile ändern möchten (zum Beispiel statt 3 Stunden Vorlesung nur 2 Stunden Vorlesung anbieten wollen), kann dies wie in Kapitel \ref{sec:meine_veranstaltungen_teile} geändert werden.

\subsection{Veranstaltung nicht mehr anbieten}
Wenn Sie eine von Ihnen angebotene Veranstaltung doch nicht mehr anbieten wollen, drücken sie auf der Seite \textit{Meine Veranstaltungen} bei der zu löschenden Veranstaltung auf den Knopf \textit{Löschen} (siehe \figref{fig:meine_veranstaltungen_vplan2}).
\begin{warning}
    Aus Versehen gelöschte Veranstaltungen können nur direkt nach dem Löschen wiederhergestellt werden wie in Kapitel \ref{sec:meine_veranstaltungen_wiederherstellen} beschrieben wird.
\end{warning}


\subsection{Aus Versehen gelöschte Veranstaltung wiederherstellen}
\label{sec:meine_veranstaltungen_wiederherstellen}
Falls eine Veranstaltung aus Versehen gelöscht wurde, drücken Sie auf den Knopf \textit{Wiederherstellen} (siehe \figref{fig:meine_veranstaltungen_wiederherstellen}).
\begin{figure}[h!]
    \centering
    \includegraphics[width=0.75\linewidth]{img/MeineVeranstaltungen//MeinVeranstaltungsplan_löschen.png}
    \caption{Aus Versehen gelöschte Veranstaltung wiederherstellen}
    \label{fig:meine_veranstaltungen_wiederherstellen}
\end{figure}
Danach wird die vorher gelöschte Veranstaltung wieder unter \textit{Aktuelle Veranstaltungen} angezeigt und alle Wunschzeiten und andere Eingaben werden wiederhergestellt.
Diese Aktion ist nur möglich, wenn sie direkt im Anschluss an das Löschen ausgeführt wird.
Ansonsten sind alle Eingaben zu dieser Veranstaltung unwiderruflich gelöscht worden.

\subsection{Veranstaltungsteile bearbeiten}
\label{sec:meine_veranstaltungen_teile}
Um die Länge oder Art von Veranstaltungsteilen zu ändern, muss zunächst in die Ansicht \textit{Veranstaltungsplan bearbeiten - Wunschzeiten auswählen} der zu ändernden Veranstaltung gewechselt werden.
Dies ist für neu angebotene Veranstaltungen in Kapitel \ref{sec:meine_veranstaltungen_add} und für schon existierende angebotenen Veranstaltungen in Kapitel \ref{sec:meine_veranstaltungen_bearbeiten} beschrieben.
Hier muss der Knopf \textit{Veranstaltungsteile bearbeiten} gedrückt werden (siehe \figref{fig:meine_veranstaltungen_w_vteile}).
\begin{figure}[h!]
    \centering
    \includegraphics[width=0.75\linewidth]{img/MeineVeranstaltungen//MeinVeranstaltungsplan_Add_Wunschzeiten_Vteile.png}
    \caption{Veranstaltungsteile ändern}
    \label{fig:meine_veranstaltungen_w_vteile}
\end{figure}

Dadurch öffnet sich die Seite \textit{Veranstaltung hinzufügen} (siehe \figref{fig:meine_veranstaltungen_w_vteile_bearbeiten}).
\begin{figure}[h!]
    \centering
    \includegraphics[width=0.75\linewidth]{img/MeineVeranstaltungen//MeinVeranstaltungsplan_Add_Wunschzeiten_VteileBearbeiten.png}
    \caption{Veranstaltungsteile hinzufügen, löschen und bearbeiten}
    \label{fig:meine_veranstaltungen_w_vteile_bearbeiten}
\end{figure}
Oben kann der Kursname sowohl in Englisch als auch in Deutsch geändert werden und ein Kürzel für den Kurs festgelegt werden sowie der Fachbereich geändert werden.
Die Paulnummer kann nur von einem Planer oder Administrator geändert werden.
Unter dem Punkt \textit{Veranstaltungsteile} können die einzelnen Teile der Veranstaltung konfiguriert werden.
Hier können durch einen Klick auf den Knopf \textit{Hinzufügen} neue Veranstaltungsteile hinzugefügt werden, wenn diese gewünscht sind.
Es können die Art des Veranstaltungsteils (Vorlesung, Übung, Zentralübung,Praktikum, ...) festgelegt werden und wie viele Stunden dieser Teil dauern soll.
Außerdem können auch hier Bemerkungen für die Planung eingegeben werden.
Um einen Veranstaltungsteil zu entfernen, muss auf den Knopf \textit{Veranstaltungsteil entfernen} unter dem jeweiligen Veranstaltungsteil gedrückt werden.
Um die Änderungen zu speichern, muss unten auf den blauen Knopf \textit{Absenden} gedrückt werden (siehe \figref{fig:meine_veranstaltungen_w_vteile_bearbeiten2}).
\begin{figure}[h!]
    \centering
    \includegraphics[width=0.75\linewidth]{img/MeineVeranstaltungen//MeinVeranstaltungsplan_Add_Wunschzeiten_VteileBearbeiten2.png}
    \caption{Veranstaltungsteile hinzufügen, löschen und bearbeiten}
    \label{fig:meine_veranstaltungen_w_vteile_bearbeiten2}
\end{figure}
Danach können die Änderungen direkt auf der Seite \textit{Veranstaltungsplan bearbeiten - Wunschzeiten auswählen} gesehen werden.

\subsection{Angebotene Veranstaltung aus vergangenen Semestern kopieren}
\label{sec:meine_veranstaltungen_copy}
Wenn Sie eine Veranstaltung schon in einem früheren Semester angeboten haben, können Sie ihre Eingaben von damals übernehmen lassen und müsssen nicht alles neu eingeben.
Dafür muss unter dem Punkt \textit{Vergangene Veranstaltungen} das Semester ausgewählt werden, aus dem die Veranstaltung kopiert werden soll und anschließend der \textit{Auswählen} Knopf gedrückt werden.
Anschließend erscheint eine Liste mit Veranstaltungen, die Sie in diesem Semester gehalten haben.
Um nun eine Veranstaltung in das aktuell ausgewählte Semester zu kopieren, muss der \textit{Kopieren} Knopf gedrückt werden (siehe \figref{fig:meine_veranstaltungen_copy}).
Anschließend erscheint die kopierte Veranstaltung unter \textit{Aktuelle Veranstaltungen} und sie kann angeklickt werden, um die Inhalte zu betrachten oder Änderungen vorzunehmen.

\begin{figure}[h!]
    \centering
    \includegraphics[width=0.75\linewidth]{img/MeineVeranstaltungen//MeinVeranstaltungsplan_Copy.png}
    \caption{Veranstaltungen aus vergangenen Semestern kopieren}
    \label{fig:meine_veranstaltungen_copy}
\end{figure}

\subsection{Angebotene Veranstaltungen für andere Benutzer erstellen (als Admin/Planer)}
Um als Planer oder Administrator für andere Benutzer Lehrveranstaltungen zu konfigurieren, muss auf die Seite \textit{Aktueller Vorlesungsplan} gewechselt werden und dort auf \textit{Hinzufügen} geklickt werden (siehe \figref{fig:veranstaltung_admin1}).
Danach erscheint ein ähnlicher Dialog wie wenn man Veranstaltungen für sich selbst erstellt.
Der einzige Unterschied ist, dass am Anfang ausgewählt werden muss, welcher Veranstalter die Veranstaltung halten soll (siehe \figref{fig:veranstaltung_admin2}).
Danach läuft alles wie in Kapitel \ref{sec:meine_veranstaltungen_add} beschrieben.
\begin{figure}[h!]
    \centering
    \includegraphics[width=0.75\linewidth]{img//MeineVeranstaltungen//Admin//MeinVeranstaltungsplan_Add_Admin1.png}
    \caption{Veranstaltungen für andere Benutzer anlegen}
    \label{fig:veranstaltung_admin1}
\end{figure}
\begin{figure}[h!]
    \centering
    \includegraphics[width=0.75\linewidth]{img//MeineVeranstaltungen//Admin//MeinVeranstaltungsplan_Add_Admin2.png}
    \caption{Veranstaltungen für andere Benutzer anlegen}
    \label{fig:veranstaltung_admin2}
\end{figure}

\subsection{Angebotene Veranstaltungen für andere Benutzer ändern oder löschen oder den Veranstalter einer Veranstaltung ändern (als Admin/Planer)}
Um als Planer oder Administrator für andere Benutzer Lehrveranstaltungen zu konfigurieren, muss auf die Seite \textit{Aktueller Vorlesungsplan} gewechselt werden.
Falls es gewünscht ist, eine angebotene Veranstaltung zu löschen, kann man dies durch Klicken des Knopfes \textit{Löschen} in der Zeile, in der die zu löschende angebotene Veranstaltung steht (siehe \figref{fig:veranstaltung_admin4}).
\begin{figure}[h!]
    \centering
    \includegraphics[width=0.75\linewidth]{img//MeineVeranstaltungen//Admin//MeinVeranstaltungsplan_Add_Admin4.png}
    \caption{Veranstaltungen für andere Benutzer löschen}
    \label{fig:veranstaltung_admin4}
\end{figure}

Um eine angebotene Veranstaltung zu bearbeiten oder den Veranstalter zu ändern, muss auf den Titel der zu ändernden Veranstaltung geklickt werden.
Danach erscheint ein Menu, in dem der Veranstalter geändert werden kann (siehe \figref{fig:veranstaltung_admin3}).
Änderungen an den Wunschzeiten, Hörerzahlen, usw. können wie in \ref{sec:meine_veranstaltungen_add} beschrieben wurde vorgenommen werden.


\begin{figure}[h!]
    \centering
    \includegraphics[width=0.75\linewidth]{img//MeineVeranstaltungen//Admin//MeinVeranstaltungsplan_Add_Admin3.png}
    \caption{Veranstaltungen für andere Benutzer bearbeiten}
    \label{fig:veranstaltung_admin3}
\end{figure}


\section{Sperrzeiten}
\label{sec:sperrzeiten}
\subsection{Eigene Sperrzeiten festlegen}
Auf der Seite \textit{Sperrzeiten} können Zeiten angegeben werden, in denen für einen Veranstalter keine Veranstaltungen geplant werden sollen.
Hier kann für jeden Tag und jede Stunde durch Klicken auf die Checkbox ausgewählt werden, dass diese Zeit nicht zur Verfügung steht (siehe \figref{fig:sperrzeiten}).
Nach der Eingabe von Zeiten müssen diese durch einen Klick auf den blauen Knopf \textit{Speichern} gespeichert werden.
\begin{figure}[h!]
    \centering
    \includegraphics[width=0.75\linewidth]{img//Sperrzeiten//Sperrzeiten.png}
    \caption{Sperrzeiten erstellen}
    \label{fig:sperrzeiten}
\end{figure}

\subsection{Sperrzeiten für andere Benutzer festlegen (als Admin/Planer)}
Falls der Planer oder Administrator Sperrzeiten für einen bestimmten Benutzer nachtragen muss, ist dies auch möglich.
Dafür muss man als Planer oder Administrator auf die Seite \textit{Aktueller Vorlesungsplan} wechseln (siehe \figref{fig:sperrzeiten_admin1}) und dort \textit{Sperrzeiten bearbeiten} klicken.
\begin{figure}[h!]
    \centering
    \includegraphics[width=0.75\linewidth]{img//Sperrzeiten//Sperrzeiten_Admin1.png}
    \caption{Sperrzeiten als Planer für andere Benutzer erstellen}
    \label{fig:sperrzeiten_admin1}
\end{figure}

Anschließend muss der Veranstalter ausgewählt werden, für den die Sperrzeiten festgelegt werden sollen (siehe \figref{fig:sperrzeiten_admin2})).
\begin{figure}[h!]
    \centering
    \includegraphics[width=0.75\linewidth]{img//Sperrzeiten//Sperrzeiten_Admin2.png}
    \caption{Sperrzeiten als Planer für andere Benutzer erstellen}
    \label{fig:sperrzeiten_admin2}
\end{figure}

Danach können die Sperrzeiten bearbeitet werden und diese durch einen Klick auf den blauen Knopf \textit{Speichern} gespeichert werden (siehe \figref{fig:sperrzeiten_admin3}).
\begin{figure}[h!]
    \centering
    \includegraphics[width=0.75\linewidth]{img//Sperrzeiten//Sperrzeiten_Admin3.png}
    \caption{Sperrzeiten als Planer für andere Benutzer erstellen}
    \label{fig:sperrzeiten_admin3}
\end{figure}


\section{Veranstaltungen unserer Fakultät}
\label{sec:vfak}

\subsection{Lehrveranstaltung anlegen}

\begin{warning}
    Alle in diesem Kapitel beschrieben Aktionen werden im aktuell ausgewählten und in allen neueren Semestern ausgeführt.
\end{warning}

\label{sec:lv_anlegen}
Bevor eine Lehrveranstaltung von einem Lehrenden gehalten werden kann, muss diese Lehrveranstaltung angelegt werden.
Es müssen zum Beispiel der Name, die Paulnummer und die Lehrveranstaltungsteile spezifiziert werden.
Dafür muss auf die Seite \textit{Veranstaltungen unserer Fakultät} gewechselt werden, wo eine Liste mit allen bereits angelegten Veranstaltungen angelegt wird und auf den \textit{Hinzufügen} Knopf gedrückt werden (siehe \figref{fig:veranstaltung_anlegen1}).
\begin{figure}[h!]
    \centering
    \includegraphics[width=0.75\linewidth]{img//VeranstaltungenDerFak//VeranstaltungAnlegen1.png}
    \caption{Veranstaltungen anlegen}
    \label{fig:veranstaltung_anlegen1}
\end{figure}
Danach öffnet sich die Seite \textit{Veranstaltung hinzufügen} (siehe \figref{fig:veranstaltung_anlegen2}).
\begin{figure}[h!]
    \centering
    \includegraphics[width=0.75\linewidth]{img//VeranstaltungenDerFak//VeranstaltungAnlegen2.png}
    \caption{Veranstaltungen anlegen}
    \label{fig:veranstaltung_anlegen2}
\end{figure}
Hier müssen für den neuen Kurs der Kursname auf Deutsch (mindestens 3 Zeichen lang), der Kursname auf Englisch (mindestens 3 Zeichen lang), ein Kürzel, ein Fachbereich und eine Paulnummer angegeben werden (genauer beschrieben in Kapitel \ref{sec:lv_anlegen_paul}).
Unter dem Punkt \textit{Veranstaltungsteile} können die einzelnen Teile der Veranstaltung konfiguriert werden.
Hier können durch einen Klick auf den Knopf \textit{Hinzufügen} neue Veranstaltungsteile hinzugefügt werden, wenn diese gewünscht sind.
Es können die Art des Veranstaltungsteils (Vorlesung, Übung, Zentralübung, Praktikum, ...) festgelegt werden und wie viele Stunden dieser Teil dauern soll.
Außerdem können auch hier Bemerkungen für die Planung eingegeben werden.
Um einen Veranstaltungsteil zu entfernen, muss auf den Knopf \textit{Veranstaltungsteil entfernen} unter dem jeweiligen Veranstaltungsteil gedrückt werden.
Um die Änderungen zu speichern, muss unten auf den blauen Knopf \textit{Absenden} gedrückt werden.

Anschließend kann die angelegte Veranstaltung auf der Seite \textit{Veranstaltungen unserer Fakultät} in der Liste mit Veranstaltungen gesehen werden (siehe \figref{fig:veranstaltung_anlegen3}).
\begin{figure}[h!]
    \centering
    \includegraphics[width=0.75\linewidth]{img//VeranstaltungenDerFak//VeranstaltungAnlegen3.png}
    \caption{Angelegte Veranstaltungen ansehen}
    \label{fig:veranstaltung_anlegen3}
\end{figure}

\subsubsection{Paulnummern}
\label{sec:lv_anlegen_paul}
Paulnummern bestehen aus einer Zeichenfolge mit dem Schema \texttt{L.XXX.XXXXX}, wobei X jeweils eine Ziffer von 0 bis 9 ist.
Für die Informatik fangen Paulnummern immer mit \texttt{L.079.} an.
Das WebTool ist eigenständig in der Lage, eine freie Paulnummer innerhalb eines bestimmten Bereichs zu finden, wenn eine freie Paulnummer existiert.
Die Eingabe in das Feld stellt einen Präfix dar, den das Tool auf jeden Fall einhält.
Eine genaue Paulnummer kann festgelegt werden, indem eine Nummmer mit dem Schema \texttt{L.XXX.XXXXX} eingegeben wird.
Falls die genaue Paulnummer nicht schon festgelegt ist, sondern nur mit einem bestimmten Präfix anfangen soll, zum Beispiel \texttt{L.079.05}, erkennt das Tool automatisch, dass die letzten drei Ziffern noch offen sind und sucht eine freie Nummer, die mit dem gegebenen Präfix startet.
In diesem Fall sind noch 1000 Paulnummern mit dem Präfix offen \texttt{L.079.05000} - \texttt{L.079.05999}.
Das Tool sucht eine freie Nummer mit dem gegebenen Präfix nach dem folgenden Algorithmus aus:
\begin{itemize}
    \item Nummern, die noch nie benutzt wurden
    \item Die Nummer, die am längsten nicht benutzt wurde (für eine \textbf{angelegte} Veranstaltung)
\end{itemize}
Falls keine Nummer gefunden wurde, die benutzt werden kann (Paulnummern können nicht zweimal pro Semester verwendet werden), wird ein Fehler angezeigt (siehe \figref{fig:veranstaltung_anlegen4}).
\begin{figure}[h!]
    \centering
    \includegraphics[width=0.75\linewidth]{img//VeranstaltungenDerFak//VeranstaltungAnlegen4.png}
    \caption{Fehlermeldung, wenn es keine freie Paulnummer gibt}
    \label{fig:veranstaltung_anlegen4}
\end{figure}
Sie beinhaltet den Namen der Veranstaltung, die eine Nummer mit dem gegebenen Präfix enthält und möglichst lange nicht \textbf{angeboten} wurde.
Der Name der anderen Veranstaltung kann angeklickt werden, um diese Veranstaltung direkt zu bearbeiten und die Nummer dieser Veranstaltung ändern zu können, um sie für neue Veranstaltungen frei zu machen.
Das Ändern von Veranstaltungen funktioniert genauso wie das Anlegen von Veranstaltungen.

\subsection{Lehrveranstaltung bearbeiten}
Um eine bereits angelegte Veranstaltung zu bearbeiten, muss vom Ersteller der Lehrveranstaltung auf die Seite \textit{Veranstaltungen unserer Fakultät} gewechselt werden und dort auf die zu bearbeitende Lehrveranstaltung geklickt werden (siehe \figref{fig:veranstaltung_anlegen6}).
\begin{figure}[h!]
    \centering
    \includegraphics[width=0.75\linewidth]{img//VeranstaltungenDerFak//VeranstaltungAnlegen6.png}
    \caption{Angelegte Veranstaltungen bearbeiten}
    \label{fig:veranstaltung_anlegen6}
\end{figure}

Danach läuft alles so wie beim ersten Anlegen einer Lehrveranstaltung, wie in Kapitel \ref{sec:lv_anlegen} beschrieben.

\subsection{Lehrveranstaltung löschen}
Um eine bereits angelegte Veranstaltung zu bearbeiten, muss vom Ersteller der Lehrveranstaltung auf die Seite \textit{Veranstaltungen unserer Fakultät} gewechselt werden und dort auf den Knopf \textit{Löschen} neben der zu bearbeitenden Lehrveranstaltung geklickt werden (siehe \figref{fig:veranstaltung_anlegen5}).

\begin{figure}[h!]
    \centering
    \includegraphics[width=0.75\linewidth]{img//VeranstaltungenDerFak//VeranstaltungAnlegen5.png}
    \caption{Lehrveranstaltung löschen}
    \label{fig:veranstaltung_anlegen5}
\end{figure}


\section{Datenexport}
\label{sec:datenexport}
Um die aktuelle Datenbank aus dem WebTool herunterzuladen, um danach mit dem Sven Tool zu planen, muss auf die Seite \textit{Daten exportieren} gewechselt werden (siehe \figref{fig:export}).
Dort muss der Knopf \textit{Aktuelle Datenbank herunterladen} geklickt werden und anschließend wird die Datenbank heruntergeladen.
Standardmäßig ist die Datenbank nach dem Namen ihres Semesters benannt. 
Im Sommersemester 2025 wäre der Name \texttt{VLPTDB\_Sommersemester2025.db} und im Wintersemester 2024/2025 wäre er \texttt{VLPTDB\_Wintersemester20242025.db}.
\begin{figure}[h!]
    \centering
    \includegraphics[width=0.75\linewidth]{img//Export//Datenexport.png}
    \caption{Daten exportieren}
    \label{fig:export}
\end{figure}




\section{Planungsstand importieren}
\label{sec:import}
Um die geplanten Veranstaltungen vom Sven Tool in das WebTool zu importieren, muss die Seite \textit{Planungsstand importieren/exportieren} aufgerufen werden.
\begin{warning}
    Die hochgeladenen Datenbanken werden für jedes Semester einzeln gespeichert.
\end{warning}
\begin{error}
    Wenn bereits eine Datei mit dem gleichen Namen vorhanden ist, wird diese beim Hochladen ohne weitere Warnmeldung von der hochgeladenen Datei überschrieben.
\end{error}

Um die Datenbank aus dem Sven Tool hochzuladen, muss zuerst auf das Feld \textit{Datei auswählen} gedrückt werden (siehe \figref{fig:import1})
\begin{figure}[h!]
    \centering
    \includegraphics[width=0.75\linewidth]{img//Import//Import1.png}
    \caption{Daten importieren}
    \label{fig:import1}
\end{figure}
Danach öffnet sich der Dialog des Betriebssystems, um eine Datei auszuwählen (siehe \figref{fig:import2}).
Hier muss die gewünschte Datenbank gefunden und mit einem Doppelklick ausgewählt werden.
\begin{figure}[h!]
    \centering
    \includegraphics[width=0.75\linewidth]{img//Import//Import2.png}
    \caption{Daten importieren - Datei auswählen}
    \label{fig:import2}
\end{figure}
Anschließend erscheint der Dateiname der ausgewählten Datei im Feld neben \textit{Datei auswählen}.
Jetzt muss der blaue Knopf \textit{Upload} gedrückt werden, um die Datei hochzuladen (siehe \figref{fig:import3}).
\begin{figure}[h!]
    \centering
    \includegraphics[width=0.75\linewidth]{img//Import//Import3.png}
    \caption{Daten importieren - Hochladen}
    \label{fig:import3}
\end{figure}

Nachdem der Upload beendet ist, erscheint die Datei in der Liste mit Planungsständen (siehe \figref{fig:import4}) und kann durch einen Klick auf \textit{Download} heruntergeladen oder durch einen Klick auf \textit{Löschen} gelöscht werden.
\begin{figure}[h!]
    \centering
    \includegraphics[width=0.75\linewidth]{img//Import//Import4.png}
    \caption{Daten importieren - Löschen und herunterladen}
    \label{fig:import4}
\end{figure}


\section{Benutzerverwaltung}
\label{sec:benutzer}

\begin{warning}
    Login Daten gelten semesterübergreifend. Rollen werden im aktuellen und in allen neueren Semestern geändert.
\end{warning}

\subsection{Benutzer erstellen}
Um neue Benutzer im WebTool zu erstellen, muss auf die Seite \textit{Benutzerverwaltung} gewechselt und  der Knopf 
\textit{Erstellen} geklickt werden (siehe \figref{fig:benutzer1}).
\begin{figure}[h!]
    \centering
    \includegraphics[width=0.75\linewidth]{img//Benutzer//Benutzer1.png}
    \caption{Benutzer erstellen}
    \label{fig:benutzer1}
\end{figure}
Hier muss ein Name für die neue Person erstellt werden, ein Kürzel festgelegt werden (wird automatisch erzeugt, kann aber verändert werden) und die Fakultät der neuen Person angegeben werden.
Anschließend muss die Rolle der Person ausgewählt werden, indem das Dropdown Menu \textit{Rollen} geklickt wird (siehe \figref{fig:benutzer2}).
\begin{figure}[h!]
    \centering
    \includegraphics[width=0.75\linewidth]{img//Benutzer//Benutzer2.png}
    \caption{Benutzer erstellen}
    \label{fig:benutzer2}
\end{figure}
In diesem Menu können eine oder mehrere Rollen ausgewählt werden (siehe \figref{fig:benutzer3}).
Die möglichen Rollen sind:
\begin{itemize}
    \item Veranstalter
    \item Veranstaltungsbeauftragter
    \item Modulhandbuchbeauftragter
    \item Administrator
    \item Planer
    \item Beobachter
\end{itemize}
\begin{figure}[h!]
    \centering
    \includegraphics[width=0.75\linewidth]{img//Benutzer//Benutzer3.png}
    \caption{Benutzer erstellen}
    \label{fig:benutzer3}
\end{figure}
Danach können die Eingaben mit einem Klick auf den blauen Knopf \textit{Speichern} gespeichert werden (siehe \figref{fig:benutzer4}).
\begin{figure}[h!]
    \centering
    \includegraphics[width=0.75\linewidth]{img//Benutzer//Benutzer4.png}
    \caption{Benutzer erstellen}
    \label{fig:benutzer4}
\end{figure}
Anschließend gelangt man in die Liste mit allen existierenden Benutzern.

\subsection{Benutzer bearbeiten}
Um einen Benutzer zu bearbeiten, muss auf diesen in der Liste mit allen Benutzern geklickt werden (siehe \figref{fig:benutzer5}).
\begin{figure}[h!]
    \centering
    \includegraphics[width=0.75\linewidth]{img//Benutzer//Benutzer5.png}
    \caption{Benutzer bearbeiten}
    \label{fig:benutzer5}
\end{figure}
Dann öffnet sich das gleiche Fenster wie beim Erstellen von Benutzern und die Eingaben können wie beim Erstellen mit einem Klick auf \textit{Speichern} gespeichert werden (siehe \figref{fig:benutzer6}).
\begin{figure}[h!]
    \centering
    \includegraphics[width=0.75\linewidth]{img//Benutzer//Benutzer6.png}
    \caption{Benutzer bearbeiten}
    \label{fig:benutzer6}
\end{figure}

\subsection{Login erstellen}
\label{sec:benutzer:create}
Um einen Benutzernamen und ein Passwort für die Benutzer zu erstellen, damit sich diese ins WebTool einloggen können, muss der Benutzer bearbeitet werden und dabei auf den Knopf \textit{Erstellen} unter der Überschrift \textit{Benutzer} gedrückt werden  (siehe \figref{fig:benutzer7}).
Ein Benutzer kann mehrere Benutzernamen und Passwörter haben. 
Dies ist sinnvoll, wenn ein Sekretariat die Eintragungen vornehmen soll und dafür ein eigenes Passwort bekommen soll.
\begin{figure}[h!]
    \centering
    \includegraphics[width=0.75\linewidth]{img//Benutzer//Benutzer7.png}
    \caption{Login erstellen}
    \label{fig:benutzer7}
\end{figure}
Hier muss eine Benutzerbezeichnung festgelegt werden sowie ein Benutzername und Passwort für den Benutzer erstellt werden.
Die Eingaben werden mit einem Klick auf den blauen Knopf \textit{Speichern} gepeichert (siehe \figref{fig:benutzer8}).
\begin{figure}[h!]
    \centering
    \includegraphics[width=0.75\linewidth]{img//Benutzer//Benutzer8.png}
    \caption{Login erstellen}
    \label{fig:benutzer8}
\end{figure}

\subsection{Login ändern}
Um die Login Daten zu ändern, muss auf den Benutzernamen geklickt werden (siehe \figref{fig:benutzer9}).
Um Login Daten zu löschen, muss auf den Knopf \textit{Löschen} gedrückt werden.
Danach können der Benutzername und das Passwort wie in Kapitel \ref{sec:benutzer:create} verändert und gespeichert werden.
\begin{figure}[h!]
    \centering
    \includegraphics[width=0.75\linewidth]{img//Benutzer//Benutzer9.png}
    \caption{Login ändern}
    \label{fig:benutzer9}
\end{figure}

\subsection{Benutzer löschen}
Wenn ein Benutzer gelöscht werden soll, etwa wenn die Person die Universität verlässt, kann der Benutzer durch einen Klick auf \textit{Löschen} in der Benutzerverwaltung gelöscht werden (siehe \figref{fig:benutzer10}).
\begin{figure}[h!]
    \centering
    \includegraphics[width=0.75\linewidth]{img//Benutzer//Benutzer10.png}
    \caption{Benutzer löschen}
    \label{fig:benutzer10}
\end{figure}


\section{Meine Modulhandbücher}
\label{sec:modulhandbuch}

Um Modulhandbücher und Druckrubriken im Web Tool anzulegen und zu ändern, muss auf die Seite \textit{Meine Modulhandbücher} gewechselt werden (siehe \figref{fig:m0}).
Alle Änderungen an Modulhandbüchern werden im aktuell ausgewählten Semester sowie in allen neueren Semestern umgesetzt.
\begin{figure}[h!]
\centering
\includegraphics[width=0.75\linewidth]{img//Modulhandbuch//M0.png}
\caption{Modulhandbuch}
\label{fig:m0}
\end{figure}

\subsection{Druckrubriken verwalten}
Um Druckrubriken zu verwalten, muss zuerst der Knopf \textit{Druckrubriken verwalten} gedrückt werden (siehe \figref{fig:m1}).

\begin{figure}[h!]
\centering
\includegraphics[width=0.75\linewidth]{img//Modulhandbuch//M1.png}
\caption{Modulhandbuch}
\label{fig:m1}
\end{figure}

\subsubsection{Druckrubrik hinzufügen}
\begin{warning}
    Das Erstellen von mehreren Druckrubriken mit der gleichen Bezeichnung führt zu unerwünschtem Verhalten. Ein versuchtes Erstellen von doppelten Druckrubriken führt zu einer Fehlermeldung.
\end{warning}
Um jetzt eine Druckrubrik hinzuzufügen, muss der Knopf \textit{Druckrubrik hinzufügen} gedrückt werden (siehe \figref{fig:m2}).

\begin{figure}[h!]
\centering
\includegraphics[width=0.75\linewidth]{img//Modulhandbuch//M2.png}
\caption{Modulhandbuch}
\label{fig:m2}
\end{figure}

Zuerst muss die Bezeichnung und die Oberkategorie der neuen Druckrubrik festgelegt werden (siehe \figref{fig:m3}).
\begin{figure}[h!]
\centering
\includegraphics[width=0.75\linewidth]{img//Modulhandbuch//M3.png}
\caption{Modulhandbuch}
\label{fig:m3}
\end{figure}

Die Auswahlmöglichkeiten sind die Bezeichnungen alle schon existierenden Druckrubriken und Unterdruckrubriken sowie \textit{Keine} (siehe \figref{fig:m4}).
\begin{figure}[h!]
\centering
\includegraphics[width=0.75\linewidth]{img//Modulhandbuch//M4.png}
\caption{Modulhandbuch}
\label{fig:m4}
\end{figure}

Falls die Druckrubrik keine Oberkategorie haben soll, muss \textit{Keine} ausgewählt werden.
Falls die neue Druckrubrik die Unterkategorie einer schon existierenden Druckrubrik werden soll, muss diese hier ausgewählt werden.
Das Feld Reihenfolge wird aktuell nicht benutzt.
Es kann demnach einfach so gelassen werden, wie es ist.
Anschließend müssen die Eingaben mit einem Klick auf den blauen Knopf \textit{Speichern} gespeichert werden (siehe \figref{fig:m5}).
\begin{figure}[h!]
\centering
\includegraphics[width=0.75\linewidth]{img//Modulhandbuch//M5.png}
\caption{Modulhandbuch}
\label{fig:m5}
\end{figure}

\subsubsection{Unterdruckrubrik direkt erstellen}
Um direkt danach eine Unterdruckrubrik der eben erstellen Druckrubrik zu erstellen, kann auf den Knopf \textit{Druckrubrik hinzufügen} geklickt werden (siehe \figref{fig:m6}).
\begin{figure}[h!]
\centering
\includegraphics[width=0.75\linewidth]{img//Modulhandbuch//M6.png}
\caption{Modulhandbuch}
\label{fig:m6}
\end{figure}

Dann öffnet sich der gleiche Dialog wie beim Erstellen einer neuen Druckrubrik mit der Ausnahme, dass die Oberkategorie schon auf die aktuell ausgewählte Druckrubrik festgelegt ist.
Auch hier muss mit einem Klick auf \textit{Speichern} gepeichert werden (siehe \figref{fig:m7}).
\begin{figure}[h!]
\centering
\includegraphics[width=0.75\linewidth]{img//Modulhandbuch//M7.png}
\caption{Modulhandbuch}
\label{fig:m7}
\end{figure}

\subsubsection{Navigieren innerhalb von Druckrubriken - Löschen und Bearbeiten}
Um von einer Unterdruckrubrik direkt zu der darüberliegenden Elterndruckrubrik zu gelangen, kann auf den blau unterstrichenen Namen der Oberkategorie gedrückt werden (siehe \figref{fig:m8}).
\begin{figure}[h!]
\centering
\includegraphics[width=0.75\linewidth]{img//Modulhandbuch//M8.png}
\caption{Modulhandbuch}
\label{fig:m8}
\end{figure}

Alle Unterdruckrubriken der aktuell ausgewählten Druckrubrik werden unten in der Druckrubrikansicht angezeigt.
Um diese zu bearbeiten, muss auf sie geklickt werden und um sie zu löschen, muss auf den Knopf \textit{Löschen} neben der zu löschenden Unterdruckrubrik geklickt werden (siehe \figref{fig:m9}).
\begin{figure}[h!]
\centering
\includegraphics[width=0.75\linewidth]{img//Modulhandbuch//M9.png}
\caption{Modulhandbuch}
\label{fig:m9}
\end{figure}

\subsection{Modulhandbuch}

\subsubsection{Modulhandbuch hinzufügen}
Um ein neues Modulhandbuch hinzuzufügen, muss auf den Knopf \textit{Hinzufügen} auf der Seite \textit{Meine Modulhandbücher} geklickt werden (siehe \figref{fig:m10}).
\begin{figure}[h!]
\centering
\includegraphics[width=0.75\linewidth]{img//Modulhandbuch//M10.png}
\caption{Modulhandbuch}
\label{fig:m10}
\end{figure}

Anschließend öffnet sich ein Dialog, in den der Name des Hauptfachs, der Name des Nebenfachs, das Fachsemester, die Prüfungsordnungsversion und ein Kürzel eingetragen werden muss.
Dieses dient der Identifikation der modellierten Modulhandbücher sowie der dazugehörigen Prüfungsordnungen.
Die Eingaben müssen durch einen Klick auf den blauen Knopf \textit{Speichern} gespeichert werden (siehe \figref{fig:m11}).
\begin{figure}[h!]
\centering
\includegraphics[width=0.75\linewidth]{img//Modulhandbuch//M11.png}
\caption{Modulhandbuch}
\label{fig:m11}
\end{figure}

\subsubsection{Module zu Modulhandbuch hinzufügen}
\label{sec:mod_create}
Um Module zu einem Modulhandbuch hinzuzufügen, muss auf den Knopf \textit{Modul hinzufügen} geklickt werden (siehe \figref{fig:m12}).
\begin{figure}[h!]
\centering
\includegraphics[width=0.75\linewidth]{img//Modulhandbuch//M12.png}
\caption{Modulhandbuch}
\label{fig:m12}
\end{figure}

Dort muss zunächst der Fachbereich des zu erstellenden Moduls ausgewählt werden und dann der Kurs ausgewählt werden, indem auf das Dropdown \textit{Kurs auswählen} gedrückt wird (siehe \figref{fig:m13}).
\begin{figure}[h!]
\centering
\includegraphics[width=0.75\linewidth]{img//Modulhandbuch//M13.png}
\caption{Modulhandbuch}
\label{fig:m13}
\end{figure}

Dort erscheint eine Liste mit allen Kursen in dem ausgewählten Fachbereich (siehe \figref{fig:m14}).
\begin{figure}[h!]
\centering
\includegraphics[width=0.75\linewidth]{img//Modulhandbuch//M14.png}
\caption{Modulhandbuch}
\label{fig:m14}
\end{figure}

Danach muss der Modulname eingegeben werden, soweit er vom Kursnamen abweicht und die ECTS festgelegt werden.
Bei \textit{Pflichtfach} soll ein Haken gesetzt werden, falls das Modul ein Pflichtmodul ist, und kein Haken gesetzt werden, wenn das Modul nicht verpflichtend ist.
Anschließend muss auf den blauen Knopf \textit{Nächster Schritt} gedrückt werden, um das Modul weiter zu konfigurieren (siehe \figref{fig:m15}).
\begin{figure}[h!]
\centering
\includegraphics[width=0.75\linewidth]{img//Modulhandbuch//M15.png}
\caption{Modulhandbuch}
\label{fig:m15}
\end{figure}

Hier kann eingestellt werden, welche Veranstaltungsteile des Kurses für dieses Modul relevant sind.
Falls in diesem Modul zum Beispiel nur die Vorlesung gehört werden soll, die Übung jedoch nicht, muss der Veranstaltungsteil \textit{Übung} im rechten Feld \textit{Aktive Veranstaltungsteile} ausgewählt werden und dann mit einem Klick auf den Pfeil nach links zu den ausgeschlossenen Veranstaltungsteilen bewegt werden (siehe \figref{fig:m16}).
Falls Veranstaltungsteile wieder aktiviert werden sollen, können diese markiert werden und analog mit dem Pfeil nach rechts wieder zu den aktiven Veranstaltungsteilen geschoben werden.

\begin{figure}[h!]
\centering
\includegraphics[width=0.75\linewidth]{img//Modulhandbuch//M16.png}
\caption{Modulhandbuch}
\label{fig:m16}
\end{figure}

Mit dem gleichen Prozess können Druckrubriken für ein Modul festgelgt werden.
Die Eingaben werden danach mit einem Klick auf den blauen Knopf \textit{Speichern} gespeichert.
\begin{figure}[h!]
\centering
\includegraphics[width=0.75\linewidth]{img//Modulhandbuch//M17.png}
\caption{Modulhandbuch}
\label{fig:m17}
\end{figure}

\subsubsection{Module bearbeiten und löschen}
Um ein Modul in einem Modulhanbuch zu bearbeiten, muss auf den Namen des entsprechenden Moduls geklickt werden. Wenn es gelöscht werden soll, muss stattdessen auf den danabenliegenden Knopf \textit{Löschen} gedrückt werden (siehe \figref{fig:m18}).
\begin{figure}[h!]
\centering
\includegraphics[width=0.75\linewidth]{img//Modulhandbuch//M18.png}
\caption{Modulhandbuch}
\label{fig:m18}
\end{figure}

\subsubsection{Modulhandbücher bearbeiten, löschen, kopieren, aktivieren und deaktivieren}
Um ein Modulhandbuch zu bearbeiten, muss in die Zeile des Modulhandbuchs geklickt werden (siehe \figref{fig:m19}).
Dann kann es wie beim Erstellen bearbeitet werden (wie in Kapitel \ref{sec:mod_create} beschrieben wurde).
Um ein Modulhandbuch zu kopieren, muss der Knopf \textit{Kopieren} gedrückt werden.
Damit wird eine exakte Kopie erstellt, was nützlich ist, wenn eine neue Prüfungsordnungsversion nur kleine Änderungen zur alten Prüfungsordnungsversion enthält und nicht alle Module neu modelliert werden müssen.
Um ein Modulhandbuch zu löschen, muss auf den Knopf \textit{Löschen} neben dem zu löschenden Modulhandbuch gedrückt werden.
Um ein Modulhandbuch zu aktivieren, muss der Knopf \textit{Aktivieren} gedrückt werden.
Danach erscheint das Modulhandbuch in der Kategorie \textit{Aktive Studiengänge}.

\begin{figure}[h!]
\centering
\includegraphics[width=0.75\linewidth]{img//Modulhandbuch//M19.png}
\caption{Modulhandbuch}
\label{fig:m19}
\end{figure}

Um ein Modulhandbuch zu deaktivieren, muss auf den Knopf \textit{Deaktivieren} in der Zeile des zu deaktivierenden Modulhandbuchs gedrückt werden (siehe \figref{fig:m21}).
\begin{figure}[h!]
\centering
\includegraphics[width=0.75\linewidth]{img//Modulhandbuch//M21.png}
\caption{Modulhandbuch}
\label{fig:m21}
\end{figure}

\subsubsection{Modulhandbuch wiederherstellen}
Falls ein Modulhandbuch unabsichtlich gelöscht wurde, kann es im direkten Anschluss durch einen Klick auf den Knopf \textit{Wiederherstellen} wiederhergestellt werden (siehe \figref{fig:m20}).
Dies muss jedoch direkt nach dem Löschen passieren und ist sonst nicht mehr möglich.
\begin{figure}[h!]
\centering
\includegraphics[width=0.75\linewidth]{img//Modulhandbuch//M20.png}
\caption{Modulhandbuch}
\label{fig:m20}
\end{figure}


\section{Ausgabe}
\label{sec:ausgabe}
Die Seite \textit{Ausgabe} ermöglicht eine Ausgabe der Daten aus dem WebTool in mehreren Dateiformaten.
Zuerst muss dafür der Dateiname des lokalen Planungsstands im Dropdown Menu ausgewählt werden, für den eine Ausgabe erzeugt werden soll und anschließend auf den blauen Knopf \textit{Auswählen} geklickt werden (siehe \figref{fig:a1}).
\begin{figure}[h!]
    \centering
    \includegraphics[width=0.75\linewidth]{img/ausgabe/A1.png}
    \caption{Ausgabe}
    \label{fig:a1}
\end{figure}
Anschließend kann der aktuell ausgewählte Planungsstand durch ein erneutes auswählen im Dropdown Menu und anschließendes Klicken auf den blauen Knopf \textit{Auswählen} geändert werden (siehe \figref{fig:a2}).
\begin{figure}[h!]
    \centering
    \includegraphics[width=0.75\linewidth]{img/ausgabe/A2.png}
    \caption{Ausgabe}
    \label{fig:a2}
\end{figure}
Jetzt können verschiedene Ausgaben durch die Wahl des Ausgabeformats ausgewählt werden.
Die Möglichkeiten sind:
\begin{itemize}
    \item Planungsstand: Um den aktuellen Planungsstand online einzusehen
    \item Vorlesungsverzeichnis: Um eine tex Vorlage für das Vorlesungsverzeichnis zu erzeugen
    \item Professor*innen-Emails: Um automatisch generierte Emails für alle Lehrenden mit ihren jeweiligen Veranstaltungen zu generieren
    \item Sperrzeiten für andere Studiengänge: Um eine Liste mit Sperrzeiten für andere Studiengänge zu erstellen, die Veranstaltungen der Informatik beinhalten und überschneidungsfrei geplant werden sollen
    \item Zentrale Raumplanung: Um einen Raumplan für die zentrale Raumplanung am Campus zu erstellen
    \item Lokale Raumplanung: Um ein SQL-Installations-Skript für das ARBS der Fürstenallee zu erstellen
\end{itemize}

\subsection{Planungsstand}
Auf der Seite \textit{Planungsstand} kann der aktuelle Planungsstand angesehen werden.
Oben auf der Seite werden alle angebotenen Module und Druckrubriken aufgelistet (siehe \figref{fig:a3}).
Wenn alle Teile einer Druckrubrik komplett geplant sind, steht neben dem Namen der Druckrubrik ein grünes \textit{Vollständig}, während bei unvollständig geplanten Druckrubriken ein gelbes \textit{Unvollständig} steht.
Durch einen Klick auf den Namen einer Druckrubrik springt man zu den Modulen der geklickten Druckrubrik.
\begin{figure}[h!]
    \centering
    \includegraphics[width=0.75\linewidth]{img/ausgabe/A3.png}
    \caption{Planungsstand}
    \label{fig:a3}
\end{figure}
In den Ansichten der einzelnen Module werden alle Veranstaltungsteile mit ihrer zugehörigen Art, Tag, Zeit, Raum und Veranstalter aufgelistet (siehe \figref{fig:a4}).
Falls eine Zelle in der Tabelle leer ist, gilt der Wert in der Spalte darüber.
In \figref{fig:a4} wird Veranstaltungsteil 2 also auch von Dr. Rita Hartel veranstaltet und die Übungen 3.2 - 3.8 werden von Mitarbeitern veranstaltet.
Dieses Modul ist vollständig geplant.
\begin{figure}[h!]
    \centering
    \includegraphics[width=0.75\linewidth]{img/ausgabe/A4.png}
    \caption{Planungsstand}
    \label{fig:a4}
\end{figure}
Falls einem Veranstaltungsteil noch kein Raum zugewiesen ist, wird die Raumzelle rot hinterlegt und der Inhalt der Zelle lautet \textit{kein Raum} (siehe \figref{fig:a5}).
\begin{figure}[h!]
    \centering
    \includegraphics[width=0.75\linewidth]{img/ausgabe/A5.png}
    \caption{Planungsstand}
    \label{fig:a5}
\end{figure}
Falls ein Veranstaltungsteil noch überhaupt nicht geplant wurde, also noch kein Tag und eine Zeit zugewiesen wurde, erscheint sowohl der Tag als auch die Zeit rot hinterlegt (siehe \figref{fig:a6}).
Für den Tag wird ein \textit{-} als Inhalt gesetzt und die Zeit startet mit $00-$.
\begin{figure}[h!]
    \centering
    \includegraphics[width=0.75\linewidth]{img/ausgabe/A6.png}
    \caption{Planungsstand}
    \label{fig:a6}
\end{figure}
Falls der Name eines Veranstalters im WebTool und in der lokalen Planungsdatenbank nicht exakt übereinstimmt, etwa wenn der Name des Veranstalters geändert wurde, nachdem die WebTool Datenbank heruntergeladen wurde, um eine lokale Planung zu starten, wird der ähnlichste Name in der WebTool Datenbank gesucht und die betreffenden Zellen werden gelb markiert (siehe \figref{fig:a7}).
\begin{figure}[h!]
    \centering
    \includegraphics[width=0.75\linewidth]{img/ausgabe/A7.png}
    \caption{Planungsstand}
    \label{fig:a7}
\end{figure}

\subsection{Vorlesungsverzeichnis}
Auf der Seite \textit{Vorlesungsverzeichnis} kann ein tex file für das Vorlesungsverzeichnis generiert werden.
Dafür muss der Titel für das Vorlesungsverzeichnis und der Untertitel für das Vorlesungsverzeichnis festgelegt werden und danach der blaue Knopf \textit{Erstellen} gedrückt werden (siehe \figref{fig:a8}).
Danach öffnet sich der Dialog des Betriebsystems, um den Speicherort der Datei festzulegen.

\begin{figure}[h!]
    \centering
    \includegraphics[width=0.75\linewidth]{img/ausgabe/A8.png}
    \caption{Vorlesungsverzeichnis}
    \label{fig:a8}
\end{figure}

\subsection{Professor*innen-Emails}
Auf der Seite \textit{Professor*innen-Emails} können automatisch Emails generiert werden, die für jeden Veranstalter personalisiert die Räume und Zeiten der eigenen Veranstaltungen beinhalten (siehe \figref{fig:a9} und (siehe \figref{fig:a10})).
Der Inhalt des Textfeldes kann durch einen Klick auf \textit{Email-Inhalt Kopieren} in die Zwischenablage kopiert werden.
Durch einen Klick auf \textit{Vorlage Bearbeiten} kann die Vorlage bearbeitet werden (siehe \figref{fig:a11})). 
\begin{figure}[h!]
    \centering
    \includegraphics[width=0.75\linewidth]{img/ausgabe/A9.png}
    \caption{Professor*innen-Emails}
    \label{fig:a9}
\end{figure}

\begin{figure}[h!]
    \centering
    \includegraphics[width=0.75\linewidth]{img/ausgabe/A10.png}
    \caption{Professor*innen-Emails}
    \label{fig:a10}
\end{figure}

\begin{figure}[h!]
    \centering
    \includegraphics[width=0.75\linewidth]{img/ausgabe/A11.png}
    \caption{Professor*innen-Emails}
    \label{fig:a11}
\end{figure}

\subsection{Sperrzeiten für andere Studiengänge}
Auf der Seite \textit{Sperrzeiten für andere Studiengänge} kann automatisch eine Textdatei mit Sperrzeiten für andere Studiengänge erstellt werden (siehe \figref{fig:a12}).
Diese Textdatei beinhaltet alle Zeiten für Veranstaltungen der Informatik, die auch in den ausgewählten Studiengängen gewählt werden können.
Durch Strg+Klick können mehrere Studiengänge auf einmal ausgewählt werden.
Nach einem Klick auf den blauen Knopf \textit{Sperrzeitenliste erstellen} öffnet sich der Betriebsystem Dialog um den Speicherort für die Sperrzeitenliste festzulegen.

\begin{figure}[h!]
    \centering
    \includegraphics[width=0.75\linewidth]{img/ausgabe/A12.png}
    \caption{Sperrzeiten für andere Studiengänge}
    \label{fig:a12}
\end{figure}




\subsection{Zentrale Raumplanung}
Auf der Seite \textit{Zentrale Raumplanung} kann automatisch eine Textdatei mit Raumbelegungen für die Zentrale Raumplanung erstellt werden.
Man kann auswählen für welche Räume der Plan erstellt werden soll. 
Man kann ganze Gebäude abwählen, aber auch einzelne Räume aus den Gebäuden einzeln abwählen, indem man auf den Pfeil bei einem Gebäude drückt (siehe \figref{fig:r1}).
Danach kann man für das entsprechende Gebäude sehen, welche Räume existieren und durch einen Klick auf die jeweilige Checkbox neben einem Raum auswählen, ob er in der Textdatei auftauchen soll (siehe \figref{fig:r2}).
\begin{figure}[h!]
    \centering
    \includegraphics[width=0.75\linewidth]{img/ausgabe/R1.png}
    \caption{Ausgabe}
    \label{fig:r1}
\end{figure}
\begin{figure}[h!]
    \centering
    \includegraphics[width=0.75\linewidth]{img/ausgabe/R2.png}
    \caption{Ausgabe}
    \label{fig:r2}
\end{figure}
Falls kein Raum aus einem Gebäude ausgewählt wurde, erscheint die Checkbox neben dem Gebäude grau, falls alle Räume ausgewählt wurden, einscheint in der Checkbox ein grüner Haken und falls nur manche Räume aus einem Gebäude ausgewählt wurde, erscheint ein grünes Viereck in der Checkbox (siehe \figref{fig:r3}).
\begin{figure}[h!]
    \centering
    \includegraphics[width=0.75\linewidth]{img/ausgabe/R3.png}
    \caption{Ausgabe}
    \label{fig:r3}
\end{figure}
Um Zeit zu sparen, wenn nur die Belegungen der großen oder nur die Belegungen der kleinen Räume exportiert werden sollen, kann oben auf \textit{Große Räume} oder \textit{Kleine Räume} geklickt werden.
Nach der Auswahl kann der Raumplan durch einen Klick auf den blauen Knopf \textit{Raumplan erstellen} erstellt werden (siehe \figref{fig:r4}).
Danach öffnet sich der Dialog vom Betriebsystem, um den Speicherort für den Raumplan festzulegen.
\begin{figure}[h!]
    \centering
    \includegraphics[width=0.75\linewidth]{img/ausgabe/R4.png}
    \caption{Ausgabe}
    \label{fig:r4}
\end{figure}


\subsection{Lokale Raumplanung}
Die Seite \textit{Lokale Raumplanung} kann benutzt werden, um ein SQL Skript zu erstellen, das die geplanten Räume für die Fürstenallee automatisch in das ARBS einträgt.
Zunächst muss die \textit{Skript-Art} ausgewählt werden, indem auf das Dropdown Menu geklickt wird (siehe \figref{fig:lokal1}).
Hier kann entweder \textit{Erstellen} oder \textit{Löschen} ausgewählt werden (siehe \figref{fig:lokal2}).
Wenn das Skript nur für einzelne Räume eingesetzt werden soll, können die übrigen Räume durch die Checkboxen abgewählt werden.
Um das Skript zu erstellen muss nun auf den blauen Knopf \textit{Skript erstellen} gedrückt werden.
Anschließend öffnet sich ein Dialog des Betriebssystems, in dem der Speicherort für das Skript festgelegt werden muss.
\begin{figure}[h!]
    \centering
    \includegraphics[width=0.75\linewidth]{img/ausgabe/Lokal1.png}
    \caption{Lokale Raumplanung}
    \label{fig:lokal1}
\end{figure}
\begin{figure}[h!]
    \centering
    \includegraphics[width=0.75\linewidth]{img/ausgabe/Lokal2.png}
    \caption{Lokale Raumplanung}
    \label{fig:lokal2}
\end{figure}




\end{document}
