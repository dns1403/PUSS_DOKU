\section{Meine Veranstaltungen}
\label{sec:meine_veranstaltungen}
Auf der Seite \textit{Meine Veranstaltungen} können Veranstaltungen des Veranstaltungskatalogs angeboten werden.
Sie wird genutzt, um angebotene Veranstaltungen konfigurieren zu können und dafür die Anzahl der maximalen Hörer einzugeben, sowie Wünsche bezüglich des Raumes zu äußern und Wunschzeiten anzugeben.
Am Anfang, wenn Sie noch keine Kurse angeboten haben, sieht die Seite noch leer aus und bietet oben unter dem Punkt \textit{Aktuelle Veranstaltungen} die Möglichkeit, eine Veranstaltung anzubieten (siehe \figref{fig:meine_veranstaltungen_vplan}).
Diese Seite zeigt immer die Veranstaltungen an, die Sie in Ihrem ausgewählten Semester anbieten.
Bevor Sie Veranstaltungen hinzufügen oder löschen, vergewissern Sie sich, dass Sie sich im richtigen Semester befinden.
Lehrende können nur Veranstaltungen im Semester des aktuellen Planungszyklus Änderungen vornehmen.
Falls kurzfristige Änderungen für das vorherige Semester vorgenommen werden sollen, muss dies über den aktuellen Planer geschehen.
\begin{figure}[h!]
    \centering
    \includegraphics[width=0.75\linewidth]{img/MeineVeranstaltungen//MeinVeranstaltungsplan.png}
    \caption{Meine Veranstaltungen}
    \label{fig:meine_veranstaltungen_vplan}
\end{figure}

\subsection{Veranstaltung anbieten}
\label{sec:meine_veranstaltungen_add}

Um eine Veranstaltung für das aktuell ausgwählte Semester anzubieten, muss der \textit{Hinzufügen} Knopf unter \textit{Aktuelle Veranstaltungen} gedrückt werden.
Dadurch wird die Seite \textit{Veranstaltungsplan bearbeiten} geöffnet (siehe \figref{fig:meine_veranstaltungen_vplan_add}).
\begin{figure}[h!]
    \centering
    \includegraphics[width=0.75\linewidth]{img/MeineVeranstaltungen//MeinVeranstaltungsplan_AddVPlan.png}
    \caption{Meine Veranstaltungen - Veranstaltungsplan bearbeiten}
    \label{fig:meine_veranstaltungen_vplan_add}
\end{figure}
Hier muss zuerst der Fachbereich ausgewählt werden, von dem die Veranstaltung angeboten werden soll.
Für Veranstaltungen wie Analysis, Algebra und Stochastik muss hier \textit{Mathematik} ausgewählt werden, für Veranstaltungen wie Grundlagen der Elektrotechnik, Signalverarbeitung und Nachrichtentechnik muss hier \textit{Elektrotechnik} ausgewählt werden.
Danach kann durch einen Klick auf das Dropdown Menu in der Reihe Kurs ein Kurs ausgewählt werden.
Hier werden alle Kurse des vorher ausgewählten Fachbereichs angezeigt und können durch einen Klick ausgewählt werden (siehe \figref{fig:meine_veranstaltungen_vplan_add_kursliste}).
\begin{figure}[h!]
    \centering
    \includegraphics[width=0.75\linewidth]{img/MeineVeranstaltungen//MeinVeranstaltungsplan_AddVPlan_Kursliste.png}
    \caption{Meine Veranstaltungen - Veranstaltungsplan bearbeiten - Kursauswahl}
    \label{fig:meine_veranstaltungen_vplan_add_kursliste}
\end{figure}
Anschließend muss noch die Sprache, in der die Lehrveranstaltung gehalten wird, durch einen Klick auf Deutsch oder Englisch ausgewählt werden.
Danach muss auf den Knopf \textit{Nächster Schritt} gedrückt werden, um die Veranstaltung weiter zu konfigurieren.

Nun öffnet sich die Seite \textit{Veranstaltungsplan bearbeiten - Wunschzeiten auswählen} (siehe \figref{fig:meine_veranstaltungen_vplan_add_wunschzeiten}).
\begin{figure}[h!]
    \centering
    \includegraphics[width=0.75\linewidth]{img/MeineVeranstaltungen//MeinVeranstaltungsplan_Add_Wunschzeiten.png}
    \caption{Meine Veranstaltungen - Wunschzeiten konfigurieren}
    \label{fig:meine_veranstaltungen_vplan_add_wunschzeiten}
\end{figure}
Hier sollte zunächst überprüft werden, dass unter \textit{Ausgewählte Veranstaltung} die Veranstaltung steht, die angeboten werden soll.
Darunter befinden sich die Veranstaltungsteile der Veranstaltung mit weiteren Konfigurationsmöglichkeiten aufgelistet.
Wenn Sie den Namen der Veranstaltung oder die Art oder Dauer der Teile ändern möchten (zum Beispiel statt 3 Stunden Vorlesung nur 2 Stunden Vorlesung anbieten wollen), kann dies wie in Kapitel \ref{sec:meine_veranstaltungen_teile} beschrieben geändert werden.
Im folgenden wird der Konfigurationsprozess am Beispiel von Vorlesungen, Übungen und Zentralübungen beschrieben.

\subsubsection{Vorlesungen konfigurieren}
Für den Veranstaltungsteil \textit{Vorlesung} können Wunschzeiten ausgewählt werden (siehe \figref{fig:meine_veranstaltungen_vplan_add_wunschzeiten_v}).
\begin{figure}[h!]
    \centering
    \includegraphics[width=0.75\linewidth]{img/MeineVeranstaltungen//MeinVeranstaltungsplan_Add_Wunschzeiten_Vorlesung.png}
    \caption{Vorlesungen konfigurieren}
    \label{fig:meine_veranstaltungen_vplan_add_wunschzeiten_v}
\end{figure}
Dabei können die vom Planer definierten Zeitblöcke mit einem Klick ausgewählt und abgewählt werden.
Ausgewählte Wunschzeiten werden durch einen blauen Haken in der entsprechenden Checkbox angezeigt.
Es können beliebig viele Wunschzeiten ausgewählt werden.
Diese werden bei der Planung beachtet, wenn dies möglich ist.
Es kann jedoch vorkommen, dass es nicht möglich ist, den Wunschzeiten zu entsprechen.
Außerdem muss die \textit{Erwartete Hörerzahl} angegeben werden, damit die Vorlesung einen Raum zugewiesen bekommen kann, der groß genug ist.
Bitte schätzen Sie die Hörerzahl, z.B. anhand der Teilnehmer der letzten Durchführung dieser Veranstaltung.
Darunter können weitere \textit{Raumanforderungen} definiert werden. 
Es kann ausgewählt werden, ob eine Tafel, ein Beamer und Fenster vorhanden sein sollen und ob die Vorlesung in der Fürstenallee oder am Campus stattfinden soll.
Falls weitere Raumanforderungen bestehen, können diese in das Textfeld \textit{Sonstige Raumanforderungen} eingetragen werden.
Dies kann zum Beispiel benutzt werden, um zu sagen, dass der Raum möglichst 6 oder mehr Tafeln haben soll, weil in der Vorlesung viel an die Tafel geschrieben wird.
Danach muss im Feld \textit{Beginnt in Semesterwoche} die Semesterwoche angegeben werden, in der die Vorlesung zum ersten Mal stattfindet.
Im Feld \textit{Sonstige Bemerkung zur Veranstaltungsplanung} können weitere Wüsche und Informationen für die Planung angegeben werden.
Dies kann benutzt werden, falls die Veranstaltung in besonderen Räumen stattfinden soll (zum Beispiel die eigenen Räume einer Fachgruppe) oder wenn die Vorlesung möglichst mindestens 2 Tage vor der Übung stattfinden soll oder falls diese Vorlesung auf keinen Fall parallel zu einer anderen spezifischen Vorlesung liegen soll.

\subsubsection{Übungen konfigurieren}
Übungen haben alle Konfigurationsparameter einer Vorlesung, jedoch zusätzlich zwei weitere Optionen (siehe \figref{fig:meine_veranstaltungen_vplan_add_wunschzeiten_u}).
\begin{figure}[h!]
    \centering
    \includegraphics[width=0.75\linewidth]{img/MeineVeranstaltungen//MeinVeranstaltungsplan_Add_Wunschzeiten_Übung.png}
    \caption{Übungen konfigurieren}
    \label{fig:meine_veranstaltungen_vplan_add_wunschzeiten_u}
\end{figure}
In das Feld \textit{Übungsanzahl insgesamt} muss eingetragen werden, wie viele Übungsgruppen es pro Woche geben soll.
Im Feld \textit{Davon selber halten} soll eingetragen werden, wie viele Übungsgruppen vom Dozenten der Vorlesung gehalten werden und nicht von Mitarbeitern oder Hilfskräften übernommen werden.
Die \textit{Erwartete Hörerzahl} bei Übungen bezieht sich auf die Anzahl der Teilnehmer pro Übungsgruppe.
Wenn es 10 Übungsgruppen bei einer Veranstaltung mit 200 Teilnehmern gibt, sollte hier also 20 angegeben werden.

\subsubsection{Zentralübungen konfigurieren}
Auch eine Zentralübung hat alle Parameter einer Vorlesung und zusätzlich weitere für Zentralübungen spezifische Optionen (siehe \figref{fig:meine_veranstaltungen_vplan_add_wunschzeiten_z}).
\begin{figure}[h!]
    \centering
    \includegraphics[width=0.75\linewidth]{img/MeineVeranstaltungen//MeinVeranstaltungsplan_Add_Wunschzeiten_Zentralübung.png}
    \caption{Zentralübungen konfigurieren}
    \label{fig:meine_veranstaltungen_vplan_add_wunschzeiten_z}
\end{figure}
Es muss ausgewählt werden, ob die Zentralübung vom Dozenten der Vorlesung selbst gehalten wird oder ob sie von einem Mitarbeiter durchgeführt wird oder ob sie in diesem Semester nicht angeboten wird.
Diese Eingabe ist wichtig, damit der Dozent der Vorlesung nicht gleichzeitig für zwei Veranstaltungen eingeplant wird und im Fall einer nicht stattfindenden Zentralübung kein Raum unnötig blockiert wird.
Außerdem kann für Zentralübungen bei \textit{Wochentakt} ausgewählt werden, ob die Zentralübung 14-tägig zweistündig oder wöchentlich einstündig gehalten werden soll.

\subsubsection{Änderungen speichern}
Um die getroffenen Eingaben zu speichern, muss unten der blaue Knopf \textit{Absenden} gedrückt werden.
Danach gelangt man wieder auf die Seite \textit{Meine Veranstaltungen}.
Hier wird jetzt die geplante Veranstaltung unter \textit{Aktuelle Veranstaltungen} angezeigt (siehe \figref{fig:meine_veranstaltungen_vplan2}).
\begin{figure}[h!]
    \centering
    \includegraphics[width=0.75\linewidth]{img/MeineVeranstaltungen//MeinVeranstaltungsplan2.png}
    \caption{Meine Veranstaltungen}
    \label{fig:meine_veranstaltungen_vplan2}
\end{figure}

\subsection{Angebotene Veranstaltung bearbeiten}
\label{sec:meine_veranstaltungen_bearbeiten}
Wenn nach dem ursprünglichen Erstellen einer angebotenen Veranstaltung noch Änderungen an dieser gemacht werden sollen, kann diese bearbeitet werden, indem die gewünschte Veranstaltung unter dem Punkt \textit{Aktuelle Veranstaltungen} angeklickt wird (zum Beispiel auf den Namen der Veranstaltung klicken wie in \figref{fig:meine_veranstaltungen_vplan2}).
Dadurch öffnet sich wieder der Dialog, mit der die Veranstaltung ursprünglich erstellt wurde.
Alle Eingaben die bisher gespeichert wurden, sind hier vorausgewählt, sodass nur Änderungen gemacht werden müssen und die Veranstaltung nicht vollständig neu konfiguriert werden muss.
Ab hier sind die Schritte wieder wie in Kapitel \ref{sec:meine_veranstaltungen_add} beschrieben durchzuführen.
Wenn Sie den Namen der Veranstaltung oder die Art oder Dauer der Teile ändern möchten (zum Beispiel statt 3 Stunden Vorlesung nur 2 Stunden Vorlesung anbieten wollen), kann dies wie in Kapitel \ref{sec:meine_veranstaltungen_teile} geändert werden.

\subsection{Veranstaltung nicht mehr anbieten}
Wenn Sie eine von Ihnen angebotene Veranstaltung doch nicht mehr anbieten wollen, drücken sie auf der Seite \textit{Meine Veranstaltungen} bei der zu löschenden Veranstaltung auf den Knopf \textit{Löschen} (siehe \figref{fig:meine_veranstaltungen_vplan2}).
\begin{warning}
    Aus Versehen gelöschte Veranstaltungen können nur direkt nach dem Löschen wiederhergestellt werden wie in Kapitel \ref{sec:meine_veranstaltungen_wiederherstellen} beschrieben wird.
\end{warning}


\subsection{Aus Versehen gelöschte Veranstaltung wiederherstellen}
\label{sec:meine_veranstaltungen_wiederherstellen}
Falls eine Veranstaltung aus Versehen gelöscht wurde, drücken Sie auf den Knopf \textit{Wiederherstellen} (siehe \figref{fig:meine_veranstaltungen_wiederherstellen}).
\begin{figure}[h!]
    \centering
    \includegraphics[width=0.75\linewidth]{img/MeineVeranstaltungen//MeinVeranstaltungsplan_löschen.png}
    \caption{Aus Versehen gelöschte Veranstaltung wiederherstellen}
    \label{fig:meine_veranstaltungen_wiederherstellen}
\end{figure}
Danach wird die vorher gelöschte Veranstaltung wieder unter \textit{Aktuelle Veranstaltungen} angezeigt und alle Wunschzeiten und andere Eingaben werden wiederhergestellt.
Diese Aktion ist nur möglich, wenn sie direkt im Anschluss an das Löschen ausgeführt wird.
Ansonsten sind alle Eingaben zu dieser Veranstaltung unwiderruflich gelöscht worden.

\subsection{Veranstaltungsteile bearbeiten}
\label{sec:meine_veranstaltungen_teile}
Um die Länge oder Art von Veranstaltungsteilen zu ändern, muss zunächst in die Ansicht \textit{Veranstaltungsplan bearbeiten - Wunschzeiten auswählen} der zu ändernden Veranstaltung gewechselt werden.
Dies ist für neu angebotene Veranstaltungen in Kapitel \ref{sec:meine_veranstaltungen_add} und für schon existierende angebotenen Veranstaltungen in Kapitel \ref{sec:meine_veranstaltungen_bearbeiten} beschrieben.
Hier muss der Knopf \textit{Veranstaltungsteile bearbeiten} gedrückt werden (siehe \figref{fig:meine_veranstaltungen_w_vteile}).
\begin{figure}[h!]
    \centering
    \includegraphics[width=0.75\linewidth]{img/MeineVeranstaltungen//MeinVeranstaltungsplan_Add_Wunschzeiten_Vteile.png}
    \caption{Veranstaltungsteile ändern}
    \label{fig:meine_veranstaltungen_w_vteile}
\end{figure}

Dadurch öffnet sich die Seite \textit{Veranstaltung hinzufügen} (siehe \figref{fig:meine_veranstaltungen_w_vteile_bearbeiten}).
\begin{figure}[h!]
    \centering
    \includegraphics[width=0.75\linewidth]{img/MeineVeranstaltungen//MeinVeranstaltungsplan_Add_Wunschzeiten_VteileBearbeiten.png}
    \caption{Veranstaltungsteile hinzufügen, löschen und bearbeiten}
    \label{fig:meine_veranstaltungen_w_vteile_bearbeiten}
\end{figure}
Oben kann der Kursname sowohl in Englisch als auch in Deutsch geändert werden und ein Kürzel für den Kurs festgelegt werden sowie der Fachbereich geändert werden.
Die Paulnummer kann nur von einem Planer oder Administrator geändert werden.
Unter dem Punkt \textit{Veranstaltungsteile} können die einzelnen Teile der Veranstaltung konfiguriert werden.
Hier können durch einen Klick auf den Knopf \textit{Hinzufügen} neue Veranstaltungsteile hinzugefügt werden, wenn diese gewünscht sind.
Es können die Art des Veranstaltungsteils (Vorlesung, Übung, Zentralübung,Praktikum, ...) festgelegt werden und wie viele Stunden dieser Teil dauern soll.
Außerdem können auch hier Bemerkungen für die Planung eingegeben werden.
Um einen Veranstaltungsteil zu entfernen, muss auf den Knopf \textit{Veranstaltungsteil entfernen} unter dem jeweiligen Veranstaltungsteil gedrückt werden.
Um die Änderungen zu speichern, muss unten auf den blauen Knopf \textit{Absenden} gedrückt werden (siehe \figref{fig:meine_veranstaltungen_w_vteile_bearbeiten2}).
\begin{figure}[h!]
    \centering
    \includegraphics[width=0.75\linewidth]{img/MeineVeranstaltungen//MeinVeranstaltungsplan_Add_Wunschzeiten_VteileBearbeiten2.png}
    \caption{Veranstaltungsteile hinzufügen, löschen und bearbeiten}
    \label{fig:meine_veranstaltungen_w_vteile_bearbeiten2}
\end{figure}
Danach können die Änderungen direkt auf der Seite \textit{Veranstaltungsplan bearbeiten - Wunschzeiten auswählen} gesehen werden.

\subsection{Angebotene Veranstaltung aus vergangenen Semestern kopieren}
\label{sec:meine_veranstaltungen_copy}
Wenn Sie eine Veranstaltung schon in einem früheren Semester angeboten haben, können Sie ihre Eingaben von damals übernehmen lassen und müsssen nicht alles neu eingeben.
Dafür muss unter dem Punkt \textit{Vergangene Veranstaltungen} das Semester ausgewählt werden, aus dem die Veranstaltung kopiert werden soll und anschließend der \textit{Auswählen} Knopf gedrückt werden.
Anschließend erscheint eine Liste mit Veranstaltungen, die Sie in diesem Semester gehalten haben.
Um nun eine Veranstaltung in das aktuell ausgewählte Semester zu kopieren, muss der \textit{Kopieren} Knopf gedrückt werden (siehe \figref{fig:meine_veranstaltungen_copy}).
Anschließend erscheint die kopierte Veranstaltung unter \textit{Aktuelle Veranstaltungen} und sie kann angeklickt werden, um die Inhalte zu betrachten oder Änderungen vorzunehmen.

\begin{figure}[h!]
    \centering
    \includegraphics[width=0.75\linewidth]{img/MeineVeranstaltungen//MeinVeranstaltungsplan_Copy.png}
    \caption{Veranstaltungen aus vergangenen Semestern kopieren}
    \label{fig:meine_veranstaltungen_copy}
\end{figure}
