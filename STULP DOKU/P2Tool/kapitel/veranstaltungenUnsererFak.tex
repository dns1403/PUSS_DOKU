\section{Veranstaltungen unserer Fakultät}
\label{sec:vfak}

\subsection{Lehrveranstaltung anlegen}

\begin{warning}
    Alle in diesem Kapitel beschrieben Aktionen werden im aktuell ausgewählten und in allen neueren Semestern ausgeführt.
\end{warning}

\label{sec:lv_anlegen}
Bevor eine Lehrveranstaltung von einem Lehrenden gehalten werden kann, muss diese Lehrveranstaltung angelegt werden.
Es müssen zum Beispiel der Name, die Paulnummer und die Lehrveranstaltungsteile spezifiziert werden.
Dafür muss auf die Seite \textit{Veranstaltungen unserer Fakultät} gewechselt werden, wo eine Liste mit allen bereits angelegten Veranstaltungen angelegt wird und auf den \textit{Hinzufügen} Knopf gedrückt werden (siehe \figref{fig:veranstaltung_anlegen1}).
\begin{figure}[h!]
    \centering
    \includegraphics[width=0.75\linewidth]{img//VeranstaltungenDerFak//VeranstaltungAnlegen1.png}
    \caption{Veranstaltungen anlegen}
    \label{fig:veranstaltung_anlegen1}
\end{figure}
Danach öffnet sich die Seite \textit{Veranstaltung hinzufügen} (siehe \figref{fig:veranstaltung_anlegen2}).
\begin{figure}[h!]
    \centering
    \includegraphics[width=0.75\linewidth]{img//VeranstaltungenDerFak//VeranstaltungAnlegen2.png}
    \caption{Veranstaltungen anlegen}
    \label{fig:veranstaltung_anlegen2}
\end{figure}
Hier müssen für den neuen Kurs der Kursname auf Deutsch (mindestens 3 Zeichen lang), der Kursname auf Englisch (mindestens 3 Zeichen lang), ein Kürzel, ein Fachbereich und eine Paulnummer angegeben werden (genauer beschrieben in Kapitel \ref{sec:lv_anlegen_paul}).
Unter dem Punkt \textit{Veranstaltungsteile} können die einzelnen Teile der Veranstaltung konfiguriert werden.
Hier können durch einen Klick auf den Knopf \textit{Hinzufügen} neue Veranstaltungsteile hinzugefügt werden, wenn diese gewünscht sind.
Es können die Art des Veranstaltungsteils (Vorlesung, Übung, Zentralübung, Praktikum, ...) festgelegt werden und wie viele Stunden dieser Teil dauern soll.
Außerdem können auch hier Bemerkungen für die Planung eingegeben werden.
Um einen Veranstaltungsteil zu entfernen, muss auf den Knopf \textit{Veranstaltungsteil entfernen} unter dem jeweiligen Veranstaltungsteil gedrückt werden.
Um die Änderungen zu speichern, muss unten auf den blauen Knopf \textit{Absenden} gedrückt werden.

Anschließend kann die angelegte Veranstaltung auf der Seite \textit{Veranstaltungen unserer Fakultät} in der Liste mit Veranstaltungen gesehen werden (siehe \figref{fig:veranstaltung_anlegen3}).
\begin{figure}[h!]
    \centering
    \includegraphics[width=0.75\linewidth]{img//VeranstaltungenDerFak//VeranstaltungAnlegen3.png}
    \caption{Angelegte Veranstaltungen ansehen}
    \label{fig:veranstaltung_anlegen3}
\end{figure}

\subsubsection{Paulnummern}
\label{sec:lv_anlegen_paul}
Paulnummern bestehen aus einer Zeichenfolge mit dem Schema \texttt{L.XXX.XXXXX}, wobei X jeweils eine Ziffer von 0 bis 9 ist.
Für die Informatik fangen Paulnummern immer mit \texttt{L.079.} an.
Das WebTool ist eigenständig in der Lage, eine freie Paulnummer innerhalb eines bestimmten Bereichs zu finden, wenn eine freie Paulnummer existiert.
Die Eingabe in das Feld stellt einen Präfix dar, den das Tool auf jeden Fall einhält.
Eine genaue Paulnummer kann festgelegt werden, indem eine Nummmer mit dem Schema \texttt{L.XXX.XXXXX} eingegeben wird.
Falls die genaue Paulnummer nicht schon festgelegt ist, sondern nur mit einem bestimmten Präfix anfangen soll, zum Beispiel \texttt{L.079.05}, erkennt das Tool automatisch, dass die letzten drei Ziffern noch offen sind und sucht eine freie Nummer, die mit dem gegebenen Präfix startet.
In diesem Fall sind noch 1000 Paulnummern mit dem Präfix offen \texttt{L.079.05000} - \texttt{L.079.05999}.
Das Tool sucht eine freie Nummer mit dem gegebenen Präfix nach dem folgenden Algorithmus aus:
\begin{itemize}
    \item Nummern, die noch nie benutzt wurden
    \item Die Nummer, die am längsten nicht benutzt wurde (für eine \textbf{angelegte} Veranstaltung)
\end{itemize}
Falls keine Nummer gefunden wurde, die benutzt werden kann (Paulnummern können nicht zweimal pro Semester verwendet werden), wird ein Fehler angezeigt (siehe \figref{fig:veranstaltung_anlegen4}).
\begin{figure}[h!]
    \centering
    \includegraphics[width=0.75\linewidth]{img//VeranstaltungenDerFak//VeranstaltungAnlegen4.png}
    \caption{Fehlermeldung, wenn es keine freie Paulnummer gibt}
    \label{fig:veranstaltung_anlegen4}
\end{figure}
Sie beinhaltet den Namen der Veranstaltung, die eine Nummer mit dem gegebenen Präfix enthält und möglichst lange nicht \textbf{angeboten} wurde.
Der Name der anderen Veranstaltung kann angeklickt werden, um diese Veranstaltung direkt zu bearbeiten und die Nummer dieser Veranstaltung ändern zu können, um sie für neue Veranstaltungen frei zu machen.
Das Ändern von Veranstaltungen funktioniert genauso wie das Anlegen von Veranstaltungen.

\subsection{Lehrveranstaltung bearbeiten}
Um eine bereits angelegte Veranstaltung zu bearbeiten, muss vom Ersteller der Lehrveranstaltung auf die Seite \textit{Veranstaltungen unserer Fakultät} gewechselt werden und dort auf die zu bearbeitende Lehrveranstaltung geklickt werden (siehe \figref{fig:veranstaltung_anlegen6}).
\begin{figure}[h!]
    \centering
    \includegraphics[width=0.75\linewidth]{img//VeranstaltungenDerFak//VeranstaltungAnlegen6.png}
    \caption{Angelegte Veranstaltungen bearbeiten}
    \label{fig:veranstaltung_anlegen6}
\end{figure}

Danach läuft alles so wie beim ersten Anlegen einer Lehrveranstaltung, wie in Kapitel \ref{sec:lv_anlegen} beschrieben.

\subsection{Lehrveranstaltung löschen}
Um eine bereits angelegte Veranstaltung zu bearbeiten, muss vom Ersteller der Lehrveranstaltung auf die Seite \textit{Veranstaltungen unserer Fakultät} gewechselt werden und dort auf den Knopf \textit{Löschen} neben der zu bearbeitenden Lehrveranstaltung geklickt werden (siehe \figref{fig:veranstaltung_anlegen5}).

\begin{figure}[h!]
    \centering
    \includegraphics[width=0.75\linewidth]{img//VeranstaltungenDerFak//VeranstaltungAnlegen5.png}
    \caption{Lehrveranstaltung löschen}
    \label{fig:veranstaltung_anlegen5}
\end{figure}
