\subsection{Rollen im Planungsprozess}
Der Planungsprozess beinhaltet verschiedene Rollen, die lediglich Zugriff auf die Daten erhalten, die sie zur Ausführung ihrer Aufgabe benötigen.
%
\paragraph*{Lehrveranstaltungsbeauftragte*r}
Der oder die Lehrveranstaltungsbeauftragte ist für die Erstellung der Liste aller Kurse einer Fakultät verantwortlich. Dazu muss er in der Lage sein, Kurse zu erstellen. Er oder sie hat lediglich Zugriff auf die von ihm/ihr erstellten Kurse.  Es bietet sich an, dass der oder die Lehrveranstaltungsbeauftragte eines Studiengangs eng mit der oder dem Studiengangsbeauftragten des Studiengangs zusammenarbeitet - oder dass es die selbe Person ist. 
%
\paragraph*{Studiengangsbeauftragte*r}
Der oder die Studiengangsbeauftragte ist für die Erstellung und Wartung der Modulhandbücher seiner Studiengänge verantwortlich. Zur Durchführung seiner Aufgabe benötigt er die Berechtigung, Modulhandbücher zu erstellen. Er hat Zugriff auf die Liste der angebotenen Kurse und auf die Modulhandbücher, die er verwaltet.
%
\paragraph*{Professor*in}
Professor*innen sind für das Anbieten von Lehrveranstaltungen in jedem Semester verantwortlich. Dazu benötigen sie Zugriff auf das Modulhandbuch des Studiengangs, in welchem sie lehren und die Berechtigung Lehrveranstaltungen zu erstellen. Dabei haben sie lediglich Zugriff auf von ihnen angebotene Lehrveranstaltungen.  Diese Rolle wird ebenfalls von weiteren Lehrenden ohne Professur ausgefüllt, die in eigener Verantwortung eine Lehrveranstaltung anbieten. 
%
\paragraph*{Planer*in}
Der oder die Planer*in eines Studiengangs ist für die Erstellung des Stundenplans verantwortlich. Zur Ausführung der Planung benötigt er/sie Zugriff auf alle angebotenen Lehrveranstaltungen in seinem/ihrem Studiengang für das zu planende Semester. Er oder sie soll zudem die Möglichkeit haben, diese zu editieren, falls Informationen fehlen oder falsch eingetragen worden sind. Zudem soll er/sie in die Lage versetzt werden, aus den Planungsdaten das aktuelle Vorlesungsverzeichnis zu exportieren. Ihm oder ihr ist ebenfalls das Planungstool auszuhändigen.
%

\paragraph*{Admin}
Die Rolle Admin ist für die Benutzerverwaltung zuständig. Sie trägt neue User ein und ordnet ihnen die korrekte Rolle zu.
