 \subsection{Grundbegriffe der Planung}
In diesem Abschnitt wird paragraphenweise jeder Begriff erklärt, der für den Planungsprozess notwendig ist.
%
\paragraph*{Lehrveranstaltung}
Lehrveranstaltungen bilden den zentralen Bestandteil des Planungsprozesses. Lehrver- anstaltungen werden von  Lehrenden (Professor*innen etc.)  angeboten und vermitteln den Student*innen Wissen über ein bestimmtes Thema, das in der Lehrveranstaltung behandelt wird. Lehrveranstaltungen setzen sich aus mindestens einem Lehrveranstaltungsteil zusammen und können durch Aufnahme in das Modulhandbuch eines Studiengangs Teil dessen werden.
%
\paragraph*{Lehrveranstaltungsteil}
Ein Lehrveranstaltungsteil ist immer Teil einer Lehrveranstaltung und besteht aus der Dauer (Anzahl Stunden) und einer Lehrveranstaltungsart. Einem Lehrveranstaltungsteil kann im Rahmen der  lokalen  Planung ein Tag, eine Uhrzeit und ein Raum zugewiesen werden, an dem dieser stattfinden soll.
%
\paragraph*{Lehrveranstaltungsart}
Die Lehrveranstaltungsart bestimmt für einen Lehrveranstaltungsteil, welche Art von Lehre in diesem Teil  stattfindet. Gültige Lehrveranstaltungsarten sind beispielsweise Vorlesungen, Zentralübungen, Übungen, Proseminare, Seminare, Projektgruppen, Exkursionen, Praktika oder Oberseminare.
%
\paragraph*{Studiengang}
Ein Studiengang kann von Student*innen belegt werden. Es gibt zu jedem Studiengang ein Modulhandbuch, welches die von den Student*innen zu belegenden Lehrveranstaltungen beschreibt.
%
\paragraph*{Modulhandbuch}
Ein Modulhandbuch gehört immer zu genau einem Studiengang. Es besteht aus einer Ansammlung von Lehrveranstaltungen, die im Rahmen des zugehörigen Studiengangs durch die Student*innen belegt werden können. Es kann zudem Lehrveranstaltungen definieren, die von Student*innen verpflichtend belegt werden müssen.  In den meisten Studiengängen werden die Lehrveranstaltungen Semestern zugeordnet, in denen sie gehört werden sollen. Des Weiteren definiert es die zu erlangenden ECTS-Punkte der Lehrveranstaltungen. 
%
\paragraph*{Druckrubriken}
Druckrubriken dienen der Leserlichkeit des Vorlesungsverzeichnisses. Sie fassen Lehrveranstaltungen unter verschiedenen Rubriken zusammen und stellen so eine baumartige Ordnung her.
%
\paragraph*{Raum}
Ein Raum ist eine Ressource, die von der Universität bereitgestellt wird. Sie ist insofern notwendig für einen Lehrveranstaltungsteil, als dass dieser ohne einen Raum nicht stattfinden kann.
%
\paragraph*{Professor*in}
Ein*e Professor*in ist eine von der Universität für einen Studiengang zur Verfügung gestellte menschliche Ressource.  Sie haben die Möglichkeit  Lehrveranstaltungen anzubieten. Jeder Lehrveranstaltung ist genau ein*e Professor*in  oder andere Lehrperson zugeordnet  (es gibt seltene Ausnahmen mit mehreren Professor*innen pro Lehrveranstaltung), jedoch kann ein*e Professor*in mehrere Lehrveranstaltungen anbieten.
%
\paragraph*{Raumkonflikt}
Ein Raumkonflikt ist ein zentrales Problem während der Planung von Lehrveranstaltungen. Ein Raumkonflikt liegt vor, wenn zwei Lehrveranstaltungsteile eine zeitliche Überschneidung vorweisen und in demselben Raum stattfinden.
%
\paragraph*{Professorkonflikt}
Ein Professorkonflikt ist ein zentrales Problem während der Planung von Lehrveranstaltungen. Ein Professorkonflikt liegt vor, wenn zwei Lehrveranstaltungsteile, die von demselben Professor gehalten werden, eine zeitliche Überschneidung vorweisen.
%
\paragraph*{Studiengangskonflikt}
Ein Studiengangskonflikt ist ein zentrales Problem während der Planung von Lehr- veranstaltungen. Ein Studiengangskonflikt liegt vor, wenn zwei Lehrveranstaltungsteile des gleichen Studien- gangs, die von einem Studierenden verpflichtend belegt werden müssen, eine zeitliche Überschneidung vorweisen.  Unterschieden wird dabei noch zwischen leichten und schweren Studiengangskonflikten. Ein schwerer Konflikt besteht, wenn die betroffende Veranstaltung eine Pflicht-Veranstaltung ist und es keine alternativen Termine gibt. 
%

\paragraph*{Lokales Planungstool}
Das Tool zur lokalen Planung wird in diesem Dokument hauptsächlich "lokales Planungstool"  oder "Planungstool" genannt.
%
\paragraph*{p2tool}
Das p2tool ist der Webserver, der die Eingaben der Lehrenden entgegen nimmt. Alternativ auch einfach "Webserver" oder "Webprogramm".
