\section{Nutzung des lokalen Planungstools}
\label{lokales_Planungstool}
Dieses Kapitel beschreibt die Nutzung des Tools zur lokalen Planung als Planer. 
Insbesondere wird die Übertragung der Daten von Untis auf das lokale Planungstool beschrieben und wie das lokale Planungstool genutzt werden kann, um einen Stundenplan zu erstellen.
\subsection{Unterschiede zwischen der Schule und der Universität}
Bevor wir auf die Nutzung des lokalen Planungstools eingehen, wollen wir zunächst kurz die Unterschiede zwischen der Planung an der Schule und der Universität erläutern.
Dies ist notwendig, da das lokale Planungstool ursprünglich für die Planung an der Universität entwickelt wurde und es daher einige Unterschiede in der Terminologie und den Konzepten gibt.
Außerdem haben wir uns dazu entschieden die Datenbankstruktur des lokalen Planungstools beizubehalten, um eine Kompatibilität mit zukünftigen Erweiterungen zu gewährleisten.
Die resultierenden Unterschiede werden wir im Folgenden kurz zusammenfassen, und falls notwendig erläutern, wie die Begriffe der Schule in die Terminologie des lokalen Planungstools übersetzt werden können.
\begin{itemize}
	\item \textbf{Professor}: Dies entspricht dem Begriff \textbf{Lehrer} an der Schule.
	\item \textbf{Kurs}: Dies entspricht dem Begriff \textbf{Veranstaltung/Fach} an der Schule.
	\item \textbf{Studiengang}: Dies entspricht dem Begriff \textbf{Klasse} an der Schule.
	\item \textbf{Semester (eines Studiengangs)}: Dies entspricht dem Begriff \textbf{Jahrgang} der zum Studiengang gehörigen Klasse.
	\item \textbf{Nebenfach(NF)}: Dies haben wir nicht verwendet, da an der Schule keine Nebenfächer existieren. Es ist standardmäßig auf „Ohne“ gesetzt.
	\item \textbf{Prüfungsordnung(PO)}: Dieses wird beim übernehmen eines Stundenplans aus Untis auf die Schulform und Jahrgang bei der zum Studiengang gehörigen Klasse gesetzt. 
	Bei keinem bestehenden Stundenplan kann dies ignoriert werden und muss nicht gesetzt werden.
	\item \textbf{Veranstaltungsteil} Dies ist wohl die größte Abweichung in der Terminologie. Da es keine Veranstaltungsteile an der Schule gibt, haben wir diesen Begriff zweckentfremdet, um die Zuordnung von Kursen zu Klassen zu ermöglichen.
	Dabei ist der VeranstaltungsteilID eine koodierte Klasse zugeordnet. Dabei haben wir folgende Kodierung verwendet:
\end{itemize}
\begin{tabular}{|l|l|p{6cm}|l|}
	\hline
	\textbf{Bereich} & \textbf{Muster} & \textbf{Regel / Logik} & \textbf{Beispiel} \\
	\hline
	Sekundarstufe I & 05A -- 10G & Jahrgang + Punkt + Index \newline (A=1, B=2 \dots G=7) & 05C $\to$ 5.3 \\
	\hline
	Oberstufe & EF, Q1, Q2 & Fortlaufende IDs (nach Klasse 10) & EF $\to$ 11 \\
	\hline
\end{tabular}
	
\subsection{Voraussetzung zur Nutzung der Software als Planer}
\subsubsection{Benötigte zusätzliche Software}
 Um das lokale Planungstool problemlos auszuführen, wird Java 22 oder 23 benötigt.
 Falls außerdem ein bestehender Stundenplan von Untis in das lokale Planungstool übertragen werden soll, wird zusätzlich Python 3 benötigt, um die Dateien von Untis in das Format der lokalen Planungsdatenbank zu übertragen.
\subsubsection{Korrekte Installation der Software}
Das Planungstool wird als jar-Datei mit dem Namen Planungstool.jar ausgeliefert.
Eine manuelle Installation ist nicht nötig. 
Die jar-Dateien können in einem beliebigen Ordner platziert werden.
Es wird empfohlen, alle Dateien, die während der Planung benötigt oder erzeugt werden, in demselben Ordner wie die jar-Datei zu speichern. 
Auf Windows kann die jeweilige jar-Dateien  mittels eines Doppelklicks ausgeführt werden.
Auf Linux sollte das jeweilige Programm über die Kommandozeile mittels folgendem Befehl ausgeführt werden:

\begin{verbatim}
    java -jar Planunungstool.jar
\end{verbatim}

\subsection{Übertragung eines bestehenden Stundenplans von Untis in das lokale Planungstool}
\label{Stundenplanübertragung}
Falls bereits ein Stundenplan in Untis existiert, kann dieser importiert werden. 
Dafür exportieren wir die Daten aus Untis als Textdateien im Data Interchange Format (DIF) und übertragen diese anschließend in das Format der lokalen Planungsdatenbank.
Bei den DIF-Dateien handelt es sich um strukturierte Textdateien, die die relevanten Informationen über die Klassen, Lehrer, Räume, Fächer, Unterricht, Stundenplan, Stundentafel und Zeitwünsche enthalten.
Auf eine genaue Spezifikation des Formates wird an dieser Stelle verzichtet, da es für die Anwendung nicht relevant ist, und bereits eine offizielle Dokumentation von Untis existiert (siehe: \url{https://www.untis.at/manual/index.html?ti_allgemeine-schnittstellen.htm}). 
Der Planer hat die folgenden zwei Möglichkeiten, um die Daten von Untis in das lokale Planungstool zu übertragen:
\begin{itemize}
	\item Nutzen des Importskripts export\_untis.bat, welches die DIF-Dateien automatisch erstellt und in das Format der lokalen Planungsdatenbank überträgt.
    \item Manueller Export der DIF-Dateien aus der Untis-GUI oder mit Hilfe des Command Line Interface (CLI) und anschließendes Ausführen der Importskripte, um die DIF-Dateien in das Format der lokalen Planungsdatenbank zu übertragen.
\end{itemize}
Wir empfehlen die erste Methode, da sie einfacher und schneller ist. Die zweite Methode bietet keinen Mehrwert gegenüber der Methode, da sie die Schritte des Importskripts manuell ausführt.
Wir wollen diese trotzdem beschreiben, um bei möglichen Problemen mit dem Importskript eine alternative Möglichkeit zu bieten, um die Daten von Untis in das lokale Planungstool zu übertragen.

Da Untis anders als unser lokales Planungstool ausschließlich auf Windows läuft, muss der Datenexport unter Windows vorgenommen werden.
Die exportierten Daten können anschließend bei Bedarf auf den Linux-PC verschoben werden.
Der Zeitpunkt und das Vorgehen sind abhängig von der gewählten Export-Variante und werden im jeweiligen Unterkapitel beschrieben.

\subsubsection{Nutzen des Importskripts export\_untis.bat}
Um das Importskript zu verwenden muss sich der Planer auf einem Computer mit einem Windows System befinden, da das Skript als .bat Datei vorliegt, welche nur auf Windows Systemen ausgeführt werden kann.
\todo{Beschreiben wo dieses liegt und wie es gestartet wird (Doppelklick oder commandzeile).}
Nach dem Starten des Skripts wird der Planer aufgefordert den Pfad zu der Untis.exe Datei anzugeben (siehe \cref{fig:export_untis_step1}). Hierbei ist es egal, ob der Pfad mit Anführungszeichen oder ohne angegeben wird.
Im Beispiel in \cref{fig:export_untis_step1} ist der Pfad zur Untis.exe Datei Rot markiert und ohne Anführungszeichen angegeben. Sobald der Pfad eingegeben wurde, muss die Eingabe mit Enter bestätigt werden.
Nach Bestätigung mit Enter, wird der Planer aufgefordert den Pfad zu der .untis Datei anzugeben. Dies ist die Datei, welche von Planer in Untis geöffnet wird und alle benötigten Daten für den Stundenplan enthält (siehe \cref{fig:export_untis_step2}). 
Auch hier ist es egal, ob der Pfad mit Anführungszeichen oder ohne angegeben wird. Sobald der Pfad eingegeben wurde, muss die Eingabe mit Enter bestätigt werden (siehe \cref{fig:export_untis_step2}).
Danach läuft das Skript automatisch durch und führt die benötigten Schritte aus, um die DIF-Dateien zu erstellen und in das Format der lokalen Planungsdatenbank zu übertragen.
Die beendete Durchführung des Skripts wird mit einer Erfolgsmeldung (siehe \cref{fig:export_untis_step3}) bestätigt.
\begin{figure}[htbp]
    \centering
    \begin{subfigure}[b]{0.32\textwidth}
        \centering
        \includegraphics[width=\textwidth]{figures/export_untis_1.png}
        \caption{Der Pfad zur Untis.exe Datei ist Rot markiert. Enter drücken, um fortzufahren.}
        \label{fig:export_untis_step1}
    \end{subfigure}
    \hfill % Fügt flexiblen Abstand zwischen den Bildern ein
    \begin{subfigure}[b]{0.32\textwidth}
        \centering
        \includegraphics[width=\textwidth]{figures/export_untis_2.png}
        \caption{Der Pfad zur .untis Datei ist Rot markiert. Enter drücken, um fortzufahren.}
        \label{fig:export_untis_step2}
    \end{subfigure}
	\hfill
	\begin{subfigure}[b]{0.32\textwidth}
        \centering
        \includegraphics[width=\textwidth]{figures/export_untis_3.png}
        \caption{Erfolgreiche Durchführung des Skripts}
        \label{fig:export_untis_step3}
    \end{subfigure}
    \label{fig:untis_export}
\end{figure}
Die Resultierende lokale Planungsdatenbank wird im selben Ordner wie das Importskript unter dem Namen \textbf{STULP.db} erstellt.
Diese Datei kann sowohl unter Windows als auch unter Linux im lokalen Planungstool geöffnet werden und der importierte Stundenplan kann bearbeitet werden.

\subsubsection{Manueller Export aus Untis}

Für den manuellen Export aus Untis können entweder das Untis-CLI oder die Untis-GUI verwendet werden.
Anschließend muss der Inhalt der erzeugten DIF-Dateien mittels Python-Skripten in eine von STULP lesbare SQLite-Datenbank übertragen werden.

\paragraph{Manueller Export der DIF-Dateien aus der Untis-GUI}
Wie bereits erwähnt gibt es die Möglichkeit die benötigten DIF-Dateien manuell aus der Untis-GUI zu exportieren.
Wir empfehlen diese Methode nur, wenn es Probleme mit dem Importskript gibt, da sie aufwendig und fehleranfällig ist.
Zunächst muss Untis gestartet und die entsprechende Datenbank geöffnet werden. Navigieren Sie anschließend im Menü oben links über \textit{Datei} $\rightarrow$ \textit{Import/Export} $\rightarrow$ \textit{Export TXT Dateien}. Dieser Ablauf ist in \cref{fig:untis_export_steps} dargestellt.

\begin{figure}[htbp]
    \centering
    \begin{subfigure}[b]{0.32\textwidth}
        \centering
        \includegraphics[width=\textwidth]{figures/dif_erstellen_step_1.png}
        \caption{Klick auf \textit{Datei}}
        \label{fig:step1}
    \end{subfigure}
    \hfill % Fügt flexiblen Abstand zwischen den Bildern ein
    \begin{subfigure}[b]{0.32\textwidth}
        \centering
        \includegraphics[width=\textwidth]{figures/dif_erstellen_step_2.png}
        \caption{Klick auf \textit{Import/Export}}
        \label{fig:step2}
    \end{subfigure}
    \hfill
    \begin{subfigure}[b]{0.32\textwidth}
        \centering
        \includegraphics[width=\textwidth]{figures/dif_erstellen_step_3.png}
        \caption{Klick auf \textit{Export TXT}}
        \label{fig:step3}
    \end{subfigure}
    \caption{Der Navigationspfad zum Export-Menü in Untis}
    \label{fig:untis_export_steps}
\end{figure}

Nach Auswahl von \textit{Export TXT Dateien} erscheint rechts unter „Schnittstellen“ eine Auflistung aller verfügbaren Datentabellen (siehe \cref{fig:dif_export_tables}).

\begin{figure}[htbp]
    \centering
    \includegraphics[width=0.6\textwidth]{figures/dif_erstellen_step_4.png}
    \caption{Auflistung der verfügbaren Datentabellen in Untis}
    \label{fig:dif_export_tables}
\end{figure}

Durch Klicken auf eine Tabelle öffnet sich das Fenster für die Exportoptionen. Hierbei sind folgende Einstellungen essenziell, damit Umlaute und Sonderzeichen korrekt übertragen werden:

\begin{itemize}
    \item \textbf{Trennzeichen:} Komma
    \item \textbf{Textbegrenzung:} Anführungszeichen (")
    \item \textbf{Zeichensatz (Encoding):} UTF-8
\end{itemize}

Standardmäßig ist in Untis oft ein anderer Zeichensatz eingestellt; dies muss manuell auf UTF-8 (andernfalls kommt es zu Fehlern bei der Datenübertragung) geändert werden. 
Bestätigen Sie die Einstellungen anschließend mit \textit{OK}.

\begin{figure}[htbp]
    \centering
    \includegraphics[width=0.6\textwidth]{figures/dif_erstellen_step_5.png}
    \caption{Korrekte Exportoptionen (UTF-8) für eine Tabelle}
    \label{fig:dif_export_options}
\end{figure}

Der Planer wird nun aufgefordert, einen Speicherort zu wählen. Sobald dies geschehen ist, wird die Datei im DIF-Format abgelegt. Dieser Vorgang muss für alle relevanten Tabellen wiederholt werden. Die benötigten Tabellen sind:

\begin{itemize}
    \item Klasse
    \item Lehrer
    \item Räume
	\item Fächer
	\item Unterricht
	\item Stundenplan
	\item Stundentafel
	\item Zeitwünsche
\end{itemize}
Sobald alle Tabellen exportiert wurden, können die DIF-Dateien in das Planungstool importiert werden.

\FloatBarrier

\paragraph{Export via Command Line Interface}
Es gibt auch die Möglichkeit, sich alle DIF-Dateien auf einmal ausgeben zu lassen. 
Dafür ist folgender Befehl in der Kommandozeile einzugeben:
\begin{verbatim}
Untis.exe <Pfad zur .untis Datei> /exp*
\end{verbatim}

Die DIF-Dateien werden dann im aktuellen Verzeichnis abgelegt (der Name einer eventuell angegebenen Ausgabedatei wird beim Export aller Dateien ignoriert).

Möchte man hingegen nur eine spezifische Datei an einen bestimmten Ort exportieren (zum Beispiel Lehrerstammdaten), lautet der Befehl:

\begin{verbatim}
Untis.exe <Pfad zur .untis Datei> /exp1=<Pfad zur Ausgabedatei>
\end{verbatim}

Dabei ist zu beachten, dass die Exportfunktion über die Kommandozeile nicht die Möglichkeit bietet, die Zeichensatzkodierung auf UTF-8 zu setzen.
Dadurch können die im nächsten Unterpunkt beschriebenen Python-Skripte die DIF-Dateien nicht korrekt lesen, da sie nicht auf UTF-8 kodiert sind.
Um dies zu umgehen, müssen die DIF-Dateien manuell in UTF-8 konvertiert werden. Wie dies funktioniert, ist leicht im Internet zu recherchieren und hängt von dem Betriebssystem ab, welches verwendet wird.

\paragraph{Übertragung der DIF Dateien in das lokale Planungstool}

Nachdem die DIF Dateien erstellt wurden, müssen nacheinander die mitgelieferten Python-Skripte mit folgenden Befehlen ausgeführt werden:

\begin{verbatim}
    python dif-relevant-columns.py
    python dif-to-stulp.py
\end{verbatim}

Dieser Schritt kann sowohl unter Windows als auch unter Linux durchgeführt werden, wobei wichtig ist, dass die DIF-Dateien im gleichen Verzeichnis wie die Python-Skripte liegen.
Bei der Ausführung auf debian-basierten Linux-Distributionen (z.~B. Debian, Ubuntu und Mint) ist jedoch zu beachten, dass der Befehl \texttt{python} durch den Befehl \texttt{python3} ersetzt werden muss.
Alternativ kann unter Linux aber auch das Bash-Skript \texttt{untis-to-stulp.sh} ausgeführt werden.

Nach der Ausführung dieses Skripts wird die Datei \textbf{STULP.db} erstellt.
Diese kann plattformunabhängig im lokalen Planungstool geöffnet werden.

\subsection{Nutzung der lokalen Planungssoftware}
\label{STULP_Doku}
Das Planungstool ermöglicht die Zeit- und Raumplanung für die Kurse.

\subsubsection{Übersicht über die Planungssoftware}

\begin{figure}[htbp]
	\centering
	\includegraphics[width=1\textwidth, angle=90]{images/planning_tool/overview.png}
	\caption{Übersicht Planungssoftware}
	\label{fig:plt_overview}
\end{figure}
\FloatBarrier
\newpage 
In Abbildung \ref{fig:plt_overview} ist eine Übersicht über das Planungstool zu sehen. Das Planungstool besteht aus 6 Teilen. Diese sind: 
\begin{enumerate}
	\item Menüleiste für die Programmsteuerung und Einstellungen. Mithilfe der Menüleiste wird das Programm gesteuert und es können wichtige Einstellungen vorgenommen werden. 
	\item Hotkeyleiste, um oft benötigte Funktionen direkt und mittels Tastaturkommandos auszulösen.
	\item Anzeige, um die derzeitigen Konflikte anzuzeigen und zu durchsuchen. Näher erklärt in Kapitel \ref{subsubsec:plt_conf}.
	\item Anzeige, um die nicht geplanten Veranstaltungen anzuzeigen und zu durchsuchen. 
	\item Primärer Stundenplan für die Veranstaltungsplanung.
	\item Sekundärer Stundenplan für die Veranstaltungsplanung.
    \item Anzeige, welche Datenbank gerade geöffnet ist.
\end{enumerate}
Mithilfe der Teile 4, 5, 6 werden die Veranstaltungen geplant. Dazu werden Veranstaltungen mittels „Drag and Drop“ zwischen verschiedenen Stundenplänen bewegt und somit zum Beispiel die Uhrzeit oder der Raum geändert.


\subsubsection{Planungsdatenbanken öffnen}
Der Knopf \textbf{Aus Datei} im Menüabschnitt \textbf{Datenbank} ermöglicht das Öffnen einer lokalen Planungsdatenbank. 
Diese muss auf dem System des Benutzers erreichbar sein. Das Öffnen hat den folgenden Ablauf:
\begin{enumerate} 
	\item Knopf \textbf{Aus Datei} im Menüabschnitt \textbf{Datenbank} drücken
	\item Auswählen der lokalen Planungsdatenbank aus dem Dateisystem des Computers 
	\item Knopf \textbf{Öffnen} drücken
\end{enumerate}
\todo{Anordnung hier anpassen, dass das neue Kaptel auf der nächsten Seite beginnt, aber nicht }
\begin{wrapfigure}{R}{0.35\textwidth}
	\includegraphics[scale=0.5]{images/planning_tool/before_opening_db.png}
    \caption{Datenbanken öffnen}
	\label{fig:before_opening_db}
\end{wrapfigure}

% \paragraph{Web Planungsdatenbank öffnen (VLPTDB)}\label{sec:open-vlptdp}
% Ebenfalls über den Knopf \textbf{Aus Datei} im Menüabschnitt \textbf{Datenbank} 
% lässt sich eine VLPTDB im Format des Webservers öffnen. Diese lässt sich im Interface des Webservers  in das lokale Dateisystem auf dem Computer des Planenden herunterladen.
% Wird ein solches Datenbankschema erkannt, muss nach dem Bestätigen des Meldungsfensters ein Speicherort 
% für die neu zu erstellende lokale Planungsdatenbank ausgewählt werden. Das Öffnen hat den folgenden Ablauf:
% \begin{enumerate}
% 	\item Knopf \textbf{Aus Datei} im Menüabschnitt \textbf{Datenbank} drücken
% 	\item Auswählen der Web Planungsdatenbank (VLPTDB) auf dem System
% 	\item Knopf \textbf{Öffnen} drücken
% 	\item Meldungsfeld bestätigen
% 	\item Neuen Speicherort für die lokale Planungsdatenbank auswählen
% 	\item Knopf \textbf{Speichern} drücken
% \end{enumerate}

% \paragraph{Neue VLPT Daten in bestehende Planungsdatenbank laden}
% Sollen neue Daten vom Webprogramm geladen werden, kann eine VLPTDB genau wie in Abschnitt \ref{sec:open-vlptdp} geladen werden. Allerdings muss nun als Speicherort, eine bereits existierende lokale Planungsdatenbank gewählt werden. Die neuen Daten werden dann als ungeplante Veranstaltungen in der lokalen Planungsdatenbank sichtbar sein. 

% \paragraph{Fremddaten aus PAUL importieren}
% Es können mit Hilfe des PAUL-Tools Fremddaten geladen werden. Die mit dem  PAUL-Tool  erstellte Datenbank kann in dem Dialog hinter dem ausgegrauten \textbf{Aus PAUL-Export} in Abbildung \ref{fig:before_opening_db} ausgewählt werden. Die Datenbank wird von dem  lokalen Planungstool  gelesen und  die darin enthaltenen  Fremddaten in die lokale Planungsdatenbank importiert. Der Knopf  \textbf{Aus PAUL-Export}  ist erst benutzbar, sobald eine lokale Planungsdatenbank im lokalen Tool geöffnet ist. Der Knopf ist ab dann nicht mehr ausgegraut.

% \paragraph{Lokalen Planungsstand hochladen}
% Im  p2tool (Webserver)  lässt sich der Planungsstand  in der Rolle des Planenden oder Admins  unter \textbf{Planungsstand importieren/exportieren} hochladen und auch wieder herunterladen.
\subsubsection{Anzeige der Konflikte}\label{subsubsec:plt_conf}
\begin{wrapfigure}{R}{0.35\textwidth}
    \centering
	\caption{Konfliktanzeige}
	\includegraphics[scale=0.6]{images/planning_tool/conflicts.png}
	\label{fig:plt_conflicts}
\end{wrapfigure}
In Teil 3 von Abbildung \ref{fig:plt_overview} werden alle momentan existierenden nicht ignorierten Konflikte in einer Tabelle angezeigt.
Die Tabelle kann mithilfe eines Filters nach Veranstaltung und Typ des Konflikts gefiltert werden. 
Das Planungswerkzeug ermöglicht es, existierende Konflikte zu ignorieren. 
Für den Planer bedeutet das, dass diese Konflikte nicht mehr in der Tabelle und Stundenplan angezeigt werden.
\paragraph{Konfliktabelle}\label{par:conf_table}
Die Tabelle 1 in Abbildung \ref{fig:plt_conflicts} zeigt alle momentan existierenden nicht ignorierten Konflikte an. 
Dabei repräsentiert eine Zeile der Tabelle exakt einen Konflikt zwischen zwei Veranstaltungen.
Für einen Konflikt werden jeweils die KursID in der Spalte KursID und die Koodierte KursID als Veranstaltungsteil in der Spalte Teil der beteiligten Veranstaltungsteile angezeigt.
Zusätzlich wird in der Spalte Konflikt das Konfliktobjekt angezeigt. 
Dieses Konfliktobjekt ist entweder ein Lehrer, ein Raum oder ein Klasse.
Die erste Zeile der Tabelle in Abbildung \ref{fig:plt_conflicts} Teil 1 zeigt zum Beispiel den Konflikt zwischen der Veranstaltung  \textit{Chemie in der 10A} und der Veranstaltung \textit{Chemie in der 10B} an. 
Der Konflikt ist ein Raumkonflikt für den Raum mit der Raumnummer \textit{B2.03}.
Mittels der Konflikttabelle können angezeigte Konflikte ignoriert werden. 
Dazu muss lediglich mittels eines Rechtsklicks auf die Zeile des Konflikts das Kontextmenü geöffnet werden und der Menüpunkt \textbf{Ignorieren} ausgewählt werden. 
Zudem kann der Stundenplan des Konfliktobjekts aufgerufen werden. 
Dazu muss im Kontextmenü der Menüpunkt \textbf{Gehe zu} ausgewählt werden.

\paragraph{Filtern der Konfliktliste}
Die Konflikttabelle kann mittels des Filters 2 in Abbildung \ref{fig:plt_conflicts} gefiltert werden. 
Es stehen ein Veranstaltungsfilter (Veranstaltung) und ein Konflikttypfilter (Typ) zur Verfügung. 
Mittels des Veranstaltungsfilters kann eine Veranstaltung ausgewählt werden. 
Die Konflikttabelle zeigt dann nur Konflikte an, die zu dieser Veranstaltung gehören. 
Mittels des Konflikttypfilters kann ein Konflikttyp ausgewählt werden. 
Die Konflikttabelle zeigt dann nur Konflikte an, die den ausgewählten Konflikttyp haben.
Falls beide Filter aktiv sind, werden in der Konflikttabelle nur die Konflikte angezeigt, die beiden Filtern genügen.\\
Ein Filter ist aktiv, sobald der Kippschalter des Filters sich in der rechten Stellung befindet und eine gültige Eingabe für den Wert des Filter ausgewählt wird. 
Ein Filter ist deaktiviert, sobald sich der Kippschalter des Filters in der linken Stellung befindet. 
Durch Klicken auf den Kippschalter kann die Stellung verändert werden. 
In \ref{fig:plt_conflicts} ist zum Beispiel der Veranstaltungsfilter deaktiviert und es wird nicht nach einer speziellen Veranstaltung gefiltert. 
Der Konflikttypfilter ist ebenfalls deaktiviert. 

\paragraph{Konflikt Prioritäten}
In Teil 3 von Abbildung \ref{fig:plt_conflicts} können die Prioritäten der verschiedenen Konflikttypen verwaltet werden. 
Standardmäßig existieren 4 Konflikttypen: Raum, Professor, StudiengangLeicht und StudiengangSchwer. 
Beim Erstellen einer lokalen Planungsdatenbank werden diese Konflikttypen mit der Priorität 1 versehen. 
Es gibt die Prioritäten 1-5, wobei 1 die höchste Priorität ist und 5 die niedrigste. 
Über den Knopf \textbf{Prios verwalten} können diese Prioritäten angepasst werden. 
Es öffnet sich ein Fenster: Abbildung \ref{fig:plt_conflict_prios}. 
\begin{wrapfigure}{r}{0.4\textwidth}
    \centering
	\caption{Prioritäten zuweisen}
	\includegraphics[scale=0.7]{images/planning_tool/conflict_prio_dialog.png}
	\label{fig:plt_conflict_prios}
\end{wrapfigure}
Die Farbanzeige der Konflikte im Planungsfenster richtet sich nach der Konflikt-Prioritäten Zuordnung. 
Wenn Raumkonflikte die Priorität 1 besitzen und die Farbe für Konflikte der Priorität 1 rot ist, dann werden diese rot angezeigt. 
Zudem können über den Schalter \textbf{Prio} Konflikte ausgeblendet werden.
Wenn im Plan nur die Konflikte mit einer Priorität größer als 3 (also Priorität 1 und 2), kann der Schalter aktiviert werden und in dem Eingabefeld eine 2 eingetragen werden.  

\paragraph{Eigene Konflikte}
Es besteht die Möglichkeit im Tool per SQL-Query eigene Konflikte zu definieren. 
Diese existieren dann zusätzlich zu den standardmäßig unterstützten Konflikttypen. 
Beispielsweise kann ein Konflikt \textbf{WunschOrt} definiert werden, der auftritt, falls eine Veranstaltung für das Gebäude F geplant wird, obwohl als Wunschort ein anderes Gebäude angegeben ist.
Diese Funktion birgt Risiken für den Benutzer, da die Queries bestimmte Eigenschaften erfüllen müssen. 
\textbf{Besteht der Wunsch diese Funktion zu nutzen, lesen Sie sich auch die Implementations Dokumentation durch.} 
Als Beispiel: \texttt{SELECT * FROM Plan WHERE Raumnr NOT LIKE 'F\%' AND Ort = 1 AND Raumnr NOT LIKE 'kein Raum'}. 
Diese Query wählt alle Plan-Einträge aus, deren Raumnummer nicht mit „F“ beginnt (Demnach nicht im F Gebäude geplant wurden), jedoch als Wunschort (Ort) das F Gebäude (Wert 1) angegeben haben.
Ein solcher Konflikt kann entweder im Konflikt-Menü in Abbildung \ref{fig:plt_conflict_menu} unter \textbf{Eigenen Konflikt hinzufügen} hinzugefügt werden oder in Abbildung \ref{fig:plt_conflicts} über das \textbf{+}. 
Über das sich öffnende Fenster in Abbildung \ref{fig:plt_add_custom_conflict} kann der Konflikttyp dann eingegeben werden. \\
\todo{Hier überlegen, ob wir als Beispiel vielleicht doch Raum zu klein nehmen sollten}
\begin{figure}[htbp]
    \centering
    \begin{minipage}{0.5\textwidth}
        \centering
        \caption{Eigenen Konflikttyp erstellen}
        \includegraphics[scale=0.4]{images/planning_tool/add_custom_conflict.png}
        
        \label{fig:plt_add_custom_conflict}
    \end{minipage}
    \hspace{0.05\textwidth}
    \begin{minipage}{0.4\textwidth}
        \centering
        \caption{Konflikte Menü}
        \includegraphics[scale=0.4]{images/planning_tool/conflict_menu.png}
        
        \label{fig:plt_conflict_menu}
    \end{minipage}
\end{figure}

Die eigenen Konflikte können zusätzlich Im Dialog Fenster aus Abbildung \ref{fig:plt_custom_conflicts_manage} aktiviert und deaktiviert werden. 
So können eigene Konflikte aus der Übersicht genommen werden, ohne dass sie aus der Datenbank gelöscht werden müssen. 

\begin{figure}[h]
    \centering
    \caption{Eigene Konflikte verwalten}
    \includegraphics[scale=.55]%{images/planning_tool/manage_custom_conflicts.png}
    {images/planning_tool/pic08}
    \label{fig:plt_custom_conflicts_manage}
\end{figure}

%\begin{wrapfigure}{r}{0.4\textwidth}
%	\caption{Eigene Konflikte verwalten}
%	\includegraphics[scale=0.7]{images/planning_tool/manage_custom_conflicts.png}
%	\label{fig:plt_custom_conflicts_manage}
%\end{wrapfigure}

\paragraph{Verwaltung der ignorierten Konflikte}

Im Menü-abschnitt Konflikte können ignorierte Konflikte verwaltet werden. 
Durch das Drücken des Knopfes \textbf{Zurücksetzen} im Menüabschnitt Konflikte werden alle ignorierten Konflikte zu nicht-ignorierten Konflikten. 
Eine Verwaltung der Konflikte ist mittels eines Dialogs, der durch Drücken des Knopfes Verwalten im Konflikt-Menüabschnitt geöffnet wird, möglich.\\
%\begin{wrapfigure}{r}{0.4\textwidth}
%	\caption{Konflikte verwalten}
%	\includegraphics[scale=0.35]{images/planning_tool/conflicts_manage.png}
%	\label{fig:plt_conflicts_manage}
%\end{wrapfigure}
Im Dialog in \ref{fig:plt_conflicts_manage} werden die momentan nicht-ignorierten Konflikte in der Konflikttabelle 1 angezeigt. 
Die momentan ignorierten Konflikte werden in der Konflikttabelle 3 angezeigt. 
Mittels der 4 Knöpfe in 2 können einzelne oder alle Konflikte zwischen den beiden Konflikttabellen verschoben werden. 
Die Knöpfe haben die folgenden Funktionen (von oben nach unten):
\begin{enumerate}
	\item ignoriert den momentan selektierten Konflikte aus Konflikttabelle 1
	\item ignoriert alle Konflikte aus Konflikttabelle 1
	\item unignoriert den momentan selektierten Konflikte aus Konflikttabelle 3
	\item unignoriert alle Konflikte aus Konflikttabelle 3
\end{enumerate}
Die Veränderungen können mittels \textbf{Annehmen} gespeichert oder \textbf{Abbrechen} verworfen werden.
\begin{figure}[htbp]
    \centering
    \begin{minipage}{0.45\textwidth}
        \centering
        \caption{Konflikte verwalten}
        \includegraphics[scale=0.45]{images/planning_tool/conflicts_manage.png}
        \label{fig:plt_conflicts_manage}
    \end{minipage}
    \hspace{0.05\textwidth}
    \begin{minipage}{0.45\textwidth}
       \centering 
    
	  \includegraphics[scale=0.45]{images/planning_tool/unplanned.png}
   \caption{Ungeplante Veranstaltungen}
   \label{fig:plt_unplanned} 
    \end{minipage}
\end{figure}

\FloatBarrier

%\begin{wrapfigure}{HR}{0.38\textwidth}
%	\centering
%	\caption{Ungeplante Veranstaltungen}
%	\includegraphics[scale=0.40]%{images/planning_tool/unplanned.png}
%	\label{fig:plt_unplanned}
%\end{wrapfigure}

\paragraph{Unwiederufliches Ignorieren von Konflikten mittels Konfigurationsdatei}

Es gibt auch die Möglichkeit, Konflikte durch Einträge in der Konfigurationsdatei \texttt{IgnoredConflicts.json} unwiederruflich zu ignorieren.
Eine Beispieldatei wurde mit der Software ausgeliefert.
Sie ist im JSON-Format geschrieben und enthält die folgenden drei Elemente, die nachfolgend erläutert werden.
Dabei ist zu beachten, dass sowohl die Datei als auch die jeweiligen Elemente nicht existieren müssen bzw. leer gelassen werden können.

\begin{itemize}

    \item \textit{StudiengangLeicht}

    \item \textit{StudiengangSchwer}

    \item \textit{AlternierendeKurseProVeranstaltungsteil}

\end{itemize}

\subparagraph{Die Elemente \textit{StudiengangLeicht} und \textit{StudiengangSchwer} in der Konfigurationsdatei}

Diese Elemente enthalten jeweils die Unterelemente \textit{KonfliktfreieStudiengaenge} und \textit{ParalleleKurse}, welche jeweils eine Liste enthalten.
Bei dem Element \texttt{KonfliktfreieStudiengaenge} handelt es sich um eine Liste der Studiengänge, bei denen leichte bzw. schwere Studiengangskonflikte dauerhaft ignoriert werden sollen.
Das Element \texttt{ParalleleKurse} besteht aus Listen (paralleler) Kurse, bei denen die resultierenden leichten bzw. schweren Studiengangskonflikte dauerhaft ignoriert werden sollen.

\subparagraph{Das Element \textit{AlternierendeKurseProVeranstaltungsteil} in der Konfigurationsdatei}

Dieses Element enthält eine Liste von Listen.
Jede der enthaltenen Listen besteht aus genau zwei Unterlisten, von denen die erste Veranstaltungsteile und die zweite immer genau zwei Kurse enthält.
Pro Eintrag gilt, dass alle Konflikte zwischen den beiden Kursen ignoriert werden, sofern sie den gleichen Veranstaltungsteil haben, und der Veranstaltungsteil in der zugehörigen Veranstatltungsteilliste ist.

\subsubsection{Ungeplante Veranstaltungen}
In Teil 4 von Abbildung \ref{fig:plt_overview} werden alle momentan ungeplanten Veranstaltungen in einer Tabelle angezeigt. 
Die Tabelle kann mithilfe eines Filters nach Veranstaltungen gefiltert werden. 

\paragraph{Ungeplante Veranstaltungstabelle}\label{par:unplan_table}
Die Tabelle 1 in \ref{fig:plt_unplanned} zeigt alle momentan ungeplanten Veranstaltung an, also diejenigen, die keinen Raum, Tag und Uhrzeit haben.
Dabei repräsentiert eine Zeile der Tabelle exakt einen ungeplante Veranstaltung. 
Für einen ungeplanten Veranstaltung wird die KursID in der Spalte KursID, der Veranstaltungsteil in der Spalte Teil, der Professor in der Spalte ProfID und die Dauer der Veranstaltung in der Spalte Dauer angezeigt.\\ 
Die erste Zeile der Tabelle in Abbildung \ref{fig:plt_unplanned} zeigt zum Beispiel die ungeplante Veranstaltung D L1 in der Q1 gehalten vom Lehrer mit der ProfID JA. 
Die Veranstaltung dauert zwei Schulstunden.
\paragraph{Filtern der Ungeplanten Veranstaltungstabelle}
Die Tabelle kann mittels des Filters 2 in \ref{fig:plt_unplanned} gefiltert werden.  
% Es steht ein Veranstaltungsfilter zur Verfügung. 
% Mittels des Veranstaltungsfilters kann eine Veranstaltung ausgewählt werden. 
% Die Tabelle zeigt dann nur ungeplante Veranstaltungsteile an, die zu dieser Veranstaltung gehören.\\
Ein Filter ist aktiv, sobald der Kippschalter des Filters sich in der rechten Stellung befindet und eine gültige Eingabe für den Wert des Filters ausgewählt wird. 
Ein Filter ist deaktiviert, sobald sich der Kippschalter des Filters in der linken Stellung befindet. 
Durch Klicken auf den Kippschalter kann die Stellung verändert werden. 
In \ref{fig:plt_unplanned} ist zum Beispiel der Veranstaltungsfilter aktiv und es wird nach Veranstaltungsteilen mit der KursID D L1 gefiltert.


\subsubsection{Planen von Veranstaltungen}
Das Planen von Veranstaltungen wird mittels des primären und sekundären Stundenplanfensters und der ungeplanten Veranstaltungstabelle durchgeführt. 
Dazu können die Elemente der Stundenpläne und der ungeplanten Veranstaltungstabelle mittels Drag- and Drop-Gesten zwischen den Stundenplänen und innerhalb eines Stundenplans verschoben werden. 
Dabei werden die, durch die Verschiebung entstanden Änderungen an Raum und Zeit der Veranstaltung, gespeichert. 
Das Verschieben einer Veranstaltung zu der ungeplanten Veranstaltungstabelle entfernt die Zeit und den Raum der Veranstaltung. Dieser ist somit ungeplant.

\paragraph{Stundenplantypen}\label{par:agenda_types}
Die Planungssoftware unterstützt 5 verschiedene Stundenplantypen. Jeder Stundenplantyp zeigt verschiedene Veranstaltungsteile an.
\begin{table}[ht]
	\begin{tabular}{|p{0.17\textwidth}|p{0.73\textwidth}|}
		\hline
		Typ & Angezeigte Veranstaltungen\\\hline
		Raum & Veranstaltungen, die in dem Raum stattfinden\\
		Professor& Veranstaltungen, die vom Lehrer gehalten werden\\
		Studiengang& Veranstaltungen, die von einer Klasse besucht werden\\
		Veranstaltung & Veranstaltungen, die von diesem Typ gehalten werden\\
		Filter &  beliebige Veranstaltungen, die mittels Filtern aus den anderen Stundenplantypen erstellt werden können  \\\hline
	\end{tabular}
\end{table}
\paragraph{Aufbau des primären und sekundären Stundenplanfensters}
In einem Stundenplanfenster (primär und sekundär) findet die eigentliche Planung der Veranstaltungen statt.\\

In der Menüleiste unter Planung befindet sich für das primäre und das sekundäre Stundenplanfenster jeweils eine Menügruppe mit Knöpfen, mit denen das jeweilige Stundenplanfenster manipuliert werden kann. 
Die Knöpfe, zu sehen in  Abbildung  \ref{fig:plt_menu_plan}, bieten die folgenden Funktionen an: 
\begin{enumerate}
	\item blendet das jeweilige Stundenplanfenster aus
	\item öffnet einen Raum-Stundenplantab
	\item öffnet einen Professor-Stundenplantab
	\item öffnet einen Veranstaltung-Stundenplantab
	\item öffnet einen Studiengang-Stundenplantab
	\item öffnet einen Filter-Stundenplantab
\end{enumerate}
\begin{figure}[h]
	\centering
	\caption{Menü Stundenplanfenster}
	\includegraphics[scale=0.5]{images/planning_tool/menu_planung.png}
	\label{fig:plt_menu_plan}
\end{figure}
%\begin{figure}[H]
%   \centering
%    	\caption{Auswahl Stundenplantab}
%    	\includegraphics[scale=0.4]{images/planning_tool/menu_plan_popup.png}
%    	\label{fig:plt_plan_popup}
%\end{figure}

%\begin{figure}[H]
%      \centering
%		\caption{Aufbau Stundenplanfenster}
%		\includegraphics[scale=0.35]{images/planning_tool/plan_window.png}
%		\label{fig:plt_pimary_plan}
%\end{figure}

%\begin{figure}[H]
 %   \centering
  %  \begin{minipage}{0.45\textwidth}
  %      \centering
%		\caption{Aufbau Stundenplanfenster}
%		\includegraphics[scale=0.25]{images/planning_tool/plan_window.png}
%		\label{fig:plt_pimary_plan}
%    \end{minipage}
%    \hspace{0.05\textwidth}
%    \begin{minipage}{0.45\textwidth}
%        \centering
%    	\caption{Auswahl Stundenplantab}
%    	\includegraphics[scale=0.4]{images/planning_tool/menu_plan_popup.png}
%    	\label{fig:plt_plan_popup}
%    \end{minipage}
%\end{figure}

\begin{wrapfigure}{r}{0.4\textwidth}
    \centering
	\caption{Auswahl Stundenplantab}
	\includegraphics[scale=0.6]{images/planning_tool/menu_plan_popup.png}
	\label{fig:plt_plan_popup}
\end{wrapfigure}

Beim Öffnen eines Raum-, Professor-, Veranstaltung- oder Studiengang-Stundenplantabs muss zusätzlich mittels eines Menüs, 
zu sehen in Abbildung \ref{fig:plt_plan_popup}, das Objekt des Stundenplans ausgewählt werden. 
Dazu muss lediglich das gewünschte Objekt in der Liste ausgewählt werden und die Auswahl mittels des Knopfes \textbf{Öffne Stundenplan} bestätigt werden.
Um die Übersicht zu erleichtern kann über das Suchfeld nach einem speziellen Objekt gesucht werden.\\
In Abbildung \ref{fig:plt_pimary_plan} ist eine Beispielabbildung für das primäre Stundenplanfenster zu sehen (die Planung mit dem sekundären Stundenplanfenster ist analog).
Die derzeit geöffneten Stundeplantabs können in der Leiste in 1 gesehen werden. 
Mittels Klick auf das Tab wird der jeweilige Stundenplan des Objekts geöffnet. 
Der aktuell geöffnete Tab ist hell hinterlegt. 
Die Tabs lassen sich mittels Drag- and Drop-Gesten zwischen dem primären und sekundären Stundenplanfenster verschieben. 
Dazu muss der gewünschte Tab in der Liste in die Tabliste des anderen Stundeplanfensters mittels Drag- and Drop Geste  gezogen  werden.\\ 
In Teil 2 von Abbildung \ref{fig:plt_pimary_plan}  werden Informationen über das Objekt des Stundenplans angezeigt, für einen Raum zum Beispiel Raumnummer und Sitzplatzanzahl.\\

\begin{figure}[h]
      \centering
		\caption{Aufbau Stundenplanfenster}
		\includegraphics[scale=0.45]{images/planning_tool/plan_window.png}
		\label{fig:plt_pimary_plan}
\end{figure}

Die Toolbar in  Abbildung \ref{fig:plt_pimary_plan} Teil 3  bietet einige Knöpfe an, um die Darstellung des aktuell angezeigten Stundenplans zu manipulieren. 
Die Knöpfe, zu sehen in \ref{fig:plt_agenda_toolbar}, bieten die folgenden Funktionen an:
\begin{enumerate}
	\item Öffnet die eine Spalte mit verfügbaren Räumen (aktuell ohne Effekt)
	\item Vergrößert die Breite einer Zelle im Stundenplan (horizontales Hineinzoomen)
	\item Verringert die Breite einer Zelle im Stundenplan (horizontales Herauszoomen)
	\item Vergrößert die Höhe einer Zelle im Stundenplan (vertikales Hineinzoomen)
	\item Verringert die Höhe einer Zelle im Stundenplan (vertikales Herauszoomen)
    \item Wechselt die Anzeige zwischen \textit{nur 2 Veranstaltungen nebeneinander} (für mehr Übersichtlichkeit) 
	und \textit{alle Veranstaltungen nebeneinander} (um Screenshots zu machen, auf denen alle VAs zu sehen sind)
	\item Öffnet den Vollbildmodus für den Stundenplan (maximales horizontales und vertikales Herauszoomen)
\end{enumerate}

\begin{figure}[h]
	\centering
	\caption{Toolbar Stundenplan}
	\includegraphics[scale=0.6]{images/planning_tool/plan_toolbar.png}
	\label{fig:plt_agenda_toolbar}
\end{figure}

Der eigentliche Stundenplan des Objekts ist in  Abbildung \ref{fig:plt_pimary_plan} Teil 4 zu sehen.
\paragraph{Veranstaltungen im Stundenplan}
In einem Stundenplan werden je nach Stundenplantyp die jeweiligen Veranstaltungen angezeigt. 
Ein Veranstaltung wird mithilfe eines Rechtecks an der von Tag und Uhrzeit bestimmten Stelle im Stundenplan angezeigt. 
Die folgenden Informationen werden im Rechteck für einen Veranstaltung in Abbildung \ref{fig:plt_single} angezeigt:

\begin{enumerate}
	\item KursID und Nummer der Veranstaltung
	\item Raum, in dem die Veranstaltung stattfindet
	\item ProfID des Lehrers, der die Veranstaltung hält
	\item Klassen, zu denen die Veranstaltung gehört. Falls der Veranstaltung zu mehr als einem Klasse gehört, 
	lässt sich mittels des Pfeils eine Liste aller betroffenen Studiengänge anzeigen 
 \item Anzeige ob für diesen Veranstaltung ein Konflikt existiert, dessen Konflikttyp manuell hinzugefügt wurde. Falls ein Konflikt existiert, wird dieses Feld farblich hervorgehoben
\end{enumerate}

\begin{figure}[h]
	\centering
	\caption{Veranstaltung im Stundenplan}
	\begin{subfigure}{0.3\linewidth}
        \centering
		\caption{}
		\includegraphics[scale=0.8]%{images/planning_tool/plan_single.png}
        {images/planning_tool/pic01.png}
		\label{fig:plt_single}
	\end{subfigure}
	\begin{subfigure}{0.3\linewidth}
    \centering
		\caption{}
		\includegraphics[scale=0.8]%{images/planning_tool/plan_single_conflict.png}
        {images/planning_tool/pic02.png}
		\label{fig:plt_single_conflict}
	\end{subfigure}
	\begin{subfigure}{0.3\linewidth}
    \centering
	\caption{}
	\includegraphics[scale=0.3]%{images/planning_tool/plan_multiple.png}
    {images/planning_tool/pic03.png}
	\label{fig:plt_multiple}
\end{subfigure}
\end{figure}

Je nach Veranstaltungsart wird das Rechteck mit einer anderen Farbe ausgefüllt. 
Wenn für Raum, Lehrer oder ein der Klassen ein Konflikt vorliegt wird dieser farbig hinterlegt. 
Standardmäßig  wird  rot für einen schweren und gelb für einen leichten Konflikt genutzt.
Das Rechteck in Abbildung \ref{fig:plt_single_conflict} visualisiert somit die folgenden Informationen über die Veranstaltung. 
Die Veranstaltung \textit{D 9.4} findet im Raum \textit{B1.12} statt, wir von dem Lehrer mit der ProfID \textit{AM} gehalten und gehört zu der Klasse \textit{09D}.
Zudem existiert für die Veranstaltung ein leichter Studiengangskonflikt.
Wenn mehrere Veranstaltungsteile einer Zelle des Stundenplans zugeordnet sind, werden diese nebeneinander angezeigt. So zu sehen in \ref{fig:plt_multiple}. \\

Die Start- und Endzeit des Stundenplans kann frei konfiguriert werden. Mehr dazu in \autoref{par:settings_time}. 
Daher kann es passieren, dass Veranstaltungen nur teilweise oder gar nicht angezeigt werden. 
Falls ein Veranstaltung nur partiell angezeigt wird, erscheint ein Warndreieck im zugehörigen Rechteck. 
Zudem erscheint ein rotes Warndreieck in der linken oberen Ecke des Stundenplans, falls ein Veranstaltung existiert die nur partiell oder gar nicht angezeigt wird. 
Beide Indikatoren sind in \autoref{fig:plt_agenda_indicator} zu sehen. 
Der Veranstaltungsteil \textit{M 9.4} geht von 8 bis 10 Uhr.
Da der Stundenplan allerdings erst um 9 Uhr beginnt, wird die Veranstaltung nur partiell angezeigt.

\begin{figure}[ht]
	\centering
	\caption{Warnung Stundenplan}
	\includegraphics[scale=0.5]%{images/planning_tool/agenda_warning_indicator.png}
    {images/planning_tool/pic04.png}
	\label{fig:plt_agenda_indicator}
\end{figure}

Durch Rechtsklick auf eine Veranstaltung öffnet sich ein Kontextmenü. 
Mittels des Kontextmenüs kann sofort der Stundenplan der Veranstaltung, des Professors, des Raumes oder des Studiengangs geöffnet werden.\\
Falls der Stundenplantyp Professor ist, werden zusätzlich zu den Rechtecken für die Veranstaltungen, 
ebenfalls die Sperrzeiten des Lehrers im Stundenplan mittels dunkelroten Rechtecken in den jeweiligen Zellen angezeigt. 
Die Anzeige von Sperrzeiten in einem Stundenplan ist in \ref{fig:plt_plan_wunsch_sperr} zu sehen. 

% \paragraph{Verfügbare Räume Spalten}\label{par:available_rooms}
% \begin{wrapfigure}{}{0.53\linewidth}
% 	\centering
% 	\caption{Verfügbare Räume Spalten}
% 	\includegraphics[scale=0.5]%{images/planning_tool/agenda_available_rooms.png}
%     {images/planning_tool/pic05.png}
% 	\label{fig:plt_agenda_available_rooms}
% \end{wrapfigure}

% Wenn die verfügbaren-Räume-Spalten eingeblendet werden, wird für jeden Tag eine zusätzliche Spalte in dem Stundenplan angezeigt. 
% Für jeden Zeitslot existiert in der Spalte eine Liste. In der Liste wird angezeigt welche Räume für den Tag und die Zeit frei sind. 
% Eine Veranstaltung kann mithilfe einer Drag- and Drop-Geste direkt in die Liste bewegt werden, um die Veranstaltung direkt zu dem Raum zu bewegen. 
% In \autoref{fig:plt_agenda_available_rooms} ist ein Beispiel für die verfügbare-Räume-Spalte zu sehen. 
% In Teil 1 von \autoref{fig:plt_agenda_available_rooms} werden alle verfügbaren Räume für den Zeitslot \textit{Montag 9 Uhr} angezeigt. 
% Das heißt, in den Räumen \textit{F0 530} und \textit{F2 211} findet keine Veranstaltung am \textit{Montag um 9 Uhr} statt. 
% Dasselbe gilt für \textit{F0 530}, \textit{F2 211} und \textit{L 1} am \textit{Montag um 10 Uhr} nach Teil 2 der Abbildung.
% \todo{Beispiel überarbeiten}
%\FloatBarrier

\paragraph{Planen einer Veranstaltung}
Ein Veranstaltungsteil kann innerhalb eines Stundenplans, zwischen zwei Stundenplänen und der ungeplanten Veranstaltungstabelle mittels einer Drag- and Drop-Geste bewegt werden. 
Wenn ein Veranstaltungsteil innerhalb eines Stundenplans bewegt wird, wird lediglich der Tag und die Uhrzeit entsprechend verändert. 
Bei zwei verschiedenen Stundenplänen (beide müssen den Typ Raum haben) wird der Tag, die Uhrzeit und der Raum entsprechend geändert. 
Bei einer Bewegung von einem Stundenplan zur ungeplanten Veranstaltungstabelle wird der Tag, die Uhrzeit und der Raum für den Veranstaltungsteil gelöscht. Er ist danach ungeplant.
Bei einer Drag- and Drop-Geste wird angezeigt, ob eine Bewegung erlaubt ist oder nicht. Dazu wird der Zielzeitslot grün bei einer erlaubten und rot bei einer verbotenen Bewegung eingefärbt. 
In Abbildung \ref{fig:plt_plan_move_allowed} ist eine erlaubte Bewegung und in Abbildung \ref{fig:plt_plan_move_forbidden} eine verbotene Bewegung zu sehen.  
Eine erlaubte Bewegung ist eine Bewegung von (ungeplanten) Veranstaltungen in Räume oder in die Ansichten, die zu den Veranstaltungsdaten passen. 
Eine Veranstaltung die von Lehrerin X gehalten wird, kann nur in die Ansicht von Lehrerin X verschoben werden. 
Wenn versucht wird eine Veranstaltung in eine abweichende Ansicht (z.~B. Lehrerin Y) zu verschieben, handelt es sich um eine verbotene Bewegung.
Darunter fällt auch das Verschieben ungeplanter Veranstaltung in eine Nicht-Raum Ansicht.  
Falls eine Drag- and Drop-Geste für eine verbotene Bewegung durchgeführt wird, werden keine Veränderungen ausgelöst.
Falls für den bewegten Veranstaltung Wunschzeiten und für den Lehrer des Veranstaltung Sperrzeiten bekannt sind, werden diese während der Drag- and Drop-Geste in dem Stundenplan, 
über dem sich die Maus befindet, angezeigt: Wunschzeiten des Veranstaltungsteils in dunkelgrün und Sperrzeiten des Lehrers dunkelrot. 
So zu sehen in Abbildung \ref{fig:plt_plan_wunsch_sperr}.
\begin{figure}[htbp] %htbp
	%\centering
    \caption{Angezeigte Informationen bei einer Bewegung}
	\begin{subfigure}{0.33\linewidth}
    \centering
		\caption{Erlaubte Bewegung}
		\includegraphics[scale=0.4]%{images/planning_tool/plan_allowed_move.png}
        {images/planning_tool/pic06.png}
		\label{fig:plt_plan_move_allowed}
	\end{subfigure}
	\begin{subfigure}{0.33\linewidth}
        \centering
		\caption{Verbotene Bewegung}
		\includegraphics[scale=0.4]%{images/planning_tool/plan_forbidden_move.png}
        {images/planning_tool/pic07.png}
		\label{fig:plt_plan_move_forbidden}
	\end{subfigure}
	\begin{subfigure}{.33\linewidth}
		\centering
		\caption{Wunsch- und Sperrzeiten}
		\includegraphics[scale=0.5]{images/planning_tool/plan_wunsch_sperr.png}
		\label{fig:plt_plan_wunsch_sperr}
	\end{subfigure}
 
\end{figure}


\paragraph{Stundenplan für Raum „Ohne Raum“}
Um die Planung zu erleichtern, gibt es den Raum „Ohne Raum“.
Veranstaltungen, die sich in diesem Raum befinden, haben einen Tag und eine Uhrzeit zugewiesen, jedoch keinen Raum. 
Dieser Raum dient als Zwischenablage für Veranstaltungen, für die schon ein Tag und eine Uhrzeit, aber noch kein Raum geplant ist. 
\newpage 


\subsubsection{Übersicht über Daten}
Der Menüabschnitt \textbf{Daten} (Abbildung \ref{fig:plt_menu_data}) bietet einige Funktionen an, um einen Überblick über die Daten zu gewinnen und diese zu manipulieren.

\begin{figure}[h]
	\caption{Daten Übersicht}
	\centering
	\includegraphics[scale=0.65]{images/planning_tool/menu_daten_neu.png}
	\label{fig:plt_menu_data}
\end{figure}


\paragraph{Überblick}
Der Knopf \textbf{Überblick} im Menüabschnitt \textbf{Daten} ermöglicht einen Überblick über alle Sperrzeiten und Wunschzeiten der Kurse der Professoren. 
In dem sich öffnenden Dialog wird für jeden Professor ein Stundenplan angezeigt. 
In dem Stundenplan werden nur die Sperrzeiten des Professors und die Wunschzeiten von Veranstaltungen des Professors angezeigt. 
Es kann ebenfalls durch einen Filter nach einem bestimmten Professor gesucht werden, dessen Stundenplan dann groß angezeigt wird.

\paragraph{Raum hinzufügen}
\begin{wrapfigure}{}{0.35\linewidth}
\centering
        \caption{Raum hinzufügen}
    	\includegraphics[scale=0.6]{images/planning_tool/add_room.png}
    	\label{fig:plt_menu_add_room}
\end{wrapfigure}

Der Knopf \textbf{Überblick} im Menüabschnitt \textbf{Daten} ermöglicht es dem Planer einen Raum zur Planungsdatenbank hinzuzufügen.
Falls eine Veranstaltung in einem Raum geplant werden soll, der in der Planungsdatenbank noch nicht existiert, kann der Raum folgendermaßen hinzugefügt werden:

\begin{enumerate}
	\item Knopf \textbf{Raum hinzufügen} im Menüabschnitt \textbf{Daten} drücken. Ein Fenster wie in Abbildung \ref{fig:plt_menu_add_room} öffnet sich
	\item Raumnummer eingeben
	\item Gebäude, in dem sich der Raum befindet, eingeben
	\item Sitzplatzzahl des Raums eingeben
	\item Priorität des Raumes angeben. Ein Raum mit einer hohen Priorität wird weiter oben in den Raumlisten angezeigt.
	\item \textbf{Einreichen} drücken, um Raum der Planungsdatenbank hinzuzufügen 
\end{enumerate}
\FloatBarrier

Ebenfalls können über den Knopf \textbf{Räume aus Datei hinzufügen} Räume aus einer Text-Datei gelesen werden. 
Die Eintrage der Textdatei müssen dem Format: \\
 \texttt{<Raumnummer>-<Gebäude>-<Priorität>-<Hörerzahl>} 
(Beispiel: \texttt{A 1-A-w-250}) entsprechen. Es können auch CSV Dateien gelesen werden. 
Dabei entspricht jede Zeile einem Raum-Eintrag. 
Die Formatierung ist ähnlich zu der in der Text-Datei: \\
 (\texttt{<Raumnummer>;<Gebäude>;} \texttt{<Priorität>;<Hörerzahl>}). 
In Tabellenform ist jeder Wert in einer eigenen Zelle. Es sollten keine zusätzlichen Texte in der Datei stehen.

\paragraph{Verfügbare Räume}
Der Knopf \textbf{Verfügbare Räume} im Menüabschnitt \textbf{Daten} ermöglicht es dem Planer für jeden Zeitslot die verfügbaren Räume einzusehen. 
Somit kann der Planer auf einen Blick sehen, welche Räume für einen Zeitslot noch frei sind.

\begin{wrapfigure}{}{0.35\linewidth}
        \caption{Raum sperren}
    	\centering
    	\includegraphics[scale=0.6]{images/planning_tool/raum_sperren.png}
    	\label{fig:plt_block_room}
\end{wrapfigure}

\paragraph{Räume sperren}
Über den Knopf \textbf{Räume sperren} können Räume gesperrt werden. 
In der sich öffnenden Maske (Abbildung \ref{fig:plt_block_room}) müssen die Felder ausgefüllt werden und bestätigt werden. 
Dabei ist das Feld \textbf{Kommentar} optional. Der ausgewählte Raum ist dann zu dieser Zeit als blockiert angezeigt.
\FloatBarrier

\paragraph{Überprüfung}
Durch Klicken auf den Knopf \textbf{Überprüfung} im Menüabschnitt Daten öffnet sich das Überprüfungsfenster. 
Das Fenster ist in Abbildung \ref{fig:plt_daten_check} zu sehen. Es fungiert als Abschlussüberprüfung vor Abschluss der Phase 4. 
Die folgenden Informationen werden angezeigt:
\begin{enumerate}
	\item alle Konflikte 
	\item alle ignorierten Konflikte 
	\item alle Veranstaltungen, die keinen Raum zugewiesen haben
	\item alle Veranstaltungen, die keinen Raum und keinen Tag und keine Uhrzeit zugewiesen haben
\end{enumerate}
Für die Darstellung der Konflikte, ignorierten Konflikte und Veranstaltungen werden die bereits erklärten Tabellen aus Abschnitt 
\ref{par:conf_table} (Konflikttabelle) und \ref{par:unplan_table} (Ungeplante Veranstaltungen) verwendet.
Ein Stundenplan ist erst bereit für die Veröffentlichung, wenn keine Veranstaltungen mehr ohne Raum oder ohne Raum, Tag und Uhrzeit existieren. 
Der Benutzer kann frei entscheiden, ob er die Phase 4 abschließen möchte, obwohl noch Konflikte zwischen Veranstaltungen existieren.
\begin{figure}[ht]
    \centering
		\caption{Daten überprüfen}
	\includegraphics[scale=0.4]{images/planning_tool/daten_check.png}
	\label{fig:plt_daten_check}
\end{figure}

\subsubsection{Einstellungen}
Im Menüabschnitt Einstellungen, zu sehen in \ref{fig:plt_menu_settings_overview}, können Einstellungen für das Planungstool vorgenommen werden. 
Es können Wochentage und Uhrzeiten im Stundenplan, die Farben der Rechtecke im Stundenplan und Standardwerte für das Webprogramm  eingestellt werden. 
Die Einstellungen werden automatisch lokal auf dem Computer gespeichert und beim nächsten Öffnen des Planungswerkzeugs geladen.
\begin{figure}[h]
    %\centering
	\caption{Einstellungen Übersicht}
	\includegraphics[scale=0.6]{images/planning_tool/settings.png}
	\label{fig:plt_menu_settings_overview}
\end{figure}
\paragraph{Stundenplan}\label{par:settings_time}
\begin{wrapfigure}{R}{0.35\textwidth}
    \centering
	\caption{Einstellungen Stundenplan}
	\includegraphics[scale=0.55]{images/planning_tool/menu_settings_stundenplan.png}
	\label{fig:plt_menu_settings_stundenplan}
\end{wrapfigure}
Für die Stundenpläne im Stundenplanfenster lässt sich das folgende einstellen (zu sehen in \ref{fig:plt_menu_settings_stundenplan}):
\begin{enumerate}
	\item Anfangs- und Endzeit des Stundenplans. Der linke Knopf bestimmt die Anfangszeit. Der rechte Knopf bestimmt die Endzeit.
	\item Spalten für Samstag und Sonntag aus- und einblenden
\end{enumerate}
In Abbildung \ref{fig:plt_menu_settings_stundenplan} gehen die Stundepläne von 6 Uhr bis 19 Uhr. 
Die Spalten für Samstag und Sonntag werden nicht angezeigt.
\paragraph{Farben}
Für die Farben der Rechtecke in einem Stundenplan lässt sich das folgende einstellen (zu sehen in \ref{fig:plt_menu_settings_farben}):

\begin{enumerate}
	\item Farbe für Konflikte der Priorität 1
	\item Farbe für Konflikte der Priorität 2
    \item Farbe für Konflikte der Priorität 3
    \item Farbe für Konflikte der Priorität 4
    \item Farbe für Konflikte der Priorität 5
	\item Farbe für eine Vorlesung
	\item Farbe für eine Übung
	\item Farbe für eine Zentralübung
	\item Farbe für einen importierten Kurs
	\item Farbe für eine Wunschzeit eines Veranstaltungsteils
	\item Farbe für eine Sperrzeit eines Professors	
\end{enumerate}

Bei Klick auf einen der Knöpfe für die Farben öffnet sich ein Fenster, in dem die jeweilige Farbe frei konfiguriert werden kann. So zu sehen in \ref{fig:plt_menu_settings_farben_choose}.
\begin{figure}[h]
\centering
\caption{Farbeinstellungen} 
	\begin{subfigure}{0.7\linewidth}
    \centering
		\caption{}
		\includegraphics[scale=0.5]{images/planning_tool/color_settings.png}
		\label{fig:plt_menu_settings_farben}
	\end{subfigure}
	\begin{subfigure}{0.29\linewidth}
    \centering
		\caption{}
		\includegraphics[scale=0.5]{images/planning_tool/menu_settings_farben_choose.png}
		\label{fig:plt_menu_settings_farben_choose}
	\end{subfigure}
    
\end{figure}

\subsubsection{Weitere Hinweise}

\begin{enumerate}
    \item Die Anzeige hinter dem Knopf \textbf{Verfügbare Räume} unter dem Reiter \textbf{Daten} zeigt keine Daten an. Dieser Fehler konnte nicht rechtzeitig behoben werden.
\end{enumerate}
