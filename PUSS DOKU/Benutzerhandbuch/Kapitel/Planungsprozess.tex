\section{Der Planungsprozess}
\subsection{Grundbegriffe der Planung}
In diesem Abschnitt wird paragraphenweise jeder Begriff erklärt, der für den Planungsprozess notwendig ist.
%
\paragraph*{(Lehr)veranstaltung}
Lehrveranstaltungen bilden den zentralen Bestandteil des Planungsprozesses. 
Diese werden von Lehrern gehalten und vermitteln den Schülern Wissen über ein bestimmtes Unterrichtsfach.
Jede Lehrveranstaltung wird zu einer festen Zeit, an einem bestimmten Ort durchgeführt und hat eine bestimmte Dauer.
Der vereinfachte Begriff „Veranstaltung“ wird im weiteren Verlauf dieses Dokuments synonym für „Lehrveranstaltung“ verwendet.
% \paragraph*{Lehrveranstaltungsteil}
% Ein Lehrveranstaltungsteil ist immer Teil einer Lehrveranstaltung und besteht aus der Dauer (Anzahl Stunden) und einer Lehrveranstaltungsart. Einem Lehrveranstaltungsteil kann im Rahmen der  lokalen  Planung ein Tag, eine Uhrzeit und ein Raum zugewiesen werden, an dem dieser stattfinden soll.
%
% \paragraph*{Lehrveranstaltungsart}
% Die Lehrveranstaltungsart bestimmt für einen Lehrveranstaltungsteil, welche Art von Lehre in diesem Teil stattfindet. Gültige Lehrveranstaltungsarten sind beispielsweise Vorlesungen, Zentralübungen, Übungen, Proseminare, Seminare, Projektgruppen, Exkursionen, Praktika oder Oberseminare.
%
\paragraph*{Klasse}
Ein Klasse besteht aus einer Menge von Schülern, die gemeinsam bestimmte Lehrveranstaltungen besuchen. 
Jede Klasse ist dabei einem bestimmten Jahrgang zugeordnet.
Ein Klasse hat immer eine feste Anzahl von Schülern und besucht eine bestimmte Menge von Lehrveranstaltungen im Laufe eines Schuljahres.
\paragraph*{Jahrgang}
Ein Jahrgang ist eine Menge von Klassen. In jedem Jahrgang müssen bestimmte Lehrveranstaltungen besucht werden. 
Diese Veranstaltung können in bestimmten Fällen von mehreren Klassen gemeinsam besucht werden.
\paragraph{Oberstufenjahrgang}
Ein Oberstufenjahrgang ist ein spezieller Jahrgang, der sich aus Schülern zusammensetzt, die sich auf das Abitur vorbereiten. 
In der Oberstufe haben die Schüler die Möglichkeit, aus einer Vielzahl von Lehrveranstaltungen zu wählen, um ihren individuellen Interessen und Stärken gerecht zu werden. 
Die Planung der Lehrveranstaltungen in den Oberstufenjahrgängen erfordert besondere Aufmerksamkeit, da die Schüler unterschiedliche Fächerkombinationen wählen können und somit eine flexible Gestaltung des Stundenplans notwendig ist.
% \paragraph{Schuljahr}
% Ein Schuljahr ist der Zeitraum, in dem der Unterricht an einer Schule stattfindet.
% Es beginnt in der Regel im August oder September und endet im Juni oder Juli des folgenden Jahres.
\paragraph*{Raum}
Ein Raum ist eine Ressource, die von der Schule bereitgestellt wird. Sie ist insofern notwendig für einen Veranstaltung, als dass diese ohne einen Raum nicht stattfinden kann.
%
\paragraph*{Lehrperson}
Eine Lehrperson ist eine von der Schule zum halten von Lehrveranstaltung zur Verfügung gestellte menschliche Ressource.
Jeder Lehrveranstaltung ist genau einer Lehrerperson (es gibt seltene Ausnahmen mit mehreren Lehrpersonen pro Lehrveranstaltung), % jedoch kann ein*e Lehrer*in mehrere Lehrveranstaltungen anbieten.
Wir werden im folgenden den Begriff „Lehrer“ als geschlechtsneutralen Begriff für „Lehrperson“ verwenden.
%
\paragraph*{Raumkonflikt}
Ein Raumkonflikt ist ein zentrales Problem während der Planung von Lehrveranstaltungen.
Ein Raumkonflikt liegt vor, wenn zwei Lehrveranstaltungsteile eine zeitliche Überschneidung vorweisen und in demselben Raum stattfinden.
%
\paragraph*{Lehrerkonflikt}
Ein Lehrerkonflikt ist ein zentrales Problem während der Planung von Lehrveranstaltungen. 
Ein Lehrerkonflikt liegt vor, wenn zwei Lehrveranstaltungsteile, die von demselben Lehrer gehalten werden, eine zeitliche Überschneidung vorweisen.
%
\paragraph*{Klassenkonflikt}
Ein Klassenkonflikt ist ein zentrales Problem während der Planung von Lehrveranstaltungen. 
Ein Klassenkonflikt liegt vor, wenn zwei Lehrveranstaltungsteile der selben Klasse, die von der Klasse verpflichtend belegt besucht werden müssen, eine zeitliche Überschneidung vorweisen.
Unterschieden wird dabei noch zwischen leichten und schweren Studiengangskonflikten. 
Ein schwerer Konflikt besteht, wenn die betroffende Veranstaltung eine Pflichtveranstaltung ist und es keine alternativen Termine gibt. 
%
\paragraph*{Lokales Planungstool}
Das Tool zur lokalen Planung wird in diesem Dokument hauptsächlich „lokales Planungstool“  oder „Planungstool“ genannt.
%
% \paragraph*{p2tool}
% Das p2tool ist der Webserver, der die Eingaben der Lehrenden entgegen nimmt. Alternativ auch einfach „Webserver“ oder „Webprogramm“.
% \subsection{Überblick über den Planungsprozess}
% \todo{Diese Subsection auf die Schulen anpassen}
% Der Planungsprozess ist in fünf Phasen unterteilt, von denen nicht alle jedes Semester durchlaufen werden.

% \paragraph*{Phase 1}
% In Phase 1 entscheidet jede Fakultät, welche Lehrveranstaltungen von ihr für welchen Studiengang angeboten werden sollen. Diese Phase wird allerdings nur alle  sieben bis acht Jahre pro Studiengang wiederholt.  Dies geschieht im Rahmen der Reakkreditierung oder wenn neue Kolleg*innen berufen werden.  Da diese nicht für jeden Studiengang im selben Jahr stattfindet, hat dies Einfluss auf die in Phase 2 erstellten Modulhandbücher, weil diese auch Lehrveranstaltungen von anderen Studiengängen enthalten, die als Nebenfach belegt werden können.

% \paragraph*{Phase 2}
% In Phase 2 erstellt  der/die für einen Studiengang zuständige  Studiengangsbeauftragte das Modulhandbuch für den Studiengang, den er verwaltet. Dazu selektiert er Lehrveranstaltungen aus der Liste der angebotenen Lehrveranstaltungen aus Phase 1 und versieht diese mit ECTS-Punkten. Aufgrund  diverser Änderungen des Lehrangebots  wird diese Phase  etwa  alle ein bis drei Jahre wiederholt.

% \paragraph*{Phase 3}
% In Phase 3 wählen die Lehreren Lehrveranstaltungen aus den Modulhandbüchern ihrer Studiengänge, die sie anbieten möchten. Ein Lehrer kann dabei keine Lehrveranstaltung anbieten, die nicht Teil eines Modulhandbuchs ist.  Für Seminare, Proseminare und Projektgruppen gibt es jedoch folgende Aussage:  Ein Lehrer darf einen  Veranstaltungstitel  wählen, sofern es ihm im Modulhandbuch freigestellt ist. Diese Phase wird jedes Semester wiederholt.

% \paragraph*{Phase 4}
% In dieser Phase wird jedem Lehrveranstaltungsteil jeder Lehrveranstaltung eines Studiengangs ein Raum, Tag und eine Uhrzeit durch den Planer des Studiengangs zugewiesen, sodass möglichst keine Studiengangskonflikte, Lehrerkonflikte oder Raumkonflikte entstehen. Der Planer hat zudem die Möglichkeit auf diverse Wünsche der Lehreren bei der Planung einzugehen. Diese Phase kann mehrfach pro Semester durchlaufen werden, falls nötig.  Dies ist der Fall, wenn Veranstaltungen nachgemeldet werden oder Meldungen geändert werden. Für diese Phase typische Ausgaben sind: Vorlesungsplan als .pdf, Semesterpläne für z.~B. Informatik-Master, Informatik-Bachelor im Semester 1,3,5 oder 2,4,6.

% \paragraph*{Phase 5}
% Phase 5 ist eine Phase der Rückmeldung für den Planer.  Es gibt das Vorlesungsverzeichnis als .pdf und .tex. Große Räume werden an die zentrale Raumverwaltung gemeldet und Pläne werden für alle Lehrenden, Studierendengruppen und Fakultätsräte erstellt und ausgehändigt. Zudem werden alle weiteren Räume am Campus und in der Fürstenallee gemeldet. Alle Beteiligten   haben nun die Möglichkeit dem Planer mitzuteilen, ob ihnen etwas an ihren Plänen unter Angabe triftiger Gründe missfällt. Sollte Nachbesserungsbedarf bestehen, werden Phase 4 und 5 wiederholt.
\subsection{Überblick über den Planungsprozess an Schulen}
Der Planungsprozess an Schulen lässt sich, ähnlich wie an Hochschulen, in phasenhafte Abschnitte unterteilen. Diese werden jedoch meist im Turnus eines Schuljahres durchlaufen.

\paragraph*{Phase 1: Rahmenvorgaben und Schulentwicklung}
In dieser Phase werden die langfristigen Rahmenbedingungen festgelegt. Dies umfasst die Vorgaben der Kultusministerien (Kontingentstundentafeln, Kernlehrpläne) sowie schulspezifische Schwerpunkte, die durch die Schulkonferenz beschlossen werden (z.~B. MINT-Profil, Sprachzweige). Diese Phase wiederholt sich in großen Abständen, etwa bei der Einführung neuer Bildungspläne (G8/G9-Umstellung) oder tiefgreifenden Strukturänderungen der Schule. Sie bildet das Fundament für die Stundentafeln der einzelnen Jahrgänge.

\paragraph*{Phase 2: Klassenbildung und Bedarfsermittlung}
Die Schulleitung ermittelt auf Basis der Anmeldezahlen und der Versetzungsstatistik die Anzahl der benötigten Klassen und Kurse für das kommende Schuljahr. Hierbei werden die Stundentafeln auf die konkreten Jahrgänge angewendet (z.~B. „Klasse 7 benötigt 4 Stunden Deutsch“). Auch das Angebot im Wahlpflichtbereich (WPI/WPII) oder bei Arbeitsgemeinschaften wird hier definiert. Diese Phase findet jährlich vor Beginn des neuen Schuljahres statt.

\paragraph*{Phase 3: Lehrverteilung und Wunscherfassung}
Im Gegensatz zur Universität wählen Lehrkräfte ihre Kurse meist nicht frei aus, sondern werden gemäß ihrer Fakultas (Lehrbefähigung) eingesetzt. In dieser Phase geben die Lehrkräfte ihre Wünsche bzgl. Fächern, Klassenstufen und zeilichen Verfügbarkeiten (z.~B. teilzeitbedingte freie Tage) an. Die Schulleitung erstellt daraufhin die sogenannte Lehrverteilung (Deputatsverteilung), die festlegt, welche Lehrkraft welche Klasse in welchem Fach unterrichtet. Diese Phase ist die direkte Vorstufe zur konkreten Zeitplanung.

\paragraph*{Phase 4: Stundenplanerstellung (Konstruktionsphase)}
In dieser Phase werden die Unterrichtseinheiten (Paarung aus Lehrer, Klasse und Fach) konkreten Zeitfenstern und Räumen zugewiesen. Dies erfolgt meist softwaregestützt (z.~B. mit Untis). Ziel ist die Vermeidung von Kollisionen (Lehrer- oder Raumdoppelbelegungen) sowie die Einhaltung pädagogischer und organisatorischer Kriterien (z.~B. keine Hohlstunden für Schüler, gleichmäßige Verteilung von Hauptfächern). Auch die Raumplanung für Fachräume (Chemie, Sport, Computer) erfolgt in diesem Schritt. Diese Phase ist die rechenintensivste und komplexeste im jährlichen Zyklus.

\paragraph*{Phase 5: Veröffentlichung und Feinjustierung}
Nach der Fertigstellung wird der vorläufige Stundenplan an das Kollegium und die Schülerschaft ausgegeben (z.~B. als PDF, Aushang oder digital via WebUntis). In dieser Rückmeldephase haben Lehrkräfte die Möglichkeit, auf gravierende Probleme oder übersehene Restriktionen hinzuweisen. Nach Einarbeitung valider Änderungswünsche wird der Plan finalisiert und an die entsprechenden Schnittstellen (digitales Klassenbuch, Vertretungsplanungs-Software) übergeben. Kleinere Anpassungen erfolgen danach nur noch im laufenden Betrieb bei akuten Änderungen.
\subsection{Rollen im Planungsprozess}
Der Planungsprozess beinhaltet verschiedene Rollen, die lediglich Zugriff auf die Daten erhalten, die sie zur Ausführung ihrer Aufgabe benötigen.
%
% \paragraph*{Lehrveranstaltungsbeauftragte*r}
% Der oder die Lehrveranstaltungsbeauftragte ist für die Erstellung der Liste aller Kurse einer Fakultät verantwortlich. Dazu muss er in der Lage sein, Kurse zu erstellen. Er oder sie hat lediglich Zugriff auf die von ihm/ihr erstellten Kurse.  Es bietet sich an, dass der oder die Lehrveranstaltungsbeauftragte eines Studiengangs eng mit der oder dem Studiengangsbeauftragten des Studiengangs zusammenarbeitet - oder dass es die selbe Person ist. 
% %
% \paragraph*{Studiengangsbeauftragte*r}
% Der oder die Studiengangsbeauftragte ist für die Erstellung und Wartung der Modulhandbücher seiner Studiengänge verantwortlich. Zur Durchführung seiner Aufgabe benötigt er die Berechtigung, Modulhandbücher zu erstellen. Er hat Zugriff auf die Liste der angebotenen Kurse und auf die Modulhandbücher, die er verwaltet.
% %
\paragraph*{Lehrer*in}
Lehrende sind für das Anbieten von Lehrveranstaltungen verantwortlich.
Dazu benötigen sie Zugriff auf die Liste der angebotenen Lehrveranstaltungen, um aus diesen die Lehrveranstaltungen auszuwählen, die sie in dem jeweiligen Schuljahr anbieten möchten.
Nach der Auswahl der Lehrveranstaltungen, die sie anbieten möchten, sollen sie zudem die Möglichkeit haben, zeitwünsche für die Planung der Lehrveranstaltungen zu äußern.
Anschließend wird diesen nach Abschluss der Planung eine Stundenübersicht ihrer Lehrveranstaltungen für das kommende Schuljahr zur Verfügung gestellt.
%
\paragraph*{Planer*in}
Diese Rolle trägt die Verantwortung für die Stundenplanung. 
Zur effektiven Durchführung ist der vollständige Zugriff auf alle Lehrveranstaltungen sowie sämtliche planungsrelevanten Stammdaten (Räume, Lehrkräfte, Klassen, Jahrgänge etc.) erforderlich. 
Um fehlende oder fehlerhafte Informationen korrigieren zu können, beinhaltet dies auch die entsprechenden Bearbeitungsrechte. 
Darüber hinaus erhält die planende Person Zugriff auf das Planungstool und die Funktion, das aktuelle Vorlesungsverzeichnis aus den Daten zu exportieren.
%
% \paragraph*{Admin}
% Die Rolle Admin ist für die Benutzerverwaltung zuständig. Sie trägt neue User ein und ordnet ihnen die korrekte Rolle zu.
