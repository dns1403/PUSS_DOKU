\section{Einleitung}

Dieses Dokument dient als Begleitmaterial für das Stundenplanunterstützungstool (PUSS Tool), das im Rahmen der Projektgruppe PUSS an der Universität Paderborn entwickelt wurde. Es richtet sich an die Anwender der Software und bietet Hilfestellung sowohl bei der Installation der benötigten Komponenten als auch bei der praktischen Nutzung des Programms.

\subsection*{Voraussetzungen und Zielgruppe}
Die Anleitung setzt voraus, dass der Anwender über grundlegende Computerkenntnisse verfügt und mit der Bedienung von Standardsoftware vertraut ist. Zwar werden spezifische technische Begriffe aus dem Kontext der Softwareentwicklung verwendet, diese werden jedoch weitestgehend erläutert, um auch weniger erfahrenen Anwendern den Zugang zu erleichtern.

Ziel dieses Dokuments ist es, eine klare und verständliche Anleitung für die erfolgreiche Installation und Verwendung des Programms bereitzustellen. Es wird empfohlen, die beschriebenen Schritte sorgfältig und in der angegebenen Reihenfolge durchzuführen, um Fehler zu vermeiden.

\subsection*{Aufbau des Dokuments}
Das Handbuch ist in thematische Kapitel unterteilt, die verschiedene Aspekte der Installation und Bedienung abdecken. Zur besseren Verständlichkeit enthält jedes Kapitel detaillierte Anweisungen, Screenshots und Beispiele.

In Kapitel 3 werden zwei grundlegende Nutzungsszenarien unterschieden, die sich nach der Art der Datenbereitstellung richten:
\begin{enumerate}
    \item \textbf{Import eines bestehenden Stundenplans (Untis):} Falls bereits ein Stundenplan in Untis existiert, ist Kapitel \todo{Ref} essenziell. Es beschreibt den korrekten Export der Daten aus Untis und deren Übertragung in das lokale Planungstool.
    \item \textbf{Erstellung eines neuen Stundenplans:} Liegt kein Datenbestand vor, kann das Kapitel \todo{Ref} übersprungen werden, da es sich primär mit der Datenmigration befasst.
\end{enumerate}

Für Rückfragen oder bei Problemen während der Installation und Nutzung stehen die Autoren zur Verfügung. Die entsprechenden Kontaktdaten entnehmen Sie bitte der Titelseite dieses Dokuments.