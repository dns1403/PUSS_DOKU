\section{Einleitung}
Dieses Dokument dient als Begleitmaterial des Studenplanunterstützungstools, welches im Rahmen der Projektgruppe PUSS an der Universität Paderborn entwickelt wurde. 
Es richtet sich an den Anwender des Programms und soll eine Hilfestellung bei der Installation der benötigten Softwarekomponenten sowie der Nutzung des Programms bieten.
Es wird davon ausgegangen, dass der Leser über grundlegende Computerkenntnisse verfügt und mit der Bedienung von Softwareanwendungen vertraut ist.
Darüber hinaus werden einige technische Begriffe verwendet, die im Kontext der Softwareentwicklung und -nutzung üblich sind. 
Diese werden jedoch soweit möglich erklärt, um auch weniger erfahrenen Anwendern den Zugang zu erleichtern.
Ziel dieses Dokuments ist es, dem Anwender eine klare und verständliche Anleitung zu bieten, um das Programm erfolgreich zu installieren und zu verwenden. 
Es wird empfohlen, dieses Dokument sorgfältig zu lesen und die Anweisungen Schritt für Schritt zu befolgen, um mögliche Probleme zu vermeiden.
Das Dokument ist in mehrere Kapitel unterteilt, die verschiedene Aspekte der Softwareinstallation und -nutzung abdecken.
Jedes Kapitel enthält detaillierte Anweisungen, Screenshots und Beispiele, um den Anwender bestmöglich zu unterstützen.
Dabei wollen wir zunächst die Gruppe der Anwender in zwei Kategorien unterteilen: Es ist bereits ein bestehender Stundenplan in UNTIS vorhanden oder es wird ein neuer Stundenplan erstellt. 
Falls bereits ein Stundenplan vorhanden ist, so sollte der Anwender beim Kapitel \todo{Ref} beginnen und die vorbereitenden Schritte für den Import des bestehenden Stundenplans in das PUSS Tool durchführen.
Andernfalls kann direkt mit dem Kapitel \todo{Ref} begonnen werden, in dem die Erstellung eines neuen Stundenplans beschrieben wird.
Bei Fragen oder Problemen während der Installation oder Nutzung des Programms kann sich gerne an einen der beiden Autoren gewandt werden. 
Die Kontaktdaten sind auf der Titelseite dieses Dokuments zu finden.
\todo{Nochmal Neuschreiben, da anderer Aufbau des Dokuments als zum Zeitpunkt der Einleitungserstellung}