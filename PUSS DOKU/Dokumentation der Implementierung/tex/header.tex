 \usepackage[T1]{fontenc}
 \usepackage[utf8]{inputenc}

% Seitenränder
\textheight23.5cm
\textwidth16cm
\topmargin-1cm
\evensidemargin0cm
\oddsidemargin0cm

% Packages
\usepackage{amssymb,amsmath}
\usepackage{amsthm,mathtools}
\usepackage{fancyhdr}
\usepackage[lflt]{floatflt}
\usepackage[pdfborderstyle={/S/U/W 1},hypertexnames=false]{hyperref}
\usepackage{braket}
\usepackage[nameinlink,capitalise]{cleveref}
\usepackage{enumerate}
\usepackage{todonotes}
\usepackage{graphicx}
\usepackage{tikz}
\usepackage{quiver}
\usepackage{pgfplots}
\usepackage{pdflscape}
\usepackage{colortbl}
\usepackage[utf8]{inputenc}
\usepackage{xcolor}
\usepackage{tcolorbox}
\usepackage{float}

\tcbuselibrary{listings, skins}
\pgfplotsset{width=10cm,compat=1.6}
\usetikzlibrary{arrows}

\definecolor{darkcerulean}{rgb}{0.03, 0.27, 0.49} 	\definecolor{darkcoral}{rgb}{0.8, 0.36, 0.27}
\usepgfplotslibrary{colormaps}
\pgfplotsset{/pgfplots/colormap/bluewhitered/.style={/pgfplots/colormap={bluewhitered}{color(0cm)=(darkcerulean); color(0.5cm)=(white); color(1cm)=(darkcoral)}}}
\pgfplotsset{/pgfplots/colormap/bluered/.style={/pgfplots/colormap={bluered}{color(0cm)=(darkcerulean); color(1cm)=(darkcoral)}}}

\usepgfplotslibrary{fillbetween}

\usepackage{shadethm}
\usepackage{framed}
\definecolor{shadecolor}{rgb}{0.9,0.9,0.95}
\definecolor{shadethmcolor}{rgb}{.9,.9,.95}

\usepackage{tabularx}
\newcolumntype{x}[1]{!{\centering\arraybackslash\vrule width #1}}


% Makros & Umgebungen
\input{tex/alphabets}
%% Makros

% Mengensymbole
\newcommand{\R}{{\BR}}
\newcommand{\Z}{{\BZ}}
\newcommand{\Q}{{\BQ}}
\newcommand{\C}{{\BC}}
\newcommand{\K}{{\BK}}
\newcommand{\N}{{\BN}}
\newcommand{\T}{{\BT}}
\newcommand{\IR}{{\BR}}
\newcommand{\IZ}{{\BZ}}
\newcommand{\IQ}{{\BQ}}
\newcommand{\IC}{{\BC}}
\newcommand{\IK}{{\BK}}
\newcommand{\IN}{{\BN}}
\newcommand{\IT}{{\BT}}


%Epsilon & Phi
\newcommand{\eps}{\varepsilon}
\newcommand{\ph}{\varphi}
\newcommand{\vr}{\varrho}

% Pfeile
\newcommand{\hra}{\hookrightarrow}
\newcommand{\lra}{\leftrightarrow}
\newcommand{\la}{\leftarrow}
\newcommand{\ra}{\rightarrow}
\newcommand{\llra}{\longleftrightarrow}
\newcommand{\lla}{\longleftarrow}
\newcommand{\lr}{\longrightarrow}
\newcommand{\Lra}{\Leftrightarrow}
\newcommand{\La}{\Leftarrow}
\newcommand{\Ra}{\Rightarrow}
\newcommand{\Llra}{\Longleftrightarrow}
\newcommand{\Lla}{\Longleftarrow}
\newcommand{\Lr}{\Longrightarrow}
\newcommand{\xra}{\xrightarrow}
\newcommand{\lms}{\longmapsto}
\newcommand{\sms}{\mapsto}
% Linien
\newcommand{\ol}{\overline}
\newcommand{\ul}{\underline}

% Diverses
\newcommand{\ds}{\displaystyle}
\newcommand{\st}{\stackrel}
\newcommand{\sfrac}[2]{{\textstyle\frac{#1}{#2}}}
\newcommand{\ex}{\exists \,}

\renewcommand{\Re}{{\rm Re\,}}
\renewcommand{\Im}{{\rm Im\,}}
\newcommand{\id}{{\rm id}}
\renewcommand{\i}{{\rm i}}
\newcommand{\e}{{\rm e}}
\renewcommand{\d}{{\, \rm d}}
\newcommand{\Id}{{\rm Id}}

\newcommand{\ran}{\operatorname{ran}}

%gleichmäßige Konvergenz
\DeclareMathOperator*{\ulim}{u-lim}
\newcommand{\uto}{\rightrightarrows}

%Ober- und Unterintegral
\def\upint{\mathchoice%
	{\mkern13mu\overline{\vphantom{\intop}\mkern7mu}\mkern-20mu}%
	{\mkern7mu\overline{\vphantom{\intop}\mkern7mu}\mkern-14mu}%
	{\mkern7mu\overline{\vphantom{\intop}\mkern7mu}\mkern-14mu}%
	{\mkern7mu\overline{\vphantom{\intop}\mkern7mu}\mkern-14mu}%
	\int}
\def\lowint{\mkern3mu\underline{\vphantom{\intop}\mkern7mu}\mkern-10mu\int}

% Betrag und Norm
\newcommand{\abs}[1]{{\lvert #1 \rvert}}
%\newcommand{\sgn}[1]{{{\rm sgn}(#1)}}
\DeclareMathOperator{\sgn}{sgn}
\newcommand{\Abs}[1]{\bigl\lvert #1 \bigr\rvert}
\newcommand{\ABS}[1]{\Bigl\lvert #1 \Bigr\rvert}
\newcommand{\labs}[1]{\left| #1 \right|}
\newcommand{\norm}[1]{\lVert #1 \rVert}
\newcommand{\Norm}[1]{\Bigl\lVert #1 \Bigr\rVert}
% Skalarprodukt
\newcommand{\skp}[1]{\langle #1 \rangle}
% Grad (eines Polynoms)
\DeclareMathOperator{\grad}{grad}
% Polygonzug
\DeclareMathOperator{\pol}{Pol}

%arccot
\DeclareMathOperator{\arccot}{arccot}

%area hyperbolicus
\DeclareMathOperator{\arsinh}{arsinh}
\DeclareMathOperator{\arcosh}{arcosh}

% Fläche
\DeclareMathOperator{\area}{area}

% Rotation
\DeclareMathOperator{\rot}{rot}
\DeclareMathOperator{\divf}{div}

% Volumen
\DeclareMathOperator{\vol}{vol}

%some useful limits
\newcommand{\glmto}{\rightrightarrows}
\newcommand{\wto}{\xrightarrow{\sfw}}
\DeclareMathOperator*{\wlim}{w-lim}

%Spektren
\newcommand{\ps}{\sigma\!_\sfp}
\newcommand{\cs}{\sigma\!_\sfc}
\newcommand{\rs}{\sigma\!_\sfr}
\newcommand{\ess}{\sigma\!_\e}
\newcommand{\dis}{\sigma\!_\d}


% Vektoren (vect = 3d, vectt = 2d)
% Note: für Binome gibt's \binom
%\renewcommand{\vec}[1]{\mathbf{#1}}

\newcommand{\vect}[3]{
	\begin{pmatrix}
		#1\\#2\\#3
	\end{pmatrix}
}
\newcommand{\vectt}[2]{
	\begin{pmatrix}
		#1\\#2
	\end{pmatrix}
}
% in den Grenzen von...
\newcommand{\at}{\Big|}

\newcommand{\eq}{=}

% Setup the matha and mathx font (from mathabx.sty) to define obot
\DeclareFontFamily{U}{matha}{\hyphenchar\font45}
\DeclareFontShape{U}{matha}{m}{n}{
	<5> <6> <7> <8> <9> <10> gen * matha
	<10.95> matha10 <12> <14.4> <17.28> <20.74> <24.88> matha12
}{}
\DeclareSymbolFont{matha}{U}{matha}{m}{n}
\DeclareFontFamily{U}{mathx}{\hyphenchar\font45}
\DeclareFontShape{U}{mathx}{m}{n}{
	<5> <6> <7> <8> <9> <10>
	<10.95> <12> <14.4> <17.28> <20.74> <24.88>
	mathx10
}{}
\DeclareSymbolFont{mathx}{U}{mathx}{m}{n}

\DeclareMathSymbol{\operp}         {2}{matha}{"6B}
\DeclareMathSymbol{\bigoperp}       {1}{mathx}{"CB}

% Environments

\theoremstyle{plain}
\newshadetheorem{satzs}{Satz}[section]  %  [chapter]  durchnumeriert
\newshadetheorem{cors}[satzs]{Korollar}
\newshadetheorem{lemmas}[satzs]{Lemma}
\newshadetheorem{theos}[satzs]{Theorem}
\newshadetheorem{props}[satzs]{Proposition}

\theoremstyle{definition}
\newshadetheorem{defis}[satzs]{Definition} %    DEFINITION
\newtheorem{bsp}[satzs]{Example}  % Beispiel
\newtheorem{bsp*}[satzs]{Example*}
\newtheorem*{rem}{Remark}  % Bemerkung ohne Nummer
\newtheorem*{ueb}{Übung}%[section]


\newenvironment{satz}[1][]{
	\begin{satzs}[#1]\hspace*{1mm}
	}{\end{satzs}}

\newenvironment{cor}[1][]{
	\begin{cors}[#1]\hspace*{1mm}
	}{\end{cors}}

\newenvironment{lemma}[1][]{
	\begin{lemmas}[#1]\hspace*{1mm}
	}{\end{lemmas}}

\newenvironment{defi}[1][]{
	\begin{defis}[#1]\hspace*{1mm}
	}{\end{defis}}

\newenvironment{theo}[1][]{
	\begin{theos}[#1]\hspace*{1mm}
	}{\end{theos}}

\newenvironment{prop}[1][]{
	\begin{props}[#1]\hspace*{1mm}
	}{\end{props}}

\crefname{bsp}{Beispiel}{die Beispiele}
\Crefname{bsp}{Beispiel}{die Beispiele}
\crefname{bsp*}{Beispiel}{die Beispiele}
\Crefname{bsp*}{Beispiel}{die Beispiele}
\crefname{enumi}{}{}
\Crefname{enumi}{}{}
\creflabelformat{enumi}{#2(#1)#3}
\numberwithin{equation}{section}
\crefname{section}{Kapitel}{den Kapiteln}
\Crefname{section}{Kapitel}{den Kapiteln}
\crefname{subsection}{Abschnitt}{den Abschnitten}
\Crefname{subsection}{Abschnitt}{den Abschnitten}

\newtheoremstyle{empty}{\topsep}{\topsep}{\normalfont}{0pt}{\bfseries}{.}{5pt plus 1pt minus 1pt}{Fortsetzung #3}
\theoremstyle{empty}
\newtheorem*{fortsetzung}{NameIgnored}

\newcommand{\axiom}[1]{\bigskip
	
	\centerline{\fbox{\sc #1}}

\bigskip}

%Fußnoten
\newcommand{\fnmenv}{\hspace*{-4mm}\footnotemark\hspace*{4mm}} %Footnotemark on Name of Environment
\newcommand{\fnmproof}{\hspace*{-2mm}\footnotemark\hspace*{2mm}} %Footnotemark on Proof

%Zähler
\newcounter{fixes}
\newcounter{lecture}
\newcounter{lectureweek}

%Randnotizen & Zählerausgabe
\newlength{\fixboxwidth}
\setlength{\fixboxwidth}{\marginparwidth}
\addtolength{\fixboxwidth}{-17pt}
\newcommand{\fix}[1]{\marginpar{\parbox{\fixboxwidth}{\raggedright\tiny#1}}\stepcounter{fixes}}
%\newcommand{\comment}{\ \newline\texttt{\thefixes\ Bemerkungen, \thelecture\ Vorlesungen}, \thelectureweek\ Vorlesungswochen}

%Booleans
\newif\ifcommentnumber\commentnumberfalse %Ausgabe der Kommentarzahlen am Vorlesungsende
\newif\ifbeforelecture\beforelecturetrue %vor Vorlesungsbeginn
\newif\iflectureend\lectureendtrue %Skriptend

\newcommand{\phantomsectiontotoc}[1]{
	\stepcounter{section}
	\addtocontents{toc}{\protect\contentsline{section}{\protect\numberline{\thesection}#1}{}{}}%
}
\newcommand{\startlecture}{\beforelecturefalse\lectureendfalse}
\newcommand{\stopnotes}{\lectureendtrue\fi\ifcommentnumber\comment\fi\iffalse}
\newcommand{\stopnotesext}{\lectureendtrue\ifcommentnumber\comment\fi}
\newcommand{\electure}{}
\newcommand{\flecture}{}
\newcommand{\electureweek}{}
\newcommand{\flectureweek}{}
\newcommand{\startcounting}{
	\renewcommand{\electure}{\stepcounter{lecture}\fix{Ende \thelecture. Vorlesung}}
	\renewcommand{\flecture}{\stepcounter{lecture}\fix{Ende \thelecture. Vorlesung}\lectureendtrue\fi\ifcommentnumber\comment\fi\iffalse }
	\renewcommand{\electureweek}{\stepcounter{lectureweek}\fix{Ende \thelectureweek. Vorlesungswoche}}
	\renewcommand{\flectureweek}{\stepcounter{lectureweek}\fix{Ende \thelectureweek. Vorlesungswoche}\lectureendtrue\fi\ifcommentnumber\comment\fi\iffalse}
}
\newcommand{\stopcounting}{
	\renewcommand{\electure}{}
	\renewcommand{\flecture}{}
	\renewcommand{\electureweek}{}
	\renewcommand{\flectureweek}{}
}

%Referenzen im Dokument überarbeiten
\def\extref{ext:}
\makeatletter
\newcommand\lref[1]{\@ifundefined{r@#1}{\cref*{\extref#1}}{\cref{#1}}}
\newcommand\llref[1]{\@ifundefined{r@#1}{\labelcref*{\extref#1}}{\labelcref{#1}}}
\newcommand\nlref[1]{\@ifundefined{r@#1}{\namecref{\extref#1}}{\namecref{#1}}}
\makeatother

%%%%%%%% diagrams %%%%%%%%
\usepackage[all]{xy}%provides xymatrix
\newcommand{\vcxymatrix}[1]{\vcenter{\xymatrix{#1}}}%provides a xymatrix which vertically centers eqnumbers

%Category
\newcommand{\op}{\text{op}}
\newcommand{\Mod}{\text{Mod}}
\newcommand{\Top}{\text{Top}}
\newcommand{\Grp}{\text{Grp}}
\newcommand{\Ring}{\text{Ring}}
\newcommand{\RMod}{\text{R-Mod}}
\newcommand{\Ab}{\text{Ab}}
\newcommand{\Vect}{\text{Vect}}
\newcommand{\Cat}{\text{Cat}}
\newcommand{\Fun}{\text{Fun}}
\newcommand{\Hom}{\text{Hom}}
\newcommand{\End}{\text{End}}
\newcommand{\Aut}{\text{Aut}}
\newcommand{\Iso}{\text{Iso}}
\newcommand{\Ob}{\text{Ob}}
\newcommand{\Mor}{\text{Mor}}
\newcommand{\Ext}{\text{Ext}}
\newcommand{\Ker}{\text{Ker}}
\newcommand{\Cok}{\text{Cok}}

% Definition der Farben für das Code-Highlighting
\definecolor{codegreen}{rgb}{0,0.6,0}
\definecolor{codegray}{rgb}{0.5,0.5,0.5}
\definecolor{codepurple}{rgb}{0.58,0,0.82}
\definecolor{backcolour}{rgb}{0.95,0.95,0.92}
\definecolor{boxheader}{rgb}{0.85, 0.85, 0.85}

% Definition des SQL-Styles
\lstdefinestyle{mystyle}{
    backgroundcolor=\color{backcolour},   
    commentstyle=\color{codegreen},
    keywordstyle=\color{magenta}\bfseries,
    numberstyle=\tiny\color{codegray},
    stringstyle=\color{codepurple},
    basicstyle=\ttfamily\footnotesize,
    breakatwhitespace=false,         
    breaklines=true,                 
    captionpos=b,                    
    keepspaces=true,                 
    numbers=left,                    
    numbersep=5pt,                  
    showspaces=false,                
    showstringspaces=false,
    showtabs=false,                  
    tabsize=2,
    language=SQL,
    deletekeywords={Time, Hour},
    % Fix für deutsche Umlaute im Code:
    literate={ä}{{\"a}}1 {ö}{{\"o}}1 {ü}{{\"u}}1 {Ä}{{\"A}}1 {Ö}{{\"O}}1 {Ü}{{\"U}}1 {ß}{{\ss}}1
}

% Definition der schönen Code-Box
\newtcblisting{sqlbox}[2][]{
    enhanced,
    skin=enhancedlast,
    attach boxed title to top left={xshift=-4mm,yshift=-2mm},
    fonttitle=\bfseries\sffamily,
    colback=backcolour,
    colframe=black!50,
    colbacktitle=boxheader,
    coltitle=black,
    boxed title style={
        empty,
        arc=0pt,
        outer arc=0pt,
        boxrule=0pt
    },
    underlay boxed title={
        \fill[white] (title.north west) -- (title.north east)
        -- +(\tcboxedtitleheight-1mm,-\tcboxedtitleheight+1mm)
        -- ([xshift=4mm,yshift=0.5mm]frame.north east) -- +(0mm,-1mm)
        -- (title.south west) -- cycle;
        \draw[black!50, line width=0.5mm] ([xshift=4mm,yshift=-0.5mm]frame.north east)
        -- +(-1mm,1mm);
    },
    title={#2},
    listing only,
    listing options={style=mystyle},
    #1
}


%======================================
% Kopf-/Fußzeilen und Absatzformat
%======================================
\setlength{\parindent}{0pt}
%TODO: das hier verzerrt auch das Inhaltsverzeichnis -- gibt's ne Abhilfe?
\setlength{\parskip}{1.4\smallskipamount}

\fancypagestyle{plain}{
    \renewcommand{\headrulewidth}{0.4pt}
    \renewcommand
{\headheight}{14pt}
    \fancyhf{}
    \fancyhead[R]{\romannumeral\thepage}
}

\newcommand{\startnotes}{
	\clearpage
	\setcounter{page}{1}
	\fancyhead[R]{\thepage}
	\fancyhead[L]{\rightmark}}

%======================================
% Anpassen der Breite für die Abschnittsnummerierung im TOC
% 3. Argument von \@dottedtocline setzt die Breite der Box für
% die Abschnittsnummerierung
% (schönen Dank an Nils!)
%======================================
\makeatletter
\renewcommand{\l@subsection}{\@dottedtocline{2}{1.5em}{3.3em}}
\renewcommand{\l@subsubsection}{\@dottedtocline{2}{1.5em}{4.5em}}
\makeatother


\pagestyle{plain}