% \section{Einleitung}
% Dieses Dokument dient als Dokumentation der Implementierung des PUSS Tools. Es richtet sich an Entwickler, die den Quellcode verstehen und erweitern möchten.
% Es wird vorausgesetzt, dass der Leser mit den Grundlagen von SQL und der Programmiersprache Python und Java vertraut ist.
% Sollte der Leser noch keine Erfahrung mit dem PUSS Tool haben, so wird empfohlen, zunächst das Benutzerhandbuch zu lesen, welches die Benutzung des Tools erläutert.
% Der Quellcode des PUSS Tools ist auf dem Gitlab-Repository der Universität Paderborn unter \todo{Link} zu finden und steht unter der \todo{Lizenz?} zur Verfügung.
% Das Dokument ist wie folgt gegliedert:
% \begin{itemize}
%     \item Test
% \end{itemize}
% \todo{Liste der Kapitel und deren Inhalte ergänzen}
% Wir hoffen, dass diese Dokumentation hilfreich ist und wünschen viel Erfolg bei der Arbeit mit dem PUSS Tool!
% Falls Sie Fragen oder Anregungen haben, zögern Sie nicht, uns zu kontaktieren unter den auf der Titelseite angegebenen E-Mail-Adressen.
\section{Einleitung}
\label{sec:einleitung}

Die konfliktfreie Erstellung von Stundenplänen setzt eine valide und strukturierte Datengrundlage voraus. 
Das in diesem Dokument beschriebene PUSS-Tool (\textit{Pla­nungs-Un­ter­stüt­zung für Schul-Stun­den­plä­ne}) basiert auf der Extraktion von Daten aus dem weit verbreiteten Stundenplanungsprogramm Untis, welches in vielen Schulen als Standardlösung für die Stundenplanung eingesetzt wird.
Da Untis als historisch gewachsenes System über ein sehr spezifisches Datenmodell verfügt, welches stark auf die Belange des klassischen Schulbetriebs zugeschnitten ist, können die Daten nicht 1:1 in das auf universitäre Planungslogiken optimierte STULP-Schema überführt werden.
Es ist vielmehr ein komplexer Transformationsprozess (ETL) notwendig, um die semantische Integrität der Daten zu wahren und implizite Informationen – wie etwa Kopplungen von Klassen oder die Logik von Zeitrastern – explizit zu machen.

\subsection{Zielsetzung des Dokuments}
Dieses Dokument dient als technische Spezifikation. Es richtet sich an Entwickler und Systemarchitekten und verfolgt drei wesentliche Ziele:

\begin{enumerate}
    \item \textbf{Dokumentation der Quell-Daten:} Eine detaillierte Analyse der relevanten DIF-Exportdateien aus Untis, einschließlich der Datenstrukturen, Schlüsselattribute und der semantischen Bedeutung der enthaltenen Informationen (siehe Kapitel \ref{sec:dif}).
    \item \textbf{Spezifikation des Mappings:} Die Darlegung der Transformationslogik (siehe Kapitel \ref{sec:mapping}) und der notwendigen Anpassungen, um die Daten in das STULP-Schema zu überführen. Dies umfasst die Identifikation von Entitäten, die Normalisierung von Daten und die Handhabung von Sonderfällen (z.B. unvollständige oder inkonsistente Daten).
    \item \textbf{Beschreibung der Implementierung vom lokalen Plannungstool der STULP Projektgruppe:} Wir wollen eine Zusammenfassung der Implementierung des lokalen Planungstools der STULP Projektgruppe geben, um die praktische Umsetzung der beschriebenen Mappings und Datenstrukturen zu veranschaulichen (siehe Kapitel \ref{sec:implementation}).
    \item \textbf{Beschreibung der Erweiterungsmöglichkeiten:} Eine Diskussion möglicher funktionaler Erweiterungen (siehe Kapitel \ref{sec:Erweiterungen}), um spezifische Schulszenarien wie Wahlpflichtbereiche oder komplexe Zeitwünsche im PUSS-Tool abzubilden.
\end{enumerate}

Insgesamt soll dieses Dokument als umfassende Referenz für die Entwicklung und Wartung des PUSS-Tools dienen, um eine effiziente und konfliktfreie Stundenplanung zu ermöglichen. 
Es wird empfohlen zunächst das Benutzerhandbuch zu lesen, um die grundlegende Funktionsweise des PUSS-Tools zu verstehen. 
Anschließend wird empfohlen, die einzelnen Kapitel in der angegebenen Reihenfolge zu lesen, um ein vollständiges Verständnis der Datenstrukturen, Transformationsprozesse und Implementierungsdetails zu erhalten.