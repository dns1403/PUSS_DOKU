\section{Einleitung}
Dieses Dokument dient als Dokumentation der Implementierung des PUSS Tools. Es richtet sich an Entwickler, die den Quellcode verstehen und erweitern möchten.
Es wird vorausgesetzt, dass der Leser mit den Grundlagen von SQL und der Programmiersprache Python vertraut ist.
Sollte der Leser noch keine Erfahrung mit dem PUSS Tool haben, so wird empfohlen, zunächst das Benutzerhandbuch zu lesen, welches die Benutzung des Tools erläutert.
Der Quellcode des PUSS Tools ist auf dem Gitlab-Repository der Universität Paderborn unter \todo{Link} zu finden und steht unter der \todo{Lizenz?} zur Verfügung.
Das Dokument ist wie folgt gegliedert:
\begin{itemize}
    \item Test
\end{itemize}
\todo{Liste der Kapitel und deren Inhalte ergänzen}
Wir hoffen, dass diese Dokumentation hilfreich ist und wünschen viel Erfolg bei der Arbeit mit dem PUSS Tool!
Falls Sie Fragen oder Anregungen haben, zögern Sie nicht, uns zu kontaktieren unter den auf der Titelseite angegebenen E-Mail-Adressen.