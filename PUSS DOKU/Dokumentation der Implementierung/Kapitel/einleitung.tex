\section{Einleitung}
\label{sec:einleitung}

Die konfliktfreie Erstellung von Stundenplänen setzt eine valide und strukturierte Datengrundlage voraus. 
Das in diesem Dokument beschriebene Mapping basiert auf der Extraktion von Daten aus dem weit verbreiteten Stundenplanungsprogramm Untis, welches in vielen Schulen als Standardlösung für die Stundenplanung eingesetzt wird.
Da Untis als historisch gewachsenes System über ein sehr spezifisches Datenmodell verfügt, welches stark auf die Belange des klassischen Schulbetriebs zugeschnitten ist, können die Daten nicht 1:1 in das auf universitäre Planungslogiken optimierte STULP-Schema überführt werden.
Es ist vielmehr ein komplexer Transformationsprozess (ETL) notwendig, um die semantische Integrität der Daten zu wahren und implizite Informationen - wie etwa Kopplungen von Klassen oder die Logik von Zeitrastern - explizit zu machen.

\subsection{Zielsetzung des Dokuments}
Dieses Dokument dient als technische Spezifikation. Es richtet sich an Entwickler und Systemarchitekten. 
Wir haben hierbei das von der Projektgruppe entwickelte Mapping-Skript zu dokumentieren und die dahinterliegende Logik zu erläutern.
Dabei wollen wir insbesondere nicht auf die technischen Details der Implementierung des lokalen Planungstools der STULP Projektgruppe eingehen. 
Diese haben bereits eine ausführliche Dokumentation zu ihrer Implementierung erstellt, auf die wir hier verweisen wollen. \todo{Referenz angeben, wo diese zu finden ist}
Insbesondere sollte sich der Leser in dieser Dokumentation mit der Datenbankstruktur des lokalen Planungstools vertraut machen, da wir diese hier nicht erneut dokumentiert haben.
Bei der Dokumentation des Mappings konzentrieren wir uns auf die folgenden Aspekte:
\begin{enumerate}
    \item \textbf{Dokumentation der Quell-Daten:} Eine detaillierte Analyse der relevanten DIF-Exportdateien aus Untis, einschließlich der Datenstrukturen, Schlüsselattribute und der semantischen Bedeutung der enthaltenen Informationen (siehe Kapitel \ref{sec:dif}).
    \item \textbf{Spezifikation des Mappings:} Die Darlegung der Transformationslogik (siehe Kapitel \ref{sec:mapping}) und der notwendigen Anpassungen, um die Daten in das STULP-Schema zu überführen. Dies umfasst die Identifikation von Entitäten, die Normalisierung von Daten und die Handhabung von Sonderfällen (z.B. unvollständige oder inkonsistente Daten).
    %\item \textbf{Beschreibung der Implementierung vom lokalen Plannungstool der STULP Projektgruppe:} Wir wollen eine Zusammenfassung der Implementierung des lokalen Planungstools der STULP Projektgruppe geben, um die praktische Umsetzung der beschriebenen Mappings und Datenstrukturen zu veranschaulichen (siehe Kapitel \ref{sec:implementation}).
    \item \textbf{Beschreibung der Erweiterungsmöglichkeiten:} Eine Diskussion möglicher funktionaler Erweiterungen (siehe Kapitel \ref{sec:Erweiterungen}), um spezifische Schulszenarien wie Wahlpflichtbereiche oder komplexe Zeitwünsche im PUSS-Tool abzubilden.
\end{enumerate}

Insgesamt soll dieses Dokument als umfassende Referenz für die (Weiter)entwicklung und Wartung des Mappings (bzw. des lokalen Planungstools) dienen, um eine effiziente und konfliktfreie Stundenplanung zu ermöglichen. 
Es wird empfohlen zunächst das Benutzerhandbuch zu lesen, um die grundlegende Funktionsweise und Möglichkeiten des lokalen Planungstools im Schulkontext zu verstehen.
Anschließend wird empfohlen, die einzelnen Kapitel in der angegebenen Reihenfolge zu lesen, um ein vollständiges Verständnis der Datenstrukturen, Transformationsprozesse und Implementierungsdetails zu erhalten.