\section{Dokumentation der Dif Dateien}
In diesem Kapitel werden die relevanten Dif Dateien beschrieben, die im PUSS Tool verwendet werden.
Diese basiert auf einer Dokumentation von Untis\footnote{siehe \url{https://www.untis.at/manual/index.html?ti_allgemeine-schnittstellen.htm}} welche ergänzt und an die Bedürfnisse des PUSS Tools angepasst wurde.
Außerdem kann in dieser Dokumentation eine vollständige Liste aller Dif Dateien gefunden werden, insbesondere auch jene, die wir im folgenden nicht beschreiben.

\subsection{Notationshinweise und Legende}
Um die Struktur der Datenbanktabellen und deren Relation zu den DIF-Dateien zu verdeutlichen, wird in den folgenden Abschnitten eine einheitliche Notation verwendet. Diese beschreibt sowohl technische Constraints als auch den Grad der Normalisierung.

\paragraph{Bedeutung der Abkürzungen (Constraints)}
\begin{itemize}
    \item \textbf{PK (Primary Key):} Primärschlüssel der Tabelle.
    \item \textbf{FK (Foreign Key):} Fremdschlüssel, der auf eine andere Relation verweist.
    \item \textbf{NN (Not Null):} Dieses Feld darf keinen leeren Wert enthalten.
    \item \textbf{U (Unique):} Der Wert in diesem Feld muss innerhalb der Tabelle eindeutig sein.
\end{itemize}

\paragraph{Farblegende zur Normalisierung und Schlüsselstruktur}
In Anlehnung an die grafische Analyse der Datenbankstruktur werden folgende farbliche Markierungen in den Tabellenschemata verwendet:

\begin{table}[h]
    \centering
    \begin{tabular}{|l|l|l|}
        \hline
        \textbf{Farbe} & \textbf{Bezeichnung} & \textbf{Beschreibung} \\ \hline
        \cellcolor[HTML]{A4C2F4} Blau & Primärschlüssel (PK) & Kennzeichnet Felder, die den Datensatz eindeutig identifizieren. \\ \hline
        \cellcolor[HTML]{F9CB9C} Orange & Verletzung 2. NF & Felder, die funktional von einem Teil des Schlüssels abhängen. \\ \hline
        \cellcolor[HTML]{B7E1CD} Grün & Fremdschlüssel (FK) & Verweis auf Stammdaten einer anderen DIF-Datei (z.B. Lehrer-ID). \\ \hline
    \end{tabular}
    \caption{Legende der verwendeten Notationsfarben}
    \label{tab:legende_notation}
\end{table}

\noindent
Ein Pfeil ($\uparrow$) vor einem Spaltennamen signalisiert zusätzlich einen Fremdschlüsselbezug in der Implementierung.
\subsection{GPU001 -- Stundenplan}
Die Datei \texttt{GPU001.TXT} enthält den Stundenplan. Da ein Unterrichtselement (Unterrichtsnummer) für mehrere Klassen gleichzeitig stattfinden kann (z.B. Koppelungen), bildet die Kombination aus Nummer und Klasse den Identifikator.

\paragraph{Schema und Beispieldaten}

    \begin{table}[h]
        \centering
        \tiny
        \begin{tabular}{|c|c|c|c|c|c|c|c|c|}
            \hline
                               & \textbf{$\uparrow$ Unterrichtsnummer} & \textbf{$\uparrow$ Klasse} & \textbf{$\uparrow$ Lehrer} & \textbf{$\uparrow$ Fach} & \textbf{$\uparrow$ Raum} & \textbf{Tag} & \textbf{Stunde} & \textbf{Stundenlänge} \\ \hline\hline
            \textbf{Data Type} & INTEGER & TEXT & TEXT & TEXT & TEXT & INTEGER & INTEGER & INTEGER \\
            \textbf{Constraints}& NN, PK & NN, PK & NN & NN & & NN & NN & \\
            \textbf{Checks}    & & & & & & $\in \{1..7\}$ & & in Minuten \\ \hline
            \textbf{Notation}  & \cellcolor[HTML]{A4C2F4} PK & \cellcolor[HTML]{A4C2F4} PK & \cellcolor[HTML]{F9CB9C} 2. NF & \cellcolor[HTML]{F9CB9C} 2. NF & \cellcolor[HTML]{F9CB9C} 2. NF & \cellcolor[HTML]{A4C2F4} PK & \cellcolor[HTML]{A4C2F4} PK & \\ \hline
            \textbf{Sample}    & 1 & 07A & SMI & D & B2.16 & 2 & 7 &  \\
                               & 1 & 08A & AML & M & B2.12 & 1 & 4 &  \\ \hline
        \end{tabular}
        \caption{Schema der Relation Stundenplan (GPU001) mit Normalform-Hinweisen und Beispieldaten}
        \label{schema:gpu001_new}
    \end{table}

