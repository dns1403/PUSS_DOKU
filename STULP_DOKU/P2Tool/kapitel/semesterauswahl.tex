\section{Semesterauswahl (als Admin / Planer)}
\label{sec:semesterauswahl}
Auf allen Seiten des Tools gibt es die Möglichkeit auszuwählen, in welchem Semester Änderungen vorgenommen werden sollen.
Oben links wird angezeigt, welches Semester aktuell ausgewählt ist.
In \figref{fig:semesterauswahl} ist aktuell zum Beispiel das Sommersemester 2025 ausgewählt, und alle Änderungen, die der Benutzer macht, werden in diesem Semester vollzogen.

\begin{figure}[h!]
    \centering
    \includegraphics[width=0.75\linewidth]{img/Semester_auswahl.png}
    \caption{Semesterverwaltung}
    \label{fig:semesterauswahl}
\end{figure}

Lehrende befinden sich immer im neuesten Semester, dass auf dem Server existiert.
Administratoren und Planer können in ein vergangenes Semester wechseln, um noch Änderungen nachzutragen, während nebenbei schon ein neueres Semester geplant wird.
\begin{warning}
    Beim Erstellen eines neuen Semesters wird das aktuell ausgewählte Semester von allen Benutzern auf das neu erstellte Semester geändert.
\end{warning}

Das derzeit ausgewählte Semester kann geändert werden, indem der Benutzer auf das aktuell ausgewählte Semester klickt. Dann öffnet sich eine Liste mit allen aktuellen Semestern, und mit einem Klick auf das gewünschte Semester wird dieses ausgewählt (siehe \figref{fig:semesterauswahldialog}).
\begin{figure}[h!]
    \centering
    \includegraphics[width=0.75\linewidth]{img/Semester_auswahl_dialog.png}
    \caption{Semesterverwaltung}
    \label{fig:semesterauswahldialog}
\end{figure}
