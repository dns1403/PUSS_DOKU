\subsection{Angebotene Veranstaltungen für andere Benutzer erstellen (als Admin/Planer)}
Um als Planer oder Administrator für andere Benutzer Lehrveranstaltungen zu konfigurieren, muss auf die Seite \textit{Aktueller Vorlesungsplan} gewechselt werden und dort auf \textit{Hinzufügen} geklickt werden (siehe \figref{fig:veranstaltung_admin1}).
Danach erscheint ein ähnlicher Dialog wie wenn man Veranstaltungen für sich selbst erstellt.
Der einzige Unterschied ist, dass am Anfang ausgewählt werden muss, welcher Veranstalter die Veranstaltung halten soll (siehe \figref{fig:veranstaltung_admin2}).
Danach läuft alles wie in Kapitel \ref{sec:meine_veranstaltungen_add} beschrieben.
\begin{figure}[h!]
    \centering
    \includegraphics[width=0.75\linewidth]{img//MeineVeranstaltungen//Admin//MeinVeranstaltungsplan_Add_Admin1.png}
    \caption{Veranstaltungen für andere Benutzer anlegen}
    \label{fig:veranstaltung_admin1}
\end{figure}
\begin{figure}[h!]
    \centering
    \includegraphics[width=0.75\linewidth]{img//MeineVeranstaltungen//Admin//MeinVeranstaltungsplan_Add_Admin2.png}
    \caption{Veranstaltungen für andere Benutzer anlegen}
    \label{fig:veranstaltung_admin2}
\end{figure}

\subsection{Angebotene Veranstaltungen für andere Benutzer ändern oder löschen oder den Veranstalter einer Veranstaltung ändern (als Admin/Planer)}
Um als Planer oder Administrator für andere Benutzer Lehrveranstaltungen zu konfigurieren, muss auf die Seite \textit{Aktueller Vorlesungsplan} gewechselt werden.
Falls es gewünscht ist, eine angebotene Veranstaltung zu löschen, kann man dies durch Klicken des Knopfes \textit{Löschen} in der Zeile, in der die zu löschende angebotene Veranstaltung steht (siehe \figref{fig:veranstaltung_admin4}).
\begin{figure}[h!]
    \centering
    \includegraphics[width=0.75\linewidth]{img//MeineVeranstaltungen//Admin//MeinVeranstaltungsplan_Add_Admin4.png}
    \caption{Veranstaltungen für andere Benutzer löschen}
    \label{fig:veranstaltung_admin4}
\end{figure}

Um eine angebotene Veranstaltung zu bearbeiten oder den Veranstalter zu ändern, muss auf den Titel der zu ändernden Veranstaltung geklickt werden.
Danach erscheint ein Menu, in dem der Veranstalter geändert werden kann (siehe \figref{fig:veranstaltung_admin3}).
Änderungen an den Wunschzeiten, Hörerzahlen, usw. können wie in \ref{sec:meine_veranstaltungen_add} beschrieben wurde vorgenommen werden.


\begin{figure}[h!]
    \centering
    \includegraphics[width=0.75\linewidth]{img//MeineVeranstaltungen//Admin//MeinVeranstaltungsplan_Add_Admin3.png}
    \caption{Veranstaltungen für andere Benutzer bearbeiten}
    \label{fig:veranstaltung_admin3}
\end{figure}
