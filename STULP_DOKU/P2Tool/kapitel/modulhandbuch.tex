\section{Meine Modulhandbücher}
\label{sec:modulhandbuch}

Um Modulhandbücher und Druckrubriken im Web Tool anzulegen und zu ändern, muss auf die Seite \textit{Meine Modulhandbücher} gewechselt werden (siehe \figref{fig:m0}).
Alle Änderungen an Modulhandbüchern werden im aktuell ausgewählten Semester sowie in allen neueren Semestern umgesetzt.
\begin{figure}[h!]
\centering
\includegraphics[width=0.75\linewidth]{img//Modulhandbuch//M0.png}
\caption{Modulhandbuch}
\label{fig:m0}
\end{figure}

\subsection{Druckrubriken verwalten}
Um Druckrubriken zu verwalten, muss zuerst der Knopf \textit{Druckrubriken verwalten} gedrückt werden (siehe \figref{fig:m1}).

\begin{figure}[h!]
\centering
\includegraphics[width=0.75\linewidth]{img//Modulhandbuch//M1.png}
\caption{Modulhandbuch}
\label{fig:m1}
\end{figure}

\subsubsection{Druckrubrik hinzufügen}
\begin{warning}
    Das Erstellen von mehreren Druckrubriken mit der gleichen Bezeichnung führt zu unerwünschtem Verhalten. Ein versuchtes Erstellen von doppelten Druckrubriken führt zu einer Fehlermeldung.
\end{warning}
Um jetzt eine Druckrubrik hinzuzufügen, muss der Knopf \textit{Druckrubrik hinzufügen} gedrückt werden (siehe \figref{fig:m2}).

\begin{figure}[h!]
\centering
\includegraphics[width=0.75\linewidth]{img//Modulhandbuch//M2.png}
\caption{Modulhandbuch}
\label{fig:m2}
\end{figure}

Zuerst muss die Bezeichnung und die Oberkategorie der neuen Druckrubrik festgelegt werden (siehe \figref{fig:m3}).
\begin{figure}[h!]
\centering
\includegraphics[width=0.75\linewidth]{img//Modulhandbuch//M3.png}
\caption{Modulhandbuch}
\label{fig:m3}
\end{figure}

Die Auswahlmöglichkeiten sind die Bezeichnungen alle schon existierenden Druckrubriken und Unterdruckrubriken sowie \textit{Keine} (siehe \figref{fig:m4}).
\begin{figure}[h!]
\centering
\includegraphics[width=0.75\linewidth]{img//Modulhandbuch//M4.png}
\caption{Modulhandbuch}
\label{fig:m4}
\end{figure}

Falls die Druckrubrik keine Oberkategorie haben soll, muss \textit{Keine} ausgewählt werden.
Falls die neue Druckrubrik die Unterkategorie einer schon existierenden Druckrubrik werden soll, muss diese hier ausgewählt werden.
Das Feld Reihenfolge wird aktuell nicht benutzt.
Es kann demnach einfach so gelassen werden, wie es ist.
Anschließend müssen die Eingaben mit einem Klick auf den blauen Knopf \textit{Speichern} gespeichert werden (siehe \figref{fig:m5}).
\begin{figure}[h!]
\centering
\includegraphics[width=0.75\linewidth]{img//Modulhandbuch//M5.png}
\caption{Modulhandbuch}
\label{fig:m5}
\end{figure}

\subsubsection{Unterdruckrubrik direkt erstellen}
Um direkt danach eine Unterdruckrubrik der eben erstellen Druckrubrik zu erstellen, kann auf den Knopf \textit{Druckrubrik hinzufügen} geklickt werden (siehe \figref{fig:m6}).
\begin{figure}[h!]
\centering
\includegraphics[width=0.75\linewidth]{img//Modulhandbuch//M6.png}
\caption{Modulhandbuch}
\label{fig:m6}
\end{figure}

Dann öffnet sich der gleiche Dialog wie beim Erstellen einer neuen Druckrubrik mit der Ausnahme, dass die Oberkategorie schon auf die aktuell ausgewählte Druckrubrik festgelegt ist.
Auch hier muss mit einem Klick auf \textit{Speichern} gepeichert werden (siehe \figref{fig:m7}).
\begin{figure}[h!]
\centering
\includegraphics[width=0.75\linewidth]{img//Modulhandbuch//M7.png}
\caption{Modulhandbuch}
\label{fig:m7}
\end{figure}

\subsubsection{Navigieren innerhalb von Druckrubriken - Löschen und Bearbeiten}
Um von einer Unterdruckrubrik direkt zu der darüberliegenden Elterndruckrubrik zu gelangen, kann auf den blau unterstrichenen Namen der Oberkategorie gedrückt werden (siehe \figref{fig:m8}).
\begin{figure}[h!]
\centering
\includegraphics[width=0.75\linewidth]{img//Modulhandbuch//M8.png}
\caption{Modulhandbuch}
\label{fig:m8}
\end{figure}

Alle Unterdruckrubriken der aktuell ausgewählten Druckrubrik werden unten in der Druckrubrikansicht angezeigt.
Um diese zu bearbeiten, muss auf sie geklickt werden und um sie zu löschen, muss auf den Knopf \textit{Löschen} neben der zu löschenden Unterdruckrubrik geklickt werden (siehe \figref{fig:m9}).
\begin{figure}[h!]
\centering
\includegraphics[width=0.75\linewidth]{img//Modulhandbuch//M9.png}
\caption{Modulhandbuch}
\label{fig:m9}
\end{figure}

\subsection{Modulhandbuch}

\subsubsection{Modulhandbuch hinzufügen}
Um ein neues Modulhandbuch hinzuzufügen, muss auf den Knopf \textit{Hinzufügen} auf der Seite \textit{Meine Modulhandbücher} geklickt werden (siehe \figref{fig:m10}).
\begin{figure}[h!]
\centering
\includegraphics[width=0.75\linewidth]{img//Modulhandbuch//M10.png}
\caption{Modulhandbuch}
\label{fig:m10}
\end{figure}

Anschließend öffnet sich ein Dialog, in den der Name des Hauptfachs, der Name des Nebenfachs, das Fachsemester, die Prüfungsordnungsversion und ein Kürzel eingetragen werden muss.
Dieses dient der Identifikation der modellierten Modulhandbücher sowie der dazugehörigen Prüfungsordnungen.
Die Eingaben müssen durch einen Klick auf den blauen Knopf \textit{Speichern} gespeichert werden (siehe \figref{fig:m11}).
\begin{figure}[h!]
\centering
\includegraphics[width=0.75\linewidth]{img//Modulhandbuch//M11.png}
\caption{Modulhandbuch}
\label{fig:m11}
\end{figure}

\subsubsection{Module zu Modulhandbuch hinzufügen}
\label{sec:mod_create}
Um Module zu einem Modulhandbuch hinzuzufügen, muss auf den Knopf \textit{Modul hinzufügen} geklickt werden (siehe \figref{fig:m12}).
\begin{figure}[h!]
\centering
\includegraphics[width=0.75\linewidth]{img//Modulhandbuch//M12.png}
\caption{Modulhandbuch}
\label{fig:m12}
\end{figure}

Dort muss zunächst der Fachbereich des zu erstellenden Moduls ausgewählt werden und dann der Kurs ausgewählt werden, indem auf das Dropdown \textit{Kurs auswählen} gedrückt wird (siehe \figref{fig:m13}).
\begin{figure}[h!]
\centering
\includegraphics[width=0.75\linewidth]{img//Modulhandbuch//M13.png}
\caption{Modulhandbuch}
\label{fig:m13}
\end{figure}

Dort erscheint eine Liste mit allen Kursen in dem ausgewählten Fachbereich (siehe \figref{fig:m14}).
\begin{figure}[h!]
\centering
\includegraphics[width=0.75\linewidth]{img//Modulhandbuch//M14.png}
\caption{Modulhandbuch}
\label{fig:m14}
\end{figure}

Danach muss der Modulname eingegeben werden, soweit er vom Kursnamen abweicht und die ECTS festgelegt werden.
Bei \textit{Pflichtfach} soll ein Haken gesetzt werden, falls das Modul ein Pflichtmodul ist, und kein Haken gesetzt werden, wenn das Modul nicht verpflichtend ist.
Anschließend muss auf den blauen Knopf \textit{Nächster Schritt} gedrückt werden, um das Modul weiter zu konfigurieren (siehe \figref{fig:m15}).
\begin{figure}[h!]
\centering
\includegraphics[width=0.75\linewidth]{img//Modulhandbuch//M15.png}
\caption{Modulhandbuch}
\label{fig:m15}
\end{figure}

Hier kann eingestellt werden, welche Veranstaltungsteile des Kurses für dieses Modul relevant sind.
Falls in diesem Modul zum Beispiel nur die Vorlesung gehört werden soll, die Übung jedoch nicht, muss der Veranstaltungsteil \textit{Übung} im rechten Feld \textit{Aktive Veranstaltungsteile} ausgewählt werden und dann mit einem Klick auf den Pfeil nach links zu den ausgeschlossenen Veranstaltungsteilen bewegt werden (siehe \figref{fig:m16}).
Falls Veranstaltungsteile wieder aktiviert werden sollen, können diese markiert werden und analog mit dem Pfeil nach rechts wieder zu den aktiven Veranstaltungsteilen geschoben werden.

\begin{figure}[h!]
\centering
\includegraphics[width=0.75\linewidth]{img//Modulhandbuch//M16.png}
\caption{Modulhandbuch}
\label{fig:m16}
\end{figure}

Mit dem gleichen Prozess können Druckrubriken für ein Modul festgelgt werden.
Die Eingaben werden danach mit einem Klick auf den blauen Knopf \textit{Speichern} gespeichert.
\begin{figure}[h!]
\centering
\includegraphics[width=0.75\linewidth]{img//Modulhandbuch//M17.png}
\caption{Modulhandbuch}
\label{fig:m17}
\end{figure}

\subsubsection{Module bearbeiten und löschen}
Um ein Modul in einem Modulhanbuch zu bearbeiten, muss auf den Namen des entsprechenden Moduls geklickt werden. Wenn es gelöscht werden soll, muss stattdessen auf den danabenliegenden Knopf \textit{Löschen} gedrückt werden (siehe \figref{fig:m18}).
\begin{figure}[h!]
\centering
\includegraphics[width=0.75\linewidth]{img//Modulhandbuch//M18.png}
\caption{Modulhandbuch}
\label{fig:m18}
\end{figure}

\subsubsection{Modulhandbücher bearbeiten, löschen, kopieren, aktivieren und deaktivieren}
Um ein Modulhandbuch zu bearbeiten, muss in die Zeile des Modulhandbuchs geklickt werden (siehe \figref{fig:m19}).
Dann kann es wie beim Erstellen bearbeitet werden (wie in Kapitel \ref{sec:mod_create} beschrieben wurde).
Um ein Modulhandbuch zu kopieren, muss der Knopf \textit{Kopieren} gedrückt werden.
Damit wird eine exakte Kopie erstellt, was nützlich ist, wenn eine neue Prüfungsordnungsversion nur kleine Änderungen zur alten Prüfungsordnungsversion enthält und nicht alle Module neu modelliert werden müssen.
Um ein Modulhandbuch zu löschen, muss auf den Knopf \textit{Löschen} neben dem zu löschenden Modulhandbuch gedrückt werden.
Um ein Modulhandbuch zu aktivieren, muss der Knopf \textit{Aktivieren} gedrückt werden.
Danach erscheint das Modulhandbuch in der Kategorie \textit{Aktive Studiengänge}.

\begin{figure}[h!]
\centering
\includegraphics[width=0.75\linewidth]{img//Modulhandbuch//M19.png}
\caption{Modulhandbuch}
\label{fig:m19}
\end{figure}

Um ein Modulhandbuch zu deaktivieren, muss auf den Knopf \textit{Deaktivieren} in der Zeile des zu deaktivierenden Modulhandbuchs gedrückt werden (siehe \figref{fig:m21}).
\begin{figure}[h!]
\centering
\includegraphics[width=0.75\linewidth]{img//Modulhandbuch//M21.png}
\caption{Modulhandbuch}
\label{fig:m21}
\end{figure}

\subsubsection{Modulhandbuch wiederherstellen}
Falls ein Modulhandbuch unabsichtlich gelöscht wurde, kann es im direkten Anschluss durch einen Klick auf den Knopf \textit{Wiederherstellen} wiederhergestellt werden (siehe \figref{fig:m20}).
Dies muss jedoch direkt nach dem Löschen passieren und ist sonst nicht mehr möglich.
\begin{figure}[h!]
\centering
\includegraphics[width=0.75\linewidth]{img//Modulhandbuch//M20.png}
\caption{Modulhandbuch}
\label{fig:m20}
\end{figure}
