\section{Benutzerverwaltung}
\label{sec:benutzer}

\begin{warning}
    Login Daten gelten semesterübergreifend. Rollen werden im aktuellen und in allen neueren Semestern geändert.
\end{warning}

\subsection{Benutzer erstellen}
Um neue Benutzer im WebTool zu erstellen, muss auf die Seite \textit{Benutzerverwaltung} gewechselt und  der Knopf 
\textit{Erstellen} geklickt werden (siehe \figref{fig:benutzer1}).
\begin{figure}[h!]
    \centering
    \includegraphics[width=0.75\linewidth]{img//Benutzer//Benutzer1.png}
    \caption{Benutzer erstellen}
    \label{fig:benutzer1}
\end{figure}
Hier muss ein Name für die neue Person erstellt werden, ein Kürzel festgelegt werden (wird automatisch erzeugt, kann aber verändert werden) und die Fakultät der neuen Person angegeben werden.
Anschließend muss die Rolle der Person ausgewählt werden, indem das Dropdown Menu \textit{Rollen} geklickt wird (siehe \figref{fig:benutzer2}).
\begin{figure}[h!]
    \centering
    \includegraphics[width=0.75\linewidth]{img//Benutzer//Benutzer2.png}
    \caption{Benutzer erstellen}
    \label{fig:benutzer2}
\end{figure}
In diesem Menu können eine oder mehrere Rollen ausgewählt werden (siehe \figref{fig:benutzer3}).
Die möglichen Rollen sind:
\begin{itemize}
    \item Veranstalter
    \item Veranstaltungsbeauftragter
    \item Modulhandbuchbeauftragter
    \item Administrator
    \item Planer
    \item Beobachter
\end{itemize}
\begin{figure}[h!]
    \centering
    \includegraphics[width=0.75\linewidth]{img//Benutzer//Benutzer3.png}
    \caption{Benutzer erstellen}
    \label{fig:benutzer3}
\end{figure}
Danach können die Eingaben mit einem Klick auf den blauen Knopf \textit{Speichern} gespeichert werden (siehe \figref{fig:benutzer4}).
\begin{figure}[h!]
    \centering
    \includegraphics[width=0.75\linewidth]{img//Benutzer//Benutzer4.png}
    \caption{Benutzer erstellen}
    \label{fig:benutzer4}
\end{figure}
Anschließend gelangt man in die Liste mit allen existierenden Benutzern.

\subsection{Benutzer bearbeiten}
Um einen Benutzer zu bearbeiten, muss auf diesen in der Liste mit allen Benutzern geklickt werden (siehe \figref{fig:benutzer5}).
\begin{figure}[h!]
    \centering
    \includegraphics[width=0.75\linewidth]{img//Benutzer//Benutzer5.png}
    \caption{Benutzer bearbeiten}
    \label{fig:benutzer5}
\end{figure}
Dann öffnet sich das gleiche Fenster wie beim Erstellen von Benutzern und die Eingaben können wie beim Erstellen mit einem Klick auf \textit{Speichern} gespeichert werden (siehe \figref{fig:benutzer6}).
\begin{figure}[h!]
    \centering
    \includegraphics[width=0.75\linewidth]{img//Benutzer//Benutzer6.png}
    \caption{Benutzer bearbeiten}
    \label{fig:benutzer6}
\end{figure}

\subsection{Login erstellen}
\label{sec:benutzer:create}
Um einen Benutzernamen und ein Passwort für die Benutzer zu erstellen, damit sich diese ins WebTool einloggen können, muss der Benutzer bearbeitet werden und dabei auf den Knopf \textit{Erstellen} unter der Überschrift \textit{Benutzer} gedrückt werden  (siehe \figref{fig:benutzer7}).
Ein Benutzer kann mehrere Benutzernamen und Passwörter haben. 
Dies ist sinnvoll, wenn ein Sekretariat die Eintragungen vornehmen soll und dafür ein eigenes Passwort bekommen soll.
\begin{figure}[h!]
    \centering
    \includegraphics[width=0.75\linewidth]{img//Benutzer//Benutzer7.png}
    \caption{Login erstellen}
    \label{fig:benutzer7}
\end{figure}
Hier muss eine Benutzerbezeichnung festgelegt werden sowie ein Benutzername und Passwort für den Benutzer erstellt werden.
Die Eingaben werden mit einem Klick auf den blauen Knopf \textit{Speichern} gepeichert (siehe \figref{fig:benutzer8}).
\begin{figure}[h!]
    \centering
    \includegraphics[width=0.75\linewidth]{img//Benutzer//Benutzer8.png}
    \caption{Login erstellen}
    \label{fig:benutzer8}
\end{figure}

\subsection{Login ändern}
Um die Login Daten zu ändern, muss auf den Benutzernamen geklickt werden (siehe \figref{fig:benutzer9}).
Um Login Daten zu löschen, muss auf den Knopf \textit{Löschen} gedrückt werden.
Danach können der Benutzername und das Passwort wie in Kapitel \ref{sec:benutzer:create} verändert und gespeichert werden.
\begin{figure}[h!]
    \centering
    \includegraphics[width=0.75\linewidth]{img//Benutzer//Benutzer9.png}
    \caption{Login ändern}
    \label{fig:benutzer9}
\end{figure}

\subsection{Benutzer löschen}
Wenn ein Benutzer gelöscht werden soll, etwa wenn die Person die Universität verlässt, kann der Benutzer durch einen Klick auf \textit{Löschen} in der Benutzerverwaltung gelöscht werden (siehe \figref{fig:benutzer10}).
\begin{figure}[h!]
    \centering
    \includegraphics[width=0.75\linewidth]{img//Benutzer//Benutzer10.png}
    \caption{Benutzer löschen}
    \label{fig:benutzer10}
\end{figure}
