\subsubsection{Prerequisites}

In order to be able to implement the web service database import for
your custom database implementation you need to have created the
following classes:
\begin{itemize}
	\item MyWebServiceDatabase (see \autoref{subsec:web-service} for more information) 
	\item MyLocalProgramDatabase (see \autoref{subsec:local-program-db} for more
	information)
\end{itemize}

\hypertarget{implementing-the-interface}{%
\subsubsection{Implementing the Interface}\label{implementing-the-interface}}

The implementation needs additional functionality that is not provided
by MyWebServiceDatabase and MyLocalProgramDatabase through their
corresponding interfaces. This is by design since you do not want to
grant the user more functionality than absolutely necessary in order to
mitigate the risk of forbidden or unwanted changes. Thus, you will need
to implement extensions to your database interfaces whose access is
restricted to the database package only. The additional functionality
for the WebServiceDatabase is provided by the interface
\href{https://git.cs.upb.de/stulp.imt/database/blob/master/src/main/java/de/upb/stulp/database/localProgram/databaseImport/WebServiceDatabaseInternal.java}{WebServiceDatabaseInternal}
and for the LocalProgramDatabase by the interface
\href{https://git.cs.upb.de/stulp.imt/database/blob/master/src/main/java/de/upb/stulp/database/BatchOperationsAvailable.java}{BatchOperationsAvailable}.

\hypertarget{webservicedatabaseinternal-implementation}{%
\subsubsection{WebServiceDatabaseInternal
implementation}\label{webservicedatabaseinternal-implementation}}

\href{https://git.cs.upb.de/stulp.imt/database/blob/master/src/main/java/de/upb/stulp/database/localProgram/databaseImport/WebServiceDatabaseInternal.java}{WebServiceDatabaseInternal}
only lists the additional functionality and not the functionality that
is provided by
\href{https://git.cs.upb.de/stulp.imt/database/blob/master/src/main/java/de/upb/stulp/database/webService/WebServiceDatabase.java}{WebServiceDatabase}.
This decision has been made because you are now able to reuse your
implementation of MyWebServiceDatabase. To implement the interface for
the import, create the following class:

\begin{lstlisting}[language=Java]
public final class MyWebServiceDatabaseInternal 
				extends MyWebServiceDatabase 
				implements WebServiceDatabaseInternal {
    /**
     * Default constructor. Initializes 
     * this class from a database.
     *
     * @param db the database to use
     * @throws DatabaseException
     */
    MyWebServiceDatabaseInternal(MyWebServiceDatabase db) 
    						    throws DatabaseException {
        super(db);
    }

    @Override
    public void makeVorlesungsplanOld() 
    							throws DatabaseException {
        //copy content from Vorlesungsplan 
        //to AlterVorlesungsPlan
    }
}
\end{lstlisting}

Now you have implemented the necessary functionality for the web service
database.

\hypertarget{batchoperationsavailable-implementation}{%
\subsubsection{BatchOperationsAvailable
implementation}\label{batchoperationsavailable-implementation}}

\href{https://git.cs.upb.de/stulp.imt/database/blob/master/src/main/java/de/upb/stulp/database/BatchOperationsAvailable.java}{BatchOperationsAvailable}
adds batch inserts and updates to a database interface. Here you can
also reuse your implementation of MyLocalProgramDatabase. To implement
the interface for the import, create the following class:

\begin{lstlisting}[language=Java]
public final class MyLocalProgramImportDatabase 
				extends MyLocalProgramDatabase 
				implements BatchOperationsAvailable {

    MyLocalProgramImportDatabase(String dbPath) 
    						throws DatabaseException {
        super(dbPath);
    }

    @Override
    public <T extends BatchRecord> 
    void batchInsert(List<T> records) 
    			throws DatabaseException {
        //TODO: implement
    }

    @Override
    public <T extends BatchRecord> 
    void batchUpdate(List<T> records) 
    			throws DatabaseException {
        //TODO: implement
    }
}
\end{lstlisting}

Now you have implemented the necessary functionality for the local
program database.

\hypertarget{webservicedatabaseimport-implementation}{%
\subsubsection{WebServiceDatabaseImport
implementation}\label{webservicedatabaseimport-implementation}}

Now that you have created the needed classes for the
\href{https://git.cs.upb.de/stulp.imt/database/blob/master/src/main/java/de/upb/stulp/database/localProgram/databaseImport/WebServiceDatabaseImport.java}{WebServiceDatabaseImport}
you can now implement this interface. Implementation of this interface
is particularly easy because all the importing can be done using the
interface methods and is therefore already implemented. You only need to
create a class feeding the algorithm with your custom database classes.
Thus, create the following class using your custom databases:

\begin{lstlisting}[language=Java]
public final class MyWebServiceDatabaseImport 
			extends WebServiceDatabaseImport<
						MyLocalProgramImportDatabase,
                        MyWebServiceDatabase> {

    /**
     * Constructs this object to import to the given database.
     *
     * @param localProgramDb  the local program database 
     		  (to import to)
     * @throws DatabaseException
     */
    public MyWebServiceDatabaseImport()
    				MyLocalProgramDatabase localProgramDb) 
    							throws DatabaseException {
        super(new MyLocalProgramImportDatabase(
        		localProgramDb.getDBPath()
        	 ));
    }

    @Override
    protected void makeVorlesungsplanOld(
    							MyWebServiceDatabase old) 
    								throws DatabaseException {
        new MyWebServiceDatabaseInternal(old)
        					.makeVorlesungsplanOld();
    }
}
\end{lstlisting}

If you have implemented everything according to the instructions given,
you will only need to replace MyWebserviceDatabaseImport,
MyLocalProgramImportDatabase, MyWebServiceDatabase,
MyLocalProgramDatabase and MyWebServiceDatabaseInternal by the class
names you have chosen.

\hypertarget{packaging-to-prevent-potential-access-issues-when-instantiating-mylocalprogramimportdatabase-and-mywebservicedatabaseinternal}{%
\subsubsection{Packaging to prevent potential access issues when instantiating
MyLocalProgramImportDatabase and
MyWebServiceDatabaseInternal}\label{packaging-to-prevent-potential-access-issues-when-instantiating-mylocalprogramimportdatabase-and-mywebservicedatabaseinternal}}

Make sure to put MyWebServiceDatabaseImport,
MyLocalProgramImportDatabase and MyWebServiceDatabaseInternal into the
same package. If you do not do this, you will get problems instantiating
MyLocalProgramImportDatabase and MyWebServiceDatabaseInternal from
MyWebServiceDatabaseImport since their constructors are package private
to prevent the user from instantiating them.
