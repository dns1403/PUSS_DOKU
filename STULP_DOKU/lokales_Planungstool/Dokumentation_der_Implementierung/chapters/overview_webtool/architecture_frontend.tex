\section{Architecture of the web tool front-end}

The front end is built in Angular framework which uses typescript as a programming language and HTML/CSS for user interface development. The figure below shows a typical architecture of an Angular application which is followed by our tool.

 \begin{figure}[H]
	\centering
	\includegraphics[width=1.3\textwidth]{chapters/overview_webtool/images/architecture_frontend.png}
	\caption{Architecture of the Angular application}
	\label{fig:ArchitectureFrontend}
\end{figure}

\subsection{Modules}
The Angular application is composed of exportable modules. The modules contains the features which can be exported and can be imported in other modules. The main module of the Angular application is the App module. An example of a module import is that App module imports the \textit{ReactiveFormsModule} using which we create forms in the application which takes user input. All such required modules are imported in the App module.

\subsection{Routes}
The Angular router is used for navigation throughout the Angular application. A route is a URL which is tied to a specific component. When a URL is entered in the browser, the Angular router looks for the associated component and renders it on the browser screen.

\subsection{Component} The Angular application is made of components. A component is a view which is displayed on the browser screen. A component can have other components inside it. The root component is the \textit{App} component. All the components are loaded inside this component. For example when a URL \textit{<base url>/kursname} is opened in the browser, a list of courses which are saved in the database can be seen. The view displayed on the screen is the \textit{Kursname} component loaded inside the root \textit{App} component.

\subsubsection{HTML Template}
A component has an HTML template. It is a file with .html extension. The user interface is built in HTML which is written in this file.

\subsubsection{Code / Controller} Each component has a controller associated with it which is a class representing the component written in a file having a .ts extension. For example the "Kursname" component has a controller file kursname.component.ts which contains the class \textit{KursnameComponent}. This controller class contains the user interaction logic related to the component.

\subsubsection{How to create a new component} To create a new component execute the following command in the Anglular CLI after navigating to \textit{src/app/components} directory.

\begin{lstlisting}[language=bash]
$ ng generate component <Your component name>
\end{lstlisting}

The above command will create .html, .ts and .css files. The HTML for user interface will be written in .html file. The user interaction logic and code for passing user input to services can be written in .ts file which is a Controller for the component. The CSS for styling of this specific component can be written in the .css file.
To bind a component to a route such that this component can be displayed by typing a URL, this mapping needs to be defined in the file \textit{src/app/app-routing.module.ts}. In this file many examples of such a mapping can be find. For example to create a new Studiengang, the route \textit{studiengang/create} is mapped to \textit{StudiengangComponent}, so when the URL \textit{<base url>/studiengang/create} is opened in the browser, the user interface written in studiengang.component.html file is displayed.

\subsection{Service} The services contain the shared code and logic which can be injected into different components. The code to interact with the backend is also written in the Services. For example the code to make an API call to backend to get a list of records can be found in \textit{HttpService} class available at the path \textit{src/app/services/http.service.ts}. This class is injected in different components where a list of records need to be fetched.

\subsubsection{How to create a new Service} To create a new service execute the following command in the Anglular CLI after navigating to \textit{src/app/services} directory.

\begin{lstlisting}[language=bash]
$ ng generate service <Your service name>
\end{lstlisting}
	
The above command will create a new .ts file in services directory containing an empty class in which new service methods can be written. To use this service in a component, inject it into the constructor of the controller class of that component. An example can be found in \textit{src/app/components/\_shared/base-list/base-list.component.ts} where \textit{HttpService} is injected in the constructor of the class to be used in this component. Using this service, this component gets the list of records from the backend via an API call.

\subsection{Structure of the front-end}

The structure of the front-end can be seen in the figure below.

 \begin{figure}[H]
	\centering
	\includegraphics[width=.7\textwidth]{chapters/overview_webtool/images/structure_frontend.png}
	\caption{Directory structure of the web tool front-end}
	\label{fig:StructureFrontend}
\end{figure}

Although there are more auto-generated files and directories which are used by the Angular framework internally and are not directly related to the development or developers. The directory structure is explained below.

 \begin{itemize}
	\item \textit{src/app/components} : This directory contains all the components. Please refer to the explanation of components above in the section 2.1.3
	
	\textit{src/app/components/\_shared} : This directory contains the shared components which are used inside other components.
	\item \textit{src/app/helpers} : This directory contain different helper classes.
	
	\textit{src/app/helpers/auth.guard.ts} : This file contains a helper class which implements \textit{CanActivate} interface of the \textit{@angular/router} module. This interface contains a  method \textit{canActivate} which is triggered for the routes which needs to be protected from unauthenticated access. For example, a user can access a login screen without logging into system, however a user cannot see a list of courses without logging into the system. The logic to handle that is written in this function which checks if a user is logged into the system or not, otherwise it is redirected to login screen.
	
	\textit{src/app/helpers/custom.validators.ts} : This class contains custom validation functions. For example a function to validate file extension and a function to validate if two passwords match. Any new custom validation functions can be added in this class.
	
	\textit{src/app/helpers/error.interceptor.ts} : This class implements \textit{HttpInterceptor} interface provided by the \textit{@angular/common/http} module. This interface intercepts and handles a \textit{HttpRequest} or {HttpResponse}. The \textit{intercept} method is implemented. This method intercepts the response of every API call made from the application and if the response contains an error, it is handled and a further action is taken, for example show the error as an alert on the screen or navigate a user to a different page.
	
	\textit{src/app/helpers/jwt.interceptor.ts} : This class also implements \textit{HttpInterceptor} interface provided by the \textit{@angular/common/http} module. Here the \textit{intercept} method is implemented such that an authorization header is added in every request sent to the back-end. The APIs of our web tool are secured by OAuth authentication mechanism and hence an authorization token is needed to be sent in the header of every request. Otherwise the request is considered as unauthenticated.
	
	\item \textit{src/app/language/language.ts} : This file is used to define static labels and options for drop-down lists. The reason for specifying these things in a separate file is to be able to easily change the labels/texts for the display in user interface without changing them in the code directly where they are used.
	
	\item \textit{src/app/metadata/metadata.ts} : This file contains the metadata for all models. For example a metadata for a \textit{Studiengang} contains the information that which fields should be displayed in a table showing a list of records for a Studiengang. The metadata also contains information about the backend API endpoint for Studiengang, the path to its component, the name of the identity column, the metadata of the fields. This metadata is defined such that the user interface is metadata driven and generic.
	
	\item \textit{src/app/models} : This directory contains different model classes. Models are defined for static typing of the objects, however it is also possible to create objects in typescript without static typing. And usually due to the flexibility required in the user interface logic, the static types are not used. This is the reason that all models are not defined here and also the models which are defined are not always used.
	
	\item \textit{src/app/services} : This directory contains all the services. Please refer to the explanation of the services above in the section 2.1.4.
	
	\item \textit{src/app/app-routing-module.ts} : The routing module is defined in this file. Please refer to the explanation of Routes above in the section 2.1.2.
	
	\item \textit{src/app/app.component.[css,html,ts]} : These three files belong to \textit{App} component. Please refer to component section 2.1.3 above for the details of the \textit{App} component.
	
	\item \textit{src/app/app.module.ts} : This file contains the main \textit{App} module. Please refer to section 2.1.1 for details about \textit{App} module.
	
	\item \textit{src/app/assets} : This directory contains the static assets, for example the university logo which is displayed in the header of the application.
	
	\item \textit{src/app/environments} : This directory contains the environment specific configuration files. The file enviroment.ts.sample is such a sample file. This file contains information such as the API URL of the backend or the client id to be sent in the authentication request. The API URL could be different for different environments for example development, testing, and production. The separate environment files can be configured for all these environments.
	
	\item \textit{src/app/index.html} : This file is the entry point of the user interface. The root \textit{App} component is placed in this file which loads all other components.
	
	\item \textit{src/app/styles.css} : This is a global style file for the application. The style defined in this class is applied to every component.
	
\end{itemize}