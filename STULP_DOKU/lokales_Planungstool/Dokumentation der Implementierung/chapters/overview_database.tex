\section{Overview of the database architecture}\label{section1}
This chapter describes the implementation of the database interfaces used by the local program. Implementing these interfaces guarantees that the used database implementations can be replaced in the local program or web tool without changing their code base.
\\\\
The following sections describe each database interface, including the schema used, its usage, and a detailed description of how to maintain the implementation if, for example, new functionality is needed.
\\\\
To make the schema definitions simpler, the following abbreviations are used:
\begin{table}[H]
	\centering
	\begin{tabular}{|c|c|}
		\hline
		Abbreviation & Meaning \\\hline\hline
		PK & Primary Key\\
		FK & Foreign Key \\
		NN & Not Null\\
		U & Unique \\
		AI & Auto Incrementing \\\hline
	\end{tabular}
	\caption{Abbrevations and their meaning}
\end{table}
%
\noindent Please note that when using \textit{SQLite}, it is more efficient not to use auto increment for non-composite integer primary keys because the database will automatically create an alias for its row id column. This is also why no auto-increment constraint is found for this kind of primary key in our implementation. When referring to a foreign key constraint referencing, e.g., the \textit{ProfID} field of the \textit{Professor} relation, \(FK(Professor)\) is used.\\


\noindent The source code for the database can be found at \\ \href{https://git.cs.uni-paderborn.de/stulp/sven\_database}{https://git.cs.uni-paderborn.de/stulp/sven\_database}.

%%\section{Web Tool Interface}
%The Web Service uses different interfaces for different purposes. In this section it is described which interfaces are used and how to create instances of these interfaces. Details on how to implement these interfaces can be found in their respective sections.
%\subsection{Web service database interface}
\label{subsec:web-service}
This is the main database interface that is used to store all the necessary
planning data that is later exported to the local program for actual
planning. Its schema has been given by Nils Rothkamm as part of his bachelor thesis.
\subsubsection{Usage}
Usage (without try/catch for exception handling to support readability):

\begin{lstlisting}[language=Java]
/**Instantiation**/
SQLiteLocalProgramDatabase ldb = 
		new SQLiteLocalProgramDatabase(
			"path/to/local_program.db"
		);
SQLiteWebServiceDatabase wdb = 
		new SQLiteWebServiceDatabase(
			"path/to/web_service.db"
		);
SQLiteWebServiceDatabaseImport imp = 
		new SQLiteWebServiceDatabaseImport(ldb);
/**invoke import**/
ldb = imp.importData(wdb);
\end{lstlisting}
\subsubsection{Schema and sample data}
\begin{landscape}
	\begin{table}
		\centering
		\begin{tabular}{|c|c|c|c|c|c|}
			\hline
			& KursID & Veranstaltungsteil & Veranstaltungsart & Dauer & BemerkungPlanung \\\hline\hline
			Data Type & TEXT & INTEGER & TEXT & INTEGER & TEXT \\
			Constraints & NN, PK, FK(Kursnamen) & NN, PK & NN & NN & \\
			Checks & & & \(\in\href{https://git.cs.upb.de/stulp.imt/database/blob/master/src/main/java/de/upb/stulp/database/enums/Veranstaltungsart.java}{Veranstaltungsart}\) & & \\\hline
			& MOD & 1 & Vorlesung & 3 & mit Projektor\\
			& MOD & 2 & Übung & 2 &  \\\hline
		\end{tabular}
		\caption{Schema of Veranstaltungen relation with sample records}
		\label{schema:veranstaltungen}
	\end{table}
	%
	\begin{table}
		\centering
		\begin{tabular}{|c|c|c|c|c|c|c|c|}
			\hline
			& KursID & Kursname & KursnameEN & Fachbereich & BeauftragterID & PaulNr & VeranstalterGeneriert \\\hline\hline
			Data Type & TEXT & TEXT & TEXT & TEXT & TEXT & TEXT & INTEGER\\
			Constraints & NN, PK, U & NN & NN & NN & NN & NN & NN \\
			Checks &  &  &  &  &  &  & \(\in\{0,1\}\) \\\hline
			& MOD & Modellierung & Modelling & Informatik & Lessmann & L.0.231.1 & 0\\
			& Prog & Programmierung & Programming & Informatik & Lessmann & L.0.1.1 & 0\\\hline
		\end{tabular}
		\caption{Schema of Kursnamen relation with sample records}
		\label{schema:kursnamen}
	\end{table}
	%
	\begin{table}
		\centering
		\begin{tabular}{|c|c|c|c|c|c|c|c|}
			\hline
			& StudiengangsID & Hauptfach & Nebenfach & Semester & Prüfungsordnung & BeauftragterID & Aktiv \\\hline\hline
			Data Type & TEXT & TEXT & TEXT & INTEGER & INTEGER & TEXT & TEXT\\
			Constraints & NN, PK, U & NN & NN & NN & NN & NN & NN \\
			Checks & &  &  &  &  &  & \(\in\{Ja, Nein\}\) \\\hline
			& infmathbpo3 & Informatik & Mathematik & 1 & 3 & Lessmann & Nein\\
			& infmathbpo4 & Informatik & Mathematik & 1 & 4 & Lessmann & Ja\\\hline
		\end{tabular}
		\caption{Schema of Studiengänge relation with sample records}
		\label{schema:studiengänge}
	\end{table}
	%
	\begin{table}
		\centering
		\begin{tabular}{|c|c|c|c|c|c|c|}
			\hline
			& StudiengangsID & KursID & Modulname & Veranstaltungsteile & ECTS & Pflichtfach \\\hline\hline
			Data Type & TEXT & TEXT & TEXT & TEXT & INTEGER & TEXT \\
			Constraints & NN, PK, FK(Studiengänge) & NN, PK, FK(Kursnamen) & NN & NN & NN & NN \\
			Checks & &  &  &  &  & \(\in\{Ja, Nein\}\) \\\hline
			& infmathbpo4 & MOD & Modellierung & [1,2,3]  & 9 & Ja \\
			& infmathbpo4 & Prog & Programmierung & [1,2,3,4] & 4 & Ja \\\hline
		\end{tabular}
		\caption{Schema of Module relation with sample records}
		\label{schema:module}
	\end{table}
	%
	\begin{table}
		\centering
		\begin{tabular}{|c|c|c|c|}
			\hline
			& StudiengangsID & KursID & Druckrubrik \\\hline\hline
			Data Type & TEXT & TEXT & TEXT \\
			Constraints & NN, PK, FK(Studiengänge) & NN, PK, FK(Kursnamen) & NN, PK \\
			Checks & &  & \\\hline
			& infmathbpo4 & MOD & Informatik Bachelor - Pflichtteil (POv4)\\
			& infmathbpo4 & Prog & Informatik Bachelor - Pflichtteil (POv4) \\\hline
		\end{tabular}
		\caption{Schema of Druckrubriken relation with sample records}
		\label{schema:druckrubriken}
	\end{table}
	%
	\begin{table}
		\centering
		\begin{tabular}{|c|c|c|c|}
			\hline
			& Bezeichnung & Parent & Reihenfolge \\\hline\hline
			Data Type & TEXT & TEXT & INTEGER \\
			Constraints & NN, PK & PK & \\
			Default & & & 1 \\\hline
			& Informatik Bachelor (POv4) & & 1 \\
			& Informatik Bachelor - Pflichtteil (POv4) & Informatik Bachelor (POv4) & 1\\\hline
		\end{tabular}
		\caption{Schema of Druckrubrikbezeichnungen relation with sample records}
		\label{schema:druckrubrikbezeichnungen}
	\end{table}
	%
	\begin{table}
		\centering
		\tiny
		\begin{tabular}{|c|c|c|c|c|c|c|c|c|}
			\hline
			& ProfID & KursID & Veranstaltungsteil & Veranstaltungsart & Übungsanzahl &
			Wunschzeit & Sprache & Hoererzahl \\\hline\hline
			Data Type & TEXT & TEXT & INTEGER & TEXT & INTEGER & TEXT & TEXT & INTEGER \\
			Constraints & \multicolumn{1}{p{3cm}}{NN, PK, FK(Professor)} & \multicolumn{1}{p{3cm}}{NN, PK, FK(Veranstaltungen)} & \multicolumn{1}{p{3cm}}{NN, PK, FK(Veranstaltungen)} & NN & NN & & NN & \\
			Check & & & & \(\in\href{https://git.cs.upb.de/stulp.imt/database/blob/master/src/main/java/de/upb/stulp/database/enums/Veranstaltungsart.java}{Veranstaltungsart}\) & & & & \\\hline
			& N.N. & MOD & 3 & Übung & 2 & [5,9,12] & Deutsch & 20 \\
			& Scheidler & MOD & 1 & Vorlesung & 0 & [] & Deutsch & 50 \\\hline
		\end{tabular}
	\begin{tabular}{|c|c|c|c|c|c|c|c|c|}
		\hline
		& Zweiwoechentlich & Startwoche & Tafel & Beamer & Fenster & Ort & SonstigesRaumplanung & SonstigesZeitplanung \\\hline\hline
		& INTEGER & INTEGER & INTEGER & INTEGER & INTEGER & INTEGER & TEXT & TEXT\\
		& Default(0) & Default(1) & Default(0) & Default(0) & Default(0) & Default(3) & & \\
		& \(\in\{0,1\}\) & & \(\in\{0,1\}\) & \(\in\{0,1\}\) & \(\in\{0,1\}\) & \(\in\href{https://git.cs.upb.de/stulp.imt/database/blob/master/src/main/java/de/upb/stulp/database/enums/Ort.java}{Ort}\) & & \\\hline
		& 1 & 2 & 1 & 1 & 1 & 1 & Seminarraum & \\
		& 0 & 0 & 1 & 1 & 1 & 1 & & egal \\\hline
	\end{tabular}
		\caption{Schema of Vorlesungsplan and Alter\_Vorlesungsplan relation with sample records}
		\label{schema:vorlesungsplan}
	\end{table}
	%
	\begin{table}
		\centering
		\begin{tabular}{|c|c|c|c|c|}
			\hline
			& ProfID & Sperrtag & Sperrzeit & Dauer \\\hline\hline
			Data Type & TEXT & TEXT & TEXT & INTEGER \\
			Constraints & NN, PK, FK(Professor) & NN, PK & NN, PK & NN \\
			Check & & \(\in\href{https://git.cs.upb.de/stulp.imt/database/blob/master/src/main/java/de/upb/stulp/database/enums/Tag.java}{Tag}\) & & \\\hline
			& Scheidler & Montag & 8:00 & 6 \\
			& Scheidler & Mittwoch & 8:00 & 1 \\\hline
		\end{tabular}
		\caption{Schema of Sperrzeiten relation with sample records}
		\label{schema:sperrzeiten}
	\end{table}
	%
	\begin{table}
		\centering
		\begin{tabular}{|c|c|c|c|}
			\hline
			& ProfID & Name & Fakultät \\\hline\hline
			Data Type & TEXT & TEXT & TEXT \\
			Constraints & NN, PK& NN & NN \\
			Check & & & \\\hline
			& Scheidler & Prof. Dr. Christian Scheidler & EIM \\
			& Böttcher & Prof. Dr. Stefan Böttcher & EIM \\\hline
		\end{tabular}
		\caption{Schema of Professor relation with sample records}
		\label{schema:professor}
	\end{table}
\end{landscape}
\subsubsection{Implementing the interface}
The main interface for the TeX/PDF Backup is the
\href{https://git.cs.upb.de/stulp.imt/database/blob/master/src/main/java/de/upb/stulp/database/webService/teXPdfBackup/TeXPdfBackupDatabase.java}{TeXPdfBackupDatabase}.
To implement this main interface simply do the following and implement
the methods you need to overwrite:

\begin{lstlisting}[language=Java]
public class MyTeXPdfBackupDatabase 
				implements TeXPdfBackupDatabase {
...
}
\end{lstlisting}

An instance of this class will then be used in the Web Service to query
the database. Keep in mind that you also need to adjust the Web Service
to use your custom class.

When implementing MyTeXPdfBackupDatabase you will notice that you also
have to return an instance of a class implementing the
\href{https://git.cs.upb.de/stulp.imt/database/blob/master/src/main/java/de/upb/stulp/database/webService/teXPdfBackup/TeXPdfBackupRecordFactory.java}{TeXPdfBackupRecordFactory}
interface as this is the only option for users of your interface to
create objects relating to tables of the database. This is needed
because the user uses these objects to pass filter information to the
queries. Thus, you will need to create the following class:

\begin{lstlisting}[language=Java]
public class MyTeXdPdfBackupRecordFactory 
				implements TeXPdfBackupRecordFactory {
...
}
\end{lstlisting}

You will also notice that the database interface only offers one insert,
update, delete and getRecords operation, yet there can be different
tables in your database. This is due to the fact that these methods
simply delegate the operation to the different table implementations. To
take advantage of this the
\href{https://git.cs.upb.de/stulp.imt/database/blob/master/src/main/java/de/upb/stulp/database/Record.java}{Record}
interface has been introduced. This interface enables you to simply call
the corresponding method of the record interface for delegation.

\subsubsection{How are Records of database relations implemented in the interface}

Every relation in your database has a corresponding entity class that
stores exactly one record of a relation, e.g.~
\href{https://git.cs.upb.de/stulp.imt/database/blob/master/src/main/java/de/upb/stulp/database/webService/teXPdfBackup/TeXPdf.java}{TeXPdf}.
Thus, when querying the database, you wrap every record into its
corresponding interface class and return this.

\subsubsection{How to implement a record class}

To implement a record class, you simply inherit from the corresponding
interface class and override the missing methods. For example:

\begin{lstlisting}[language=Java]
public class MyTeXPdf extends TeXPdf {
/*
 * make sure to make the constructors protected 
 * such that the user is forced to use 
 * the RecordFactory for instantiation
 */
...
}
\end{lstlisting}

The interface will force you to implement how such a record should be
inserted, updated, deleted and retrieved from the database. However, it
will not force you to implement the creation of a possibly necessary
table since different databases use different concepts. Keep that in
mind and consult your database manual to see what you need. You also
need to overwrite the factory methods in MyTeXPdfBackupRecordFactory to
be able to instantiate your class from outside this package.

\subsubsection{How to extend the database interface}

There are two ways of extending the database interface that are quite
different in implementation effort. 1. You need to provide new
functionality that is not yet covered by the interface but the database
schema has not changed. 2. You need to change the database schema.

\subsubsection{Add new functionality without database schema change}

In order to add new functionality to the database that does not need a
schema change, you only need to add the required method to the
\href{https://git.cs.upb.de/stulp.imt/database/blob/master/src/main/java/de/upb/stulp/database/webService/teXPdfBackup/TeXPdfBackupDatabase.java}{TeXPdfBackupDatabase}
interface and add the implementation of this method to
MyTeXPdfBackupDatabase.

\subsubsection{Change the schema}

When changing the schema you have to differentiate between updating
already existent relations and adding new relations.

When updating existent relations you only need to update the interface
class that implements the Record interface, e.g.~TeXPdf, and your
implementation, e.g.~MyTeXPdf, depending on what you need to change.
When using the
\href{https://git.cs.upb.de/stulp.imt/database/blob/master/src/main/java/de/upb/stulp/database/impl/webService/teXPdfBackup/SQLiteTeXPdf.java}{SQLiteTeXPdf}
implementation that is provided in this repository, you will also need
to update the
\href{https://git.cs.upb.de/stulp.imt/database/blob/master/src/test/resources/tex_backup_schema.db}{tex\_backup\_schema.db}
as this implementation is using JOOQ and uses automatic code generation
to create the corresponding classes to enable type safe SQL queries.

When adding a new relation to the database you need to create the
following two classes that represent a record of the new relation:

\begin{lstlisting}[language=Java]
public abstract class NewRelation implements Record {
//private nullable field for every field of the relation. 
//protected constructors
//getter and setter
//overwrite equals and hashCode methods
}
\end{lstlisting}

\begin{lstlisting}[language=Java]
public class MyNewRelation extends NewRelation {
//implement methods
}
\end{lstlisting}

After creating these classes, you need to add the factory methods for
the relation records to
\href{https://git.cs.upb.de/stulp.imt/database/blob/master/src/main/java/de/upb/stulp/database/webService/teXPdfBackup/TeXPdfBackupRecordFactory.java}{TeXPdfBackupRecordFactory}.
This will look like this:

\begin{lstlisting}[language=Java]
public interface TeXPdfBackupRecordFactory {
//place holder for old interface methods
NewRelation createNewRelation();
NewRelation createNewRelation(/*list of all fields to set*/);
}
\end{lstlisting}

Then add the implementation again to MyTeXPdfBackupRecordFactory. After
that you have successfully added a relation to your database due to
delegation in the TeXPdfBackupDatabase. There is however one exception.
If you are using the SQLite interface provided by this repository, you
will again need to create a demo table in
\href{https://git.cs.upb.de/stulp.imt/database/blob/master/src/test/resources/tex_backup_schema.db}{tex\_backup\_schema.db}
to define the schema for JOOQ. You also need to make sure to create the
new table when instantiating the database.

%\subsection{User data database interface}
The user data database interface is used to store and access all the
user information of the web service.
\subsubsection{Usage}
Usage (without try/catch for exception handling to support readability):

\begin{lstlisting}[language=Java]
/**Instantiation**/
SQLiteLocalProgramDatabase ldb = 
		new SQLiteLocalProgramDatabase(
			"path/to/local_program.db"
		);
SQLiteWebServiceDatabase wdb = 
		new SQLiteWebServiceDatabase(
			"path/to/web_service.db"
		);
SQLiteWebServiceDatabaseImport imp = 
		new SQLiteWebServiceDatabaseImport(ldb);
/**invoke import**/
ldb = imp.importData(wdb);
\end{lstlisting}
\subsubsection{Schema and sample data}
\begin{landscape}
	\begin{table}
		\centering
		\begin{tabular}{|c|c|c|c|c|c|}
			\hline
			& KursID & Veranstaltungsteil & Veranstaltungsart & Dauer & BemerkungPlanung \\\hline\hline
			Data Type & TEXT & INTEGER & TEXT & INTEGER & TEXT \\
			Constraints & NN, PK, FK(Kursnamen) & NN, PK & NN & NN & \\
			Checks & & & \(\in\href{https://git.cs.upb.de/stulp.imt/database/blob/master/src/main/java/de/upb/stulp/database/enums/Veranstaltungsart.java}{Veranstaltungsart}\) & & \\\hline
			& MOD & 1 & Vorlesung & 3 & mit Projektor\\
			& MOD & 2 & Übung & 2 &  \\\hline
		\end{tabular}
		\caption{Schema of Veranstaltungen relation with sample records}
		\label{schema:veranstaltungen}
	\end{table}
	%
	\begin{table}
		\centering
		\begin{tabular}{|c|c|c|c|c|c|c|c|}
			\hline
			& KursID & Kursname & KursnameEN & Fachbereich & BeauftragterID & PaulNr & VeranstalterGeneriert \\\hline\hline
			Data Type & TEXT & TEXT & TEXT & TEXT & TEXT & TEXT & INTEGER\\
			Constraints & NN, PK, U & NN & NN & NN & NN & NN & NN \\
			Checks &  &  &  &  &  &  & \(\in\{0,1\}\) \\\hline
			& MOD & Modellierung & Modelling & Informatik & Lessmann & L.0.231.1 & 0\\
			& Prog & Programmierung & Programming & Informatik & Lessmann & L.0.1.1 & 0\\\hline
		\end{tabular}
		\caption{Schema of Kursnamen relation with sample records}
		\label{schema:kursnamen}
	\end{table}
	%
	\begin{table}
		\centering
		\begin{tabular}{|c|c|c|c|c|c|c|c|}
			\hline
			& StudiengangsID & Hauptfach & Nebenfach & Semester & Prüfungsordnung & BeauftragterID & Aktiv \\\hline\hline
			Data Type & TEXT & TEXT & TEXT & INTEGER & INTEGER & TEXT & TEXT\\
			Constraints & NN, PK, U & NN & NN & NN & NN & NN & NN \\
			Checks & &  &  &  &  &  & \(\in\{Ja, Nein\}\) \\\hline
			& infmathbpo3 & Informatik & Mathematik & 1 & 3 & Lessmann & Nein\\
			& infmathbpo4 & Informatik & Mathematik & 1 & 4 & Lessmann & Ja\\\hline
		\end{tabular}
		\caption{Schema of Studiengänge relation with sample records}
		\label{schema:studiengänge}
	\end{table}
	%
	\begin{table}
		\centering
		\begin{tabular}{|c|c|c|c|c|c|c|}
			\hline
			& StudiengangsID & KursID & Modulname & Veranstaltungsteile & ECTS & Pflichtfach \\\hline\hline
			Data Type & TEXT & TEXT & TEXT & TEXT & INTEGER & TEXT \\
			Constraints & NN, PK, FK(Studiengänge) & NN, PK, FK(Kursnamen) & NN & NN & NN & NN \\
			Checks & &  &  &  &  & \(\in\{Ja, Nein\}\) \\\hline
			& infmathbpo4 & MOD & Modellierung & [1,2,3]  & 9 & Ja \\
			& infmathbpo4 & Prog & Programmierung & [1,2,3,4] & 4 & Ja \\\hline
		\end{tabular}
		\caption{Schema of Module relation with sample records}
		\label{schema:module}
	\end{table}
	%
	\begin{table}
		\centering
		\begin{tabular}{|c|c|c|c|}
			\hline
			& StudiengangsID & KursID & Druckrubrik \\\hline\hline
			Data Type & TEXT & TEXT & TEXT \\
			Constraints & NN, PK, FK(Studiengänge) & NN, PK, FK(Kursnamen) & NN, PK \\
			Checks & &  & \\\hline
			& infmathbpo4 & MOD & Informatik Bachelor - Pflichtteil (POv4)\\
			& infmathbpo4 & Prog & Informatik Bachelor - Pflichtteil (POv4) \\\hline
		\end{tabular}
		\caption{Schema of Druckrubriken relation with sample records}
		\label{schema:druckrubriken}
	\end{table}
	%
	\begin{table}
		\centering
		\begin{tabular}{|c|c|c|c|}
			\hline
			& Bezeichnung & Parent & Reihenfolge \\\hline\hline
			Data Type & TEXT & TEXT & INTEGER \\
			Constraints & NN, PK & PK & \\
			Default & & & 1 \\\hline
			& Informatik Bachelor (POv4) & & 1 \\
			& Informatik Bachelor - Pflichtteil (POv4) & Informatik Bachelor (POv4) & 1\\\hline
		\end{tabular}
		\caption{Schema of Druckrubrikbezeichnungen relation with sample records}
		\label{schema:druckrubrikbezeichnungen}
	\end{table}
	%
	\begin{table}
		\centering
		\tiny
		\begin{tabular}{|c|c|c|c|c|c|c|c|c|}
			\hline
			& ProfID & KursID & Veranstaltungsteil & Veranstaltungsart & Übungsanzahl &
			Wunschzeit & Sprache & Hoererzahl \\\hline\hline
			Data Type & TEXT & TEXT & INTEGER & TEXT & INTEGER & TEXT & TEXT & INTEGER \\
			Constraints & \multicolumn{1}{p{3cm}}{NN, PK, FK(Professor)} & \multicolumn{1}{p{3cm}}{NN, PK, FK(Veranstaltungen)} & \multicolumn{1}{p{3cm}}{NN, PK, FK(Veranstaltungen)} & NN & NN & & NN & \\
			Check & & & & \(\in\href{https://git.cs.upb.de/stulp.imt/database/blob/master/src/main/java/de/upb/stulp/database/enums/Veranstaltungsart.java}{Veranstaltungsart}\) & & & & \\\hline
			& N.N. & MOD & 3 & Übung & 2 & [5,9,12] & Deutsch & 20 \\
			& Scheidler & MOD & 1 & Vorlesung & 0 & [] & Deutsch & 50 \\\hline
		\end{tabular}
	\begin{tabular}{|c|c|c|c|c|c|c|c|c|}
		\hline
		& Zweiwoechentlich & Startwoche & Tafel & Beamer & Fenster & Ort & SonstigesRaumplanung & SonstigesZeitplanung \\\hline\hline
		& INTEGER & INTEGER & INTEGER & INTEGER & INTEGER & INTEGER & TEXT & TEXT\\
		& Default(0) & Default(1) & Default(0) & Default(0) & Default(0) & Default(3) & & \\
		& \(\in\{0,1\}\) & & \(\in\{0,1\}\) & \(\in\{0,1\}\) & \(\in\{0,1\}\) & \(\in\href{https://git.cs.upb.de/stulp.imt/database/blob/master/src/main/java/de/upb/stulp/database/enums/Ort.java}{Ort}\) & & \\\hline
		& 1 & 2 & 1 & 1 & 1 & 1 & Seminarraum & \\
		& 0 & 0 & 1 & 1 & 1 & 1 & & egal \\\hline
	\end{tabular}
		\caption{Schema of Vorlesungsplan and Alter\_Vorlesungsplan relation with sample records}
		\label{schema:vorlesungsplan}
	\end{table}
	%
	\begin{table}
		\centering
		\begin{tabular}{|c|c|c|c|c|}
			\hline
			& ProfID & Sperrtag & Sperrzeit & Dauer \\\hline\hline
			Data Type & TEXT & TEXT & TEXT & INTEGER \\
			Constraints & NN, PK, FK(Professor) & NN, PK & NN, PK & NN \\
			Check & & \(\in\href{https://git.cs.upb.de/stulp.imt/database/blob/master/src/main/java/de/upb/stulp/database/enums/Tag.java}{Tag}\) & & \\\hline
			& Scheidler & Montag & 8:00 & 6 \\
			& Scheidler & Mittwoch & 8:00 & 1 \\\hline
		\end{tabular}
		\caption{Schema of Sperrzeiten relation with sample records}
		\label{schema:sperrzeiten}
	\end{table}
	%
	\begin{table}
		\centering
		\begin{tabular}{|c|c|c|c|}
			\hline
			& ProfID & Name & Fakultät \\\hline\hline
			Data Type & TEXT & TEXT & TEXT \\
			Constraints & NN, PK& NN & NN \\
			Check & & & \\\hline
			& Scheidler & Prof. Dr. Christian Scheidler & EIM \\
			& Böttcher & Prof. Dr. Stefan Böttcher & EIM \\\hline
		\end{tabular}
		\caption{Schema of Professor relation with sample records}
		\label{schema:professor}
	\end{table}
\end{landscape}
\subsubsection{Implementing the interface}
The main interface for the TeX/PDF Backup is the
\href{https://git.cs.upb.de/stulp.imt/database/blob/master/src/main/java/de/upb/stulp/database/webService/teXPdfBackup/TeXPdfBackupDatabase.java}{TeXPdfBackupDatabase}.
To implement this main interface simply do the following and implement
the methods you need to overwrite:

\begin{lstlisting}[language=Java]
public class MyTeXPdfBackupDatabase 
				implements TeXPdfBackupDatabase {
...
}
\end{lstlisting}

An instance of this class will then be used in the Web Service to query
the database. Keep in mind that you also need to adjust the Web Service
to use your custom class.

When implementing MyTeXPdfBackupDatabase you will notice that you also
have to return an instance of a class implementing the
\href{https://git.cs.upb.de/stulp.imt/database/blob/master/src/main/java/de/upb/stulp/database/webService/teXPdfBackup/TeXPdfBackupRecordFactory.java}{TeXPdfBackupRecordFactory}
interface as this is the only option for users of your interface to
create objects relating to tables of the database. This is needed
because the user uses these objects to pass filter information to the
queries. Thus, you will need to create the following class:

\begin{lstlisting}[language=Java]
public class MyTeXdPdfBackupRecordFactory 
				implements TeXPdfBackupRecordFactory {
...
}
\end{lstlisting}

You will also notice that the database interface only offers one insert,
update, delete and getRecords operation, yet there can be different
tables in your database. This is due to the fact that these methods
simply delegate the operation to the different table implementations. To
take advantage of this the
\href{https://git.cs.upb.de/stulp.imt/database/blob/master/src/main/java/de/upb/stulp/database/Record.java}{Record}
interface has been introduced. This interface enables you to simply call
the corresponding method of the record interface for delegation.

\subsubsection{How are Records of database relations implemented in the interface}

Every relation in your database has a corresponding entity class that
stores exactly one record of a relation, e.g.~
\href{https://git.cs.upb.de/stulp.imt/database/blob/master/src/main/java/de/upb/stulp/database/webService/teXPdfBackup/TeXPdf.java}{TeXPdf}.
Thus, when querying the database, you wrap every record into its
corresponding interface class and return this.

\subsubsection{How to implement a record class}

To implement a record class, you simply inherit from the corresponding
interface class and override the missing methods. For example:

\begin{lstlisting}[language=Java]
public class MyTeXPdf extends TeXPdf {
/*
 * make sure to make the constructors protected 
 * such that the user is forced to use 
 * the RecordFactory for instantiation
 */
...
}
\end{lstlisting}

The interface will force you to implement how such a record should be
inserted, updated, deleted and retrieved from the database. However, it
will not force you to implement the creation of a possibly necessary
table since different databases use different concepts. Keep that in
mind and consult your database manual to see what you need. You also
need to overwrite the factory methods in MyTeXPdfBackupRecordFactory to
be able to instantiate your class from outside this package.

\subsubsection{How to extend the database interface}

There are two ways of extending the database interface that are quite
different in implementation effort. 1. You need to provide new
functionality that is not yet covered by the interface but the database
schema has not changed. 2. You need to change the database schema.

\subsubsection{Add new functionality without database schema change}

In order to add new functionality to the database that does not need a
schema change, you only need to add the required method to the
\href{https://git.cs.upb.de/stulp.imt/database/blob/master/src/main/java/de/upb/stulp/database/webService/teXPdfBackup/TeXPdfBackupDatabase.java}{TeXPdfBackupDatabase}
interface and add the implementation of this method to
MyTeXPdfBackupDatabase.

\subsubsection{Change the schema}

When changing the schema you have to differentiate between updating
already existent relations and adding new relations.

When updating existent relations you only need to update the interface
class that implements the Record interface, e.g.~TeXPdf, and your
implementation, e.g.~MyTeXPdf, depending on what you need to change.
When using the
\href{https://git.cs.upb.de/stulp.imt/database/blob/master/src/main/java/de/upb/stulp/database/impl/webService/teXPdfBackup/SQLiteTeXPdf.java}{SQLiteTeXPdf}
implementation that is provided in this repository, you will also need
to update the
\href{https://git.cs.upb.de/stulp.imt/database/blob/master/src/test/resources/tex_backup_schema.db}{tex\_backup\_schema.db}
as this implementation is using JOOQ and uses automatic code generation
to create the corresponding classes to enable type safe SQL queries.

When adding a new relation to the database you need to create the
following two classes that represent a record of the new relation:

\begin{lstlisting}[language=Java]
public abstract class NewRelation implements Record {
//private nullable field for every field of the relation. 
//protected constructors
//getter and setter
//overwrite equals and hashCode methods
}
\end{lstlisting}

\begin{lstlisting}[language=Java]
public class MyNewRelation extends NewRelation {
//implement methods
}
\end{lstlisting}

After creating these classes, you need to add the factory methods for
the relation records to
\href{https://git.cs.upb.de/stulp.imt/database/blob/master/src/main/java/de/upb/stulp/database/webService/teXPdfBackup/TeXPdfBackupRecordFactory.java}{TeXPdfBackupRecordFactory}.
This will look like this:

\begin{lstlisting}[language=Java]
public interface TeXPdfBackupRecordFactory {
//place holder for old interface methods
NewRelation createNewRelation();
NewRelation createNewRelation(/*list of all fields to set*/);
}
\end{lstlisting}

Then add the implementation again to MyTeXPdfBackupRecordFactory. After
that you have successfully added a relation to your database due to
delegation in the TeXPdfBackupDatabase. There is however one exception.
If you are using the SQLite interface provided by this repository, you
will again need to create a demo table in
\href{https://git.cs.upb.de/stulp.imt/database/blob/master/src/test/resources/tex_backup_schema.db}{tex\_backup\_schema.db}
to define the schema for JOOQ. You also need to make sure to create the
new table when instantiating the database.

%\subsection{Plan backup database interface}
The plan backup database stores uploaded copies of the planning tool
database with all the relevant information that is needed for the web
service to provide a basic version control system.
\subsubsection{Usage}
Usage (without try/catch for exception handling to support readability):

\begin{lstlisting}[language=Java]
/**Instantiation**/
SQLiteLocalProgramDatabase ldb = 
		new SQLiteLocalProgramDatabase(
			"path/to/local_program.db"
		);
SQLiteWebServiceDatabase wdb = 
		new SQLiteWebServiceDatabase(
			"path/to/web_service.db"
		);
SQLiteWebServiceDatabaseImport imp = 
		new SQLiteWebServiceDatabaseImport(ldb);
/**invoke import**/
ldb = imp.importData(wdb);
\end{lstlisting}
\subsubsection{Schema and sample data}
\begin{landscape}
	\begin{table}
		\centering
		\begin{tabular}{|c|c|c|c|c|c|}
			\hline
			& KursID & Veranstaltungsteil & Veranstaltungsart & Dauer & BemerkungPlanung \\\hline\hline
			Data Type & TEXT & INTEGER & TEXT & INTEGER & TEXT \\
			Constraints & NN, PK, FK(Kursnamen) & NN, PK & NN & NN & \\
			Checks & & & \(\in\href{https://git.cs.upb.de/stulp.imt/database/blob/master/src/main/java/de/upb/stulp/database/enums/Veranstaltungsart.java}{Veranstaltungsart}\) & & \\\hline
			& MOD & 1 & Vorlesung & 3 & mit Projektor\\
			& MOD & 2 & Übung & 2 &  \\\hline
		\end{tabular}
		\caption{Schema of Veranstaltungen relation with sample records}
		\label{schema:veranstaltungen}
	\end{table}
	%
	\begin{table}
		\centering
		\begin{tabular}{|c|c|c|c|c|c|c|c|}
			\hline
			& KursID & Kursname & KursnameEN & Fachbereich & BeauftragterID & PaulNr & VeranstalterGeneriert \\\hline\hline
			Data Type & TEXT & TEXT & TEXT & TEXT & TEXT & TEXT & INTEGER\\
			Constraints & NN, PK, U & NN & NN & NN & NN & NN & NN \\
			Checks &  &  &  &  &  &  & \(\in\{0,1\}\) \\\hline
			& MOD & Modellierung & Modelling & Informatik & Lessmann & L.0.231.1 & 0\\
			& Prog & Programmierung & Programming & Informatik & Lessmann & L.0.1.1 & 0\\\hline
		\end{tabular}
		\caption{Schema of Kursnamen relation with sample records}
		\label{schema:kursnamen}
	\end{table}
	%
	\begin{table}
		\centering
		\begin{tabular}{|c|c|c|c|c|c|c|c|}
			\hline
			& StudiengangsID & Hauptfach & Nebenfach & Semester & Prüfungsordnung & BeauftragterID & Aktiv \\\hline\hline
			Data Type & TEXT & TEXT & TEXT & INTEGER & INTEGER & TEXT & TEXT\\
			Constraints & NN, PK, U & NN & NN & NN & NN & NN & NN \\
			Checks & &  &  &  &  &  & \(\in\{Ja, Nein\}\) \\\hline
			& infmathbpo3 & Informatik & Mathematik & 1 & 3 & Lessmann & Nein\\
			& infmathbpo4 & Informatik & Mathematik & 1 & 4 & Lessmann & Ja\\\hline
		\end{tabular}
		\caption{Schema of Studiengänge relation with sample records}
		\label{schema:studiengänge}
	\end{table}
	%
	\begin{table}
		\centering
		\begin{tabular}{|c|c|c|c|c|c|c|}
			\hline
			& StudiengangsID & KursID & Modulname & Veranstaltungsteile & ECTS & Pflichtfach \\\hline\hline
			Data Type & TEXT & TEXT & TEXT & TEXT & INTEGER & TEXT \\
			Constraints & NN, PK, FK(Studiengänge) & NN, PK, FK(Kursnamen) & NN & NN & NN & NN \\
			Checks & &  &  &  &  & \(\in\{Ja, Nein\}\) \\\hline
			& infmathbpo4 & MOD & Modellierung & [1,2,3]  & 9 & Ja \\
			& infmathbpo4 & Prog & Programmierung & [1,2,3,4] & 4 & Ja \\\hline
		\end{tabular}
		\caption{Schema of Module relation with sample records}
		\label{schema:module}
	\end{table}
	%
	\begin{table}
		\centering
		\begin{tabular}{|c|c|c|c|}
			\hline
			& StudiengangsID & KursID & Druckrubrik \\\hline\hline
			Data Type & TEXT & TEXT & TEXT \\
			Constraints & NN, PK, FK(Studiengänge) & NN, PK, FK(Kursnamen) & NN, PK \\
			Checks & &  & \\\hline
			& infmathbpo4 & MOD & Informatik Bachelor - Pflichtteil (POv4)\\
			& infmathbpo4 & Prog & Informatik Bachelor - Pflichtteil (POv4) \\\hline
		\end{tabular}
		\caption{Schema of Druckrubriken relation with sample records}
		\label{schema:druckrubriken}
	\end{table}
	%
	\begin{table}
		\centering
		\begin{tabular}{|c|c|c|c|}
			\hline
			& Bezeichnung & Parent & Reihenfolge \\\hline\hline
			Data Type & TEXT & TEXT & INTEGER \\
			Constraints & NN, PK & PK & \\
			Default & & & 1 \\\hline
			& Informatik Bachelor (POv4) & & 1 \\
			& Informatik Bachelor - Pflichtteil (POv4) & Informatik Bachelor (POv4) & 1\\\hline
		\end{tabular}
		\caption{Schema of Druckrubrikbezeichnungen relation with sample records}
		\label{schema:druckrubrikbezeichnungen}
	\end{table}
	%
	\begin{table}
		\centering
		\tiny
		\begin{tabular}{|c|c|c|c|c|c|c|c|c|}
			\hline
			& ProfID & KursID & Veranstaltungsteil & Veranstaltungsart & Übungsanzahl &
			Wunschzeit & Sprache & Hoererzahl \\\hline\hline
			Data Type & TEXT & TEXT & INTEGER & TEXT & INTEGER & TEXT & TEXT & INTEGER \\
			Constraints & \multicolumn{1}{p{3cm}}{NN, PK, FK(Professor)} & \multicolumn{1}{p{3cm}}{NN, PK, FK(Veranstaltungen)} & \multicolumn{1}{p{3cm}}{NN, PK, FK(Veranstaltungen)} & NN & NN & & NN & \\
			Check & & & & \(\in\href{https://git.cs.upb.de/stulp.imt/database/blob/master/src/main/java/de/upb/stulp/database/enums/Veranstaltungsart.java}{Veranstaltungsart}\) & & & & \\\hline
			& N.N. & MOD & 3 & Übung & 2 & [5,9,12] & Deutsch & 20 \\
			& Scheidler & MOD & 1 & Vorlesung & 0 & [] & Deutsch & 50 \\\hline
		\end{tabular}
	\begin{tabular}{|c|c|c|c|c|c|c|c|c|}
		\hline
		& Zweiwoechentlich & Startwoche & Tafel & Beamer & Fenster & Ort & SonstigesRaumplanung & SonstigesZeitplanung \\\hline\hline
		& INTEGER & INTEGER & INTEGER & INTEGER & INTEGER & INTEGER & TEXT & TEXT\\
		& Default(0) & Default(1) & Default(0) & Default(0) & Default(0) & Default(3) & & \\
		& \(\in\{0,1\}\) & & \(\in\{0,1\}\) & \(\in\{0,1\}\) & \(\in\{0,1\}\) & \(\in\href{https://git.cs.upb.de/stulp.imt/database/blob/master/src/main/java/de/upb/stulp/database/enums/Ort.java}{Ort}\) & & \\\hline
		& 1 & 2 & 1 & 1 & 1 & 1 & Seminarraum & \\
		& 0 & 0 & 1 & 1 & 1 & 1 & & egal \\\hline
	\end{tabular}
		\caption{Schema of Vorlesungsplan and Alter\_Vorlesungsplan relation with sample records}
		\label{schema:vorlesungsplan}
	\end{table}
	%
	\begin{table}
		\centering
		\begin{tabular}{|c|c|c|c|c|}
			\hline
			& ProfID & Sperrtag & Sperrzeit & Dauer \\\hline\hline
			Data Type & TEXT & TEXT & TEXT & INTEGER \\
			Constraints & NN, PK, FK(Professor) & NN, PK & NN, PK & NN \\
			Check & & \(\in\href{https://git.cs.upb.de/stulp.imt/database/blob/master/src/main/java/de/upb/stulp/database/enums/Tag.java}{Tag}\) & & \\\hline
			& Scheidler & Montag & 8:00 & 6 \\
			& Scheidler & Mittwoch & 8:00 & 1 \\\hline
		\end{tabular}
		\caption{Schema of Sperrzeiten relation with sample records}
		\label{schema:sperrzeiten}
	\end{table}
	%
	\begin{table}
		\centering
		\begin{tabular}{|c|c|c|c|}
			\hline
			& ProfID & Name & Fakultät \\\hline\hline
			Data Type & TEXT & TEXT & TEXT \\
			Constraints & NN, PK& NN & NN \\
			Check & & & \\\hline
			& Scheidler & Prof. Dr. Christian Scheidler & EIM \\
			& Böttcher & Prof. Dr. Stefan Böttcher & EIM \\\hline
		\end{tabular}
		\caption{Schema of Professor relation with sample records}
		\label{schema:professor}
	\end{table}
\end{landscape}
\subsubsection{Implementing the interface}
The main interface for the TeX/PDF Backup is the
\href{https://git.cs.upb.de/stulp.imt/database/blob/master/src/main/java/de/upb/stulp/database/webService/teXPdfBackup/TeXPdfBackupDatabase.java}{TeXPdfBackupDatabase}.
To implement this main interface simply do the following and implement
the methods you need to overwrite:

\begin{lstlisting}[language=Java]
public class MyTeXPdfBackupDatabase 
				implements TeXPdfBackupDatabase {
...
}
\end{lstlisting}

An instance of this class will then be used in the Web Service to query
the database. Keep in mind that you also need to adjust the Web Service
to use your custom class.

When implementing MyTeXPdfBackupDatabase you will notice that you also
have to return an instance of a class implementing the
\href{https://git.cs.upb.de/stulp.imt/database/blob/master/src/main/java/de/upb/stulp/database/webService/teXPdfBackup/TeXPdfBackupRecordFactory.java}{TeXPdfBackupRecordFactory}
interface as this is the only option for users of your interface to
create objects relating to tables of the database. This is needed
because the user uses these objects to pass filter information to the
queries. Thus, you will need to create the following class:

\begin{lstlisting}[language=Java]
public class MyTeXdPdfBackupRecordFactory 
				implements TeXPdfBackupRecordFactory {
...
}
\end{lstlisting}

You will also notice that the database interface only offers one insert,
update, delete and getRecords operation, yet there can be different
tables in your database. This is due to the fact that these methods
simply delegate the operation to the different table implementations. To
take advantage of this the
\href{https://git.cs.upb.de/stulp.imt/database/blob/master/src/main/java/de/upb/stulp/database/Record.java}{Record}
interface has been introduced. This interface enables you to simply call
the corresponding method of the record interface for delegation.

\subsubsection{How are Records of database relations implemented in the interface}

Every relation in your database has a corresponding entity class that
stores exactly one record of a relation, e.g.~
\href{https://git.cs.upb.de/stulp.imt/database/blob/master/src/main/java/de/upb/stulp/database/webService/teXPdfBackup/TeXPdf.java}{TeXPdf}.
Thus, when querying the database, you wrap every record into its
corresponding interface class and return this.

\subsubsection{How to implement a record class}

To implement a record class, you simply inherit from the corresponding
interface class and override the missing methods. For example:

\begin{lstlisting}[language=Java]
public class MyTeXPdf extends TeXPdf {
/*
 * make sure to make the constructors protected 
 * such that the user is forced to use 
 * the RecordFactory for instantiation
 */
...
}
\end{lstlisting}

The interface will force you to implement how such a record should be
inserted, updated, deleted and retrieved from the database. However, it
will not force you to implement the creation of a possibly necessary
table since different databases use different concepts. Keep that in
mind and consult your database manual to see what you need. You also
need to overwrite the factory methods in MyTeXPdfBackupRecordFactory to
be able to instantiate your class from outside this package.

\subsubsection{How to extend the database interface}

There are two ways of extending the database interface that are quite
different in implementation effort. 1. You need to provide new
functionality that is not yet covered by the interface but the database
schema has not changed. 2. You need to change the database schema.

\subsubsection{Add new functionality without database schema change}

In order to add new functionality to the database that does not need a
schema change, you only need to add the required method to the
\href{https://git.cs.upb.de/stulp.imt/database/blob/master/src/main/java/de/upb/stulp/database/webService/teXPdfBackup/TeXPdfBackupDatabase.java}{TeXPdfBackupDatabase}
interface and add the implementation of this method to
MyTeXPdfBackupDatabase.

\subsubsection{Change the schema}

When changing the schema you have to differentiate between updating
already existent relations and adding new relations.

When updating existent relations you only need to update the interface
class that implements the Record interface, e.g.~TeXPdf, and your
implementation, e.g.~MyTeXPdf, depending on what you need to change.
When using the
\href{https://git.cs.upb.de/stulp.imt/database/blob/master/src/main/java/de/upb/stulp/database/impl/webService/teXPdfBackup/SQLiteTeXPdf.java}{SQLiteTeXPdf}
implementation that is provided in this repository, you will also need
to update the
\href{https://git.cs.upb.de/stulp.imt/database/blob/master/src/test/resources/tex_backup_schema.db}{tex\_backup\_schema.db}
as this implementation is using JOOQ and uses automatic code generation
to create the corresponding classes to enable type safe SQL queries.

When adding a new relation to the database you need to create the
following two classes that represent a record of the new relation:

\begin{lstlisting}[language=Java]
public abstract class NewRelation implements Record {
//private nullable field for every field of the relation. 
//protected constructors
//getter and setter
//overwrite equals and hashCode methods
}
\end{lstlisting}

\begin{lstlisting}[language=Java]
public class MyNewRelation extends NewRelation {
//implement methods
}
\end{lstlisting}

After creating these classes, you need to add the factory methods for
the relation records to
\href{https://git.cs.upb.de/stulp.imt/database/blob/master/src/main/java/de/upb/stulp/database/webService/teXPdfBackup/TeXPdfBackupRecordFactory.java}{TeXPdfBackupRecordFactory}.
This will look like this:

\begin{lstlisting}[language=Java]
public interface TeXPdfBackupRecordFactory {
//place holder for old interface methods
NewRelation createNewRelation();
NewRelation createNewRelation(/*list of all fields to set*/);
}
\end{lstlisting}

Then add the implementation again to MyTeXPdfBackupRecordFactory. After
that you have successfully added a relation to your database due to
delegation in the TeXPdfBackupDatabase. There is however one exception.
If you are using the SQLite interface provided by this repository, you
will again need to create a demo table in
\href{https://git.cs.upb.de/stulp.imt/database/blob/master/src/test/resources/tex_backup_schema.db}{tex\_backup\_schema.db}
to define the schema for JOOQ. You also need to make sure to create the
new table when instantiating the database.

%\subsection{TeX/PDF backup database interface}
The teX/PDF backup database stores uploaded PDFs and generated LaTeX
archives such that the web service has access to a history of different
files. This enables the web service to offer the user a selection of
files to download from.
\subsubsection{Usage}
Usage (without try/catch for exception handling to support readability):

\begin{lstlisting}[language=Java]
/**Instantiation**/
SQLiteLocalProgramDatabase ldb = 
		new SQLiteLocalProgramDatabase(
			"path/to/local_program.db"
		);
SQLiteWebServiceDatabase wdb = 
		new SQLiteWebServiceDatabase(
			"path/to/web_service.db"
		);
SQLiteWebServiceDatabaseImport imp = 
		new SQLiteWebServiceDatabaseImport(ldb);
/**invoke import**/
ldb = imp.importData(wdb);
\end{lstlisting}
\subsubsection{Schema and sample data}
\begin{landscape}
	\begin{table}
		\centering
		\begin{tabular}{|c|c|c|c|c|c|}
			\hline
			& KursID & Veranstaltungsteil & Veranstaltungsart & Dauer & BemerkungPlanung \\\hline\hline
			Data Type & TEXT & INTEGER & TEXT & INTEGER & TEXT \\
			Constraints & NN, PK, FK(Kursnamen) & NN, PK & NN & NN & \\
			Checks & & & \(\in\href{https://git.cs.upb.de/stulp.imt/database/blob/master/src/main/java/de/upb/stulp/database/enums/Veranstaltungsart.java}{Veranstaltungsart}\) & & \\\hline
			& MOD & 1 & Vorlesung & 3 & mit Projektor\\
			& MOD & 2 & Übung & 2 &  \\\hline
		\end{tabular}
		\caption{Schema of Veranstaltungen relation with sample records}
		\label{schema:veranstaltungen}
	\end{table}
	%
	\begin{table}
		\centering
		\begin{tabular}{|c|c|c|c|c|c|c|c|}
			\hline
			& KursID & Kursname & KursnameEN & Fachbereich & BeauftragterID & PaulNr & VeranstalterGeneriert \\\hline\hline
			Data Type & TEXT & TEXT & TEXT & TEXT & TEXT & TEXT & INTEGER\\
			Constraints & NN, PK, U & NN & NN & NN & NN & NN & NN \\
			Checks &  &  &  &  &  &  & \(\in\{0,1\}\) \\\hline
			& MOD & Modellierung & Modelling & Informatik & Lessmann & L.0.231.1 & 0\\
			& Prog & Programmierung & Programming & Informatik & Lessmann & L.0.1.1 & 0\\\hline
		\end{tabular}
		\caption{Schema of Kursnamen relation with sample records}
		\label{schema:kursnamen}
	\end{table}
	%
	\begin{table}
		\centering
		\begin{tabular}{|c|c|c|c|c|c|c|c|}
			\hline
			& StudiengangsID & Hauptfach & Nebenfach & Semester & Prüfungsordnung & BeauftragterID & Aktiv \\\hline\hline
			Data Type & TEXT & TEXT & TEXT & INTEGER & INTEGER & TEXT & TEXT\\
			Constraints & NN, PK, U & NN & NN & NN & NN & NN & NN \\
			Checks & &  &  &  &  &  & \(\in\{Ja, Nein\}\) \\\hline
			& infmathbpo3 & Informatik & Mathematik & 1 & 3 & Lessmann & Nein\\
			& infmathbpo4 & Informatik & Mathematik & 1 & 4 & Lessmann & Ja\\\hline
		\end{tabular}
		\caption{Schema of Studiengänge relation with sample records}
		\label{schema:studiengänge}
	\end{table}
	%
	\begin{table}
		\centering
		\begin{tabular}{|c|c|c|c|c|c|c|}
			\hline
			& StudiengangsID & KursID & Modulname & Veranstaltungsteile & ECTS & Pflichtfach \\\hline\hline
			Data Type & TEXT & TEXT & TEXT & TEXT & INTEGER & TEXT \\
			Constraints & NN, PK, FK(Studiengänge) & NN, PK, FK(Kursnamen) & NN & NN & NN & NN \\
			Checks & &  &  &  &  & \(\in\{Ja, Nein\}\) \\\hline
			& infmathbpo4 & MOD & Modellierung & [1,2,3]  & 9 & Ja \\
			& infmathbpo4 & Prog & Programmierung & [1,2,3,4] & 4 & Ja \\\hline
		\end{tabular}
		\caption{Schema of Module relation with sample records}
		\label{schema:module}
	\end{table}
	%
	\begin{table}
		\centering
		\begin{tabular}{|c|c|c|c|}
			\hline
			& StudiengangsID & KursID & Druckrubrik \\\hline\hline
			Data Type & TEXT & TEXT & TEXT \\
			Constraints & NN, PK, FK(Studiengänge) & NN, PK, FK(Kursnamen) & NN, PK \\
			Checks & &  & \\\hline
			& infmathbpo4 & MOD & Informatik Bachelor - Pflichtteil (POv4)\\
			& infmathbpo4 & Prog & Informatik Bachelor - Pflichtteil (POv4) \\\hline
		\end{tabular}
		\caption{Schema of Druckrubriken relation with sample records}
		\label{schema:druckrubriken}
	\end{table}
	%
	\begin{table}
		\centering
		\begin{tabular}{|c|c|c|c|}
			\hline
			& Bezeichnung & Parent & Reihenfolge \\\hline\hline
			Data Type & TEXT & TEXT & INTEGER \\
			Constraints & NN, PK & PK & \\
			Default & & & 1 \\\hline
			& Informatik Bachelor (POv4) & & 1 \\
			& Informatik Bachelor - Pflichtteil (POv4) & Informatik Bachelor (POv4) & 1\\\hline
		\end{tabular}
		\caption{Schema of Druckrubrikbezeichnungen relation with sample records}
		\label{schema:druckrubrikbezeichnungen}
	\end{table}
	%
	\begin{table}
		\centering
		\tiny
		\begin{tabular}{|c|c|c|c|c|c|c|c|c|}
			\hline
			& ProfID & KursID & Veranstaltungsteil & Veranstaltungsart & Übungsanzahl &
			Wunschzeit & Sprache & Hoererzahl \\\hline\hline
			Data Type & TEXT & TEXT & INTEGER & TEXT & INTEGER & TEXT & TEXT & INTEGER \\
			Constraints & \multicolumn{1}{p{3cm}}{NN, PK, FK(Professor)} & \multicolumn{1}{p{3cm}}{NN, PK, FK(Veranstaltungen)} & \multicolumn{1}{p{3cm}}{NN, PK, FK(Veranstaltungen)} & NN & NN & & NN & \\
			Check & & & & \(\in\href{https://git.cs.upb.de/stulp.imt/database/blob/master/src/main/java/de/upb/stulp/database/enums/Veranstaltungsart.java}{Veranstaltungsart}\) & & & & \\\hline
			& N.N. & MOD & 3 & Übung & 2 & [5,9,12] & Deutsch & 20 \\
			& Scheidler & MOD & 1 & Vorlesung & 0 & [] & Deutsch & 50 \\\hline
		\end{tabular}
	\begin{tabular}{|c|c|c|c|c|c|c|c|c|}
		\hline
		& Zweiwoechentlich & Startwoche & Tafel & Beamer & Fenster & Ort & SonstigesRaumplanung & SonstigesZeitplanung \\\hline\hline
		& INTEGER & INTEGER & INTEGER & INTEGER & INTEGER & INTEGER & TEXT & TEXT\\
		& Default(0) & Default(1) & Default(0) & Default(0) & Default(0) & Default(3) & & \\
		& \(\in\{0,1\}\) & & \(\in\{0,1\}\) & \(\in\{0,1\}\) & \(\in\{0,1\}\) & \(\in\href{https://git.cs.upb.de/stulp.imt/database/blob/master/src/main/java/de/upb/stulp/database/enums/Ort.java}{Ort}\) & & \\\hline
		& 1 & 2 & 1 & 1 & 1 & 1 & Seminarraum & \\
		& 0 & 0 & 1 & 1 & 1 & 1 & & egal \\\hline
	\end{tabular}
		\caption{Schema of Vorlesungsplan and Alter\_Vorlesungsplan relation with sample records}
		\label{schema:vorlesungsplan}
	\end{table}
	%
	\begin{table}
		\centering
		\begin{tabular}{|c|c|c|c|c|}
			\hline
			& ProfID & Sperrtag & Sperrzeit & Dauer \\\hline\hline
			Data Type & TEXT & TEXT & TEXT & INTEGER \\
			Constraints & NN, PK, FK(Professor) & NN, PK & NN, PK & NN \\
			Check & & \(\in\href{https://git.cs.upb.de/stulp.imt/database/blob/master/src/main/java/de/upb/stulp/database/enums/Tag.java}{Tag}\) & & \\\hline
			& Scheidler & Montag & 8:00 & 6 \\
			& Scheidler & Mittwoch & 8:00 & 1 \\\hline
		\end{tabular}
		\caption{Schema of Sperrzeiten relation with sample records}
		\label{schema:sperrzeiten}
	\end{table}
	%
	\begin{table}
		\centering
		\begin{tabular}{|c|c|c|c|}
			\hline
			& ProfID & Name & Fakultät \\\hline\hline
			Data Type & TEXT & TEXT & TEXT \\
			Constraints & NN, PK& NN & NN \\
			Check & & & \\\hline
			& Scheidler & Prof. Dr. Christian Scheidler & EIM \\
			& Böttcher & Prof. Dr. Stefan Böttcher & EIM \\\hline
		\end{tabular}
		\caption{Schema of Professor relation with sample records}
		\label{schema:professor}
	\end{table}
\end{landscape}
\subsubsection{Implementing the interface}
The main interface for the TeX/PDF Backup is the
\href{https://git.cs.upb.de/stulp.imt/database/blob/master/src/main/java/de/upb/stulp/database/webService/teXPdfBackup/TeXPdfBackupDatabase.java}{TeXPdfBackupDatabase}.
To implement this main interface simply do the following and implement
the methods you need to overwrite:

\begin{lstlisting}[language=Java]
public class MyTeXPdfBackupDatabase 
				implements TeXPdfBackupDatabase {
...
}
\end{lstlisting}

An instance of this class will then be used in the Web Service to query
the database. Keep in mind that you also need to adjust the Web Service
to use your custom class.

When implementing MyTeXPdfBackupDatabase you will notice that you also
have to return an instance of a class implementing the
\href{https://git.cs.upb.de/stulp.imt/database/blob/master/src/main/java/de/upb/stulp/database/webService/teXPdfBackup/TeXPdfBackupRecordFactory.java}{TeXPdfBackupRecordFactory}
interface as this is the only option for users of your interface to
create objects relating to tables of the database. This is needed
because the user uses these objects to pass filter information to the
queries. Thus, you will need to create the following class:

\begin{lstlisting}[language=Java]
public class MyTeXdPdfBackupRecordFactory 
				implements TeXPdfBackupRecordFactory {
...
}
\end{lstlisting}

You will also notice that the database interface only offers one insert,
update, delete and getRecords operation, yet there can be different
tables in your database. This is due to the fact that these methods
simply delegate the operation to the different table implementations. To
take advantage of this the
\href{https://git.cs.upb.de/stulp.imt/database/blob/master/src/main/java/de/upb/stulp/database/Record.java}{Record}
interface has been introduced. This interface enables you to simply call
the corresponding method of the record interface for delegation.

\subsubsection{How are Records of database relations implemented in the interface}

Every relation in your database has a corresponding entity class that
stores exactly one record of a relation, e.g.~
\href{https://git.cs.upb.de/stulp.imt/database/blob/master/src/main/java/de/upb/stulp/database/webService/teXPdfBackup/TeXPdf.java}{TeXPdf}.
Thus, when querying the database, you wrap every record into its
corresponding interface class and return this.

\subsubsection{How to implement a record class}

To implement a record class, you simply inherit from the corresponding
interface class and override the missing methods. For example:

\begin{lstlisting}[language=Java]
public class MyTeXPdf extends TeXPdf {
/*
 * make sure to make the constructors protected 
 * such that the user is forced to use 
 * the RecordFactory for instantiation
 */
...
}
\end{lstlisting}

The interface will force you to implement how such a record should be
inserted, updated, deleted and retrieved from the database. However, it
will not force you to implement the creation of a possibly necessary
table since different databases use different concepts. Keep that in
mind and consult your database manual to see what you need. You also
need to overwrite the factory methods in MyTeXPdfBackupRecordFactory to
be able to instantiate your class from outside this package.

\subsubsection{How to extend the database interface}

There are two ways of extending the database interface that are quite
different in implementation effort. 1. You need to provide new
functionality that is not yet covered by the interface but the database
schema has not changed. 2. You need to change the database schema.

\subsubsection{Add new functionality without database schema change}

In order to add new functionality to the database that does not need a
schema change, you only need to add the required method to the
\href{https://git.cs.upb.de/stulp.imt/database/blob/master/src/main/java/de/upb/stulp/database/webService/teXPdfBackup/TeXPdfBackupDatabase.java}{TeXPdfBackupDatabase}
interface and add the implementation of this method to
MyTeXPdfBackupDatabase.

\subsubsection{Change the schema}

When changing the schema you have to differentiate between updating
already existent relations and adding new relations.

When updating existent relations you only need to update the interface
class that implements the Record interface, e.g.~TeXPdf, and your
implementation, e.g.~MyTeXPdf, depending on what you need to change.
When using the
\href{https://git.cs.upb.de/stulp.imt/database/blob/master/src/main/java/de/upb/stulp/database/impl/webService/teXPdfBackup/SQLiteTeXPdf.java}{SQLiteTeXPdf}
implementation that is provided in this repository, you will also need
to update the
\href{https://git.cs.upb.de/stulp.imt/database/blob/master/src/test/resources/tex_backup_schema.db}{tex\_backup\_schema.db}
as this implementation is using JOOQ and uses automatic code generation
to create the corresponding classes to enable type safe SQL queries.

When adding a new relation to the database you need to create the
following two classes that represent a record of the new relation:

\begin{lstlisting}[language=Java]
public abstract class NewRelation implements Record {
//private nullable field for every field of the relation. 
//protected constructors
//getter and setter
//overwrite equals and hashCode methods
}
\end{lstlisting}

\begin{lstlisting}[language=Java]
public class MyNewRelation extends NewRelation {
//implement methods
}
\end{lstlisting}

After creating these classes, you need to add the factory methods for
the relation records to
\href{https://git.cs.upb.de/stulp.imt/database/blob/master/src/main/java/de/upb/stulp/database/webService/teXPdfBackup/TeXPdfBackupRecordFactory.java}{TeXPdfBackupRecordFactory}.
This will look like this:

\begin{lstlisting}[language=Java]
public interface TeXPdfBackupRecordFactory {
//place holder for old interface methods
NewRelation createNewRelation();
NewRelation createNewRelation(/*list of all fields to set*/);
}
\end{lstlisting}

Then add the implementation again to MyTeXPdfBackupRecordFactory. After
that you have successfully added a relation to your database due to
delegation in the TeXPdfBackupDatabase. There is however one exception.
If you are using the SQLite interface provided by this repository, you
will again need to create a demo table in
\href{https://git.cs.upb.de/stulp.imt/database/blob/master/src/test/resources/tex_backup_schema.db}{tex\_backup\_schema.db}
to define the schema for JOOQ. You also need to make sure to create the
new table when instantiating the database.

%\subsection{Web service database import interface}
This interface offers functionality to import the web service database
into the local program database. Note that this step is only to be
executed once for each semester since it creates the initial local
program database. 
\subsubsection{Usage}
Usage (without try/catch for exception handling to support readability):

\begin{lstlisting}[language=Java]
/**Instantiation**/
SQLiteLocalProgramDatabase ldb = 
		new SQLiteLocalProgramDatabase(
			"path/to/local_program.db"
		);
SQLiteWebServiceDatabase wdb = 
		new SQLiteWebServiceDatabase(
			"path/to/web_service.db"
		);
SQLiteWebServiceDatabaseImport imp = 
		new SQLiteWebServiceDatabaseImport(ldb);
/**invoke import**/
ldb = imp.importData(wdb);
\end{lstlisting}
\subsubsection{Implementing the interface}
The main interface for the TeX/PDF Backup is the
\href{https://git.cs.upb.de/stulp.imt/database/blob/master/src/main/java/de/upb/stulp/database/webService/teXPdfBackup/TeXPdfBackupDatabase.java}{TeXPdfBackupDatabase}.
To implement this main interface simply do the following and implement
the methods you need to overwrite:

\begin{lstlisting}[language=Java]
public class MyTeXPdfBackupDatabase 
				implements TeXPdfBackupDatabase {
...
}
\end{lstlisting}

An instance of this class will then be used in the Web Service to query
the database. Keep in mind that you also need to adjust the Web Service
to use your custom class.

When implementing MyTeXPdfBackupDatabase you will notice that you also
have to return an instance of a class implementing the
\href{https://git.cs.upb.de/stulp.imt/database/blob/master/src/main/java/de/upb/stulp/database/webService/teXPdfBackup/TeXPdfBackupRecordFactory.java}{TeXPdfBackupRecordFactory}
interface as this is the only option for users of your interface to
create objects relating to tables of the database. This is needed
because the user uses these objects to pass filter information to the
queries. Thus, you will need to create the following class:

\begin{lstlisting}[language=Java]
public class MyTeXdPdfBackupRecordFactory 
				implements TeXPdfBackupRecordFactory {
...
}
\end{lstlisting}

You will also notice that the database interface only offers one insert,
update, delete and getRecords operation, yet there can be different
tables in your database. This is due to the fact that these methods
simply delegate the operation to the different table implementations. To
take advantage of this the
\href{https://git.cs.upb.de/stulp.imt/database/blob/master/src/main/java/de/upb/stulp/database/Record.java}{Record}
interface has been introduced. This interface enables you to simply call
the corresponding method of the record interface for delegation.

\subsubsection{How are Records of database relations implemented in the interface}

Every relation in your database has a corresponding entity class that
stores exactly one record of a relation, e.g.~
\href{https://git.cs.upb.de/stulp.imt/database/blob/master/src/main/java/de/upb/stulp/database/webService/teXPdfBackup/TeXPdf.java}{TeXPdf}.
Thus, when querying the database, you wrap every record into its
corresponding interface class and return this.

\subsubsection{How to implement a record class}

To implement a record class, you simply inherit from the corresponding
interface class and override the missing methods. For example:

\begin{lstlisting}[language=Java]
public class MyTeXPdf extends TeXPdf {
/*
 * make sure to make the constructors protected 
 * such that the user is forced to use 
 * the RecordFactory for instantiation
 */
...
}
\end{lstlisting}

The interface will force you to implement how such a record should be
inserted, updated, deleted and retrieved from the database. However, it
will not force you to implement the creation of a possibly necessary
table since different databases use different concepts. Keep that in
mind and consult your database manual to see what you need. You also
need to overwrite the factory methods in MyTeXPdfBackupRecordFactory to
be able to instantiate your class from outside this package.

\subsubsection{How to extend the database interface}

There are two ways of extending the database interface that are quite
different in implementation effort. 1. You need to provide new
functionality that is not yet covered by the interface but the database
schema has not changed. 2. You need to change the database schema.

\subsubsection{Add new functionality without database schema change}

In order to add new functionality to the database that does not need a
schema change, you only need to add the required method to the
\href{https://git.cs.upb.de/stulp.imt/database/blob/master/src/main/java/de/upb/stulp/database/webService/teXPdfBackup/TeXPdfBackupDatabase.java}{TeXPdfBackupDatabase}
interface and add the implementation of this method to
MyTeXPdfBackupDatabase.

\subsubsection{Change the schema}

When changing the schema you have to differentiate between updating
already existent relations and adding new relations.

When updating existent relations you only need to update the interface
class that implements the Record interface, e.g.~TeXPdf, and your
implementation, e.g.~MyTeXPdf, depending on what you need to change.
When using the
\href{https://git.cs.upb.de/stulp.imt/database/blob/master/src/main/java/de/upb/stulp/database/impl/webService/teXPdfBackup/SQLiteTeXPdf.java}{SQLiteTeXPdf}
implementation that is provided in this repository, you will also need
to update the
\href{https://git.cs.upb.de/stulp.imt/database/blob/master/src/test/resources/tex_backup_schema.db}{tex\_backup\_schema.db}
as this implementation is using JOOQ and uses automatic code generation
to create the corresponding classes to enable type safe SQL queries.

When adding a new relation to the database you need to create the
following two classes that represent a record of the new relation:

\begin{lstlisting}[language=Java]
public abstract class NewRelation implements Record {
//private nullable field for every field of the relation. 
//protected constructors
//getter and setter
//overwrite equals and hashCode methods
}
\end{lstlisting}

\begin{lstlisting}[language=Java]
public class MyNewRelation extends NewRelation {
//implement methods
}
\end{lstlisting}

After creating these classes, you need to add the factory methods for
the relation records to
\href{https://git.cs.upb.de/stulp.imt/database/blob/master/src/main/java/de/upb/stulp/database/webService/teXPdfBackup/TeXPdfBackupRecordFactory.java}{TeXPdfBackupRecordFactory}.
This will look like this:

\begin{lstlisting}[language=Java]
public interface TeXPdfBackupRecordFactory {
//place holder for old interface methods
NewRelation createNewRelation();
NewRelation createNewRelation(/*list of all fields to set*/);
}
\end{lstlisting}

Then add the implementation again to MyTeXPdfBackupRecordFactory. After
that you have successfully added a relation to your database due to
delegation in the TeXPdfBackupDatabase. There is however one exception.
If you are using the SQLite interface provided by this repository, you
will again need to create a demo table in
\href{https://git.cs.upb.de/stulp.imt/database/blob/master/src/test/resources/tex_backup_schema.db}{tex\_backup\_schema.db}
to define the schema for JOOQ. You also need to make sure to create the
new table when instantiating the database.

\subsection{Local Program Database Interface}
The local program database provides the \href{https://git.cs.uni-paderborn.de/stulp/sven_database/-/blob/main/src/main/java/de/upb/stulp/database/localProgram/LocalProgramDatabase.java}{LocalProgramDatabase} interface which offers all necessary database functionality for the local program. This section describes how to use and create instances of the interface.
\paragraph{Local program database interface}
\label{subsec:local-program-db}
The \href{https://git.cs.uni-paderborn.de/stulp/sven_database/-/blob/main/src/main/java/de/upb/stulp/database/localProgram/LocalProgramDatabase.java}{LocalProgramDatabase} interface offers methods for predefined complex querying, inserting, and manipulating data in the database, used by the local program planning tool. The database contains planning data expressing the schedule of course parts. The schedule assigns each part of a course to a time slot, room, and lecturer. The database scheme is an extended version of the one in Monika Werner's bachelor thesis.

\paragraph{Usage of the interface}
Usage (without try/catch for exception handling to support readability):

\begin{lstlisting}[language=Java]
/**Instantiation**/
//exchange SQLiteLocalProgramDatabase with your custom 
//implementation if needed
SQLiteLocalProgramDatabase sqldb = 
	new SQLiteLocalProgramDatabase(
		"path/to/local_program.db");
/**Create a record**/
Professor prof = db.getRecordFactory().createProfessor();
/**Query for all professors in the database**/
List<Professor> allProfs = db.getRecords(prof);
/**Insert professor into database**/
prof = db.getRecordFactory()
.createProfessor(/*professor data here*/);
db.insert(prof);
\end{lstlisting}

\paragraph{Schema and sample data}\label{sec:databasescheme}
\begin{landscape}
	\begin{table}
		\centering
		\begin{tabular}{|c|c|c|c|}
			\hline
			             & ProfID & Name & Fakultaet \\\hline\hline
			Data Type & TEXT & TEXT & TEXT \\
			Constraints & NN, PK & NN & NN \\
			Checks & & & \\\hline
			& ProDr.SteBöt & Prof. Dr. Stefan Böttcher & EIM \\
			& ProDr.ChrSch & Prof. Dr. Christian Scheideler & EIM \\\hline
		\end{tabular}
		\caption{Schema of Professor relation with sample records}
		\label{schema:professor}
	\end{table}
	%
	\begin{table}
		\centering
		\begin{tabular}{|c|c|c|c|c|}
			\hline
			& Raumnr & Gebaeude & Sitzplatzzahl & Prioritaet \\\hline\hline
			Data Type & TEXT & TEXT & INTEGER & INTEGER \\
			Constraints & NN, PK, U & NN & NN & NN \\
			Checks & & & & \(\in\{0,1\}\) \\\hline
			& O 2 & O & 117 & 1 \\
			& E 1 111 & E & 30 & 0 \\\hline
		\end{tabular}
		\caption{Schema of Raum relation with sample records}
		\label{schema:raum}
	\end{table}
	%
	\begin{table}
		\centering
		\begin{tabular}{|c|c|c|c|c|}
			\hline
			& KursID & Kursname & KursnameEN & Sprache \\\hline\hline
			Data Type & TEXT & TEXT & TEXT & Text \\
			Constraints & NN, PK, U & NN & NN & NN \\
			Checks & & & & \\\hline
			& MOD-1 & Modellierung & Foundations of Modelling & Deutsch \\
			& ML2-1 & Machine Learning II (in English) & Machine Learning II (in English) & Englisch \\\hline
		\end{tabular}
		\caption{Schema of Kurs relation with sample records}
		\label{schema:kurs}
	\end{table}
	%
	\begin{table}
		\centering
		\begin{tabular}{|c|c|c|c|c|c|}
			\hline
			& StudiengangsID & Hauptfach & Nebenfach & Pruefungsordnung & Semester \\\hline\hline
			Data Type & TEXT & TEXT & TEXT & Text & INTEGER \\
			Constraints & NN, PK, U & NN & NN & NN & NN\\
			Checks & & & & & \\\hline
			& InfM-1BPO4 & Informatik Bachelor & Mathematik & 4 & 1 \\
			& InfoOhne-2MPO3 & Informatik Master & Ohne & 3 & 2 \\\hline
		\end{tabular}
		\caption{Schema of Studiengaenge relation with sample records}
		\label{schema:studiengänge}
	\end{table}
	%
	\begin{table}
		\centering
		\tiny
		\begin{tabular}{|c|c|c|c|c|c|c|c|c|c|c|c|}
			\hline
			& KursID & Veranstaltungsteil & Tag & Uhrzeit & Dauer & ProfID & Raumnr & Veranstaltungsart & Wunschzeit & Hoererzahl & Bemerkung \\\hline\hline
			Data Type & TEXT & TEXT & TEXT & INTEGER & INTEGER & TEXT & TEXT & TEXT & TEXT & INTEGER & TEXT \\
			Constraints & NN, PK & NN, PK & & & & NN, PK & & NN & & & \\
			Checks & & & \(\in Tag\) & & & & & \(\in Veranstaltungsart\) & & & \\\hline
			& MOD-1 & 1.2 & Freitag & 11 & 2 & ProDr.JohBlö & L 1 & Vorlesung & [] & 400 & \\
			& ML2-1 & 1 & Donnerstag & 14 & 2 & ProDr.EykHül & O 2 & Vorlesung & [3,4,7] & 60 & \\\hline
		\end{tabular}
		\begin{tabular}{|c|c|c|c|c|c|c|c|c|c|}
			\hline
			& Zweiwoechentlich & Startwoche & Tafel & Beamer & Fenster & Ort & SonstigesRaumplanung & SonstigesZeitplanung & Fremdkurs \\\hline\hline
			Data Type & INTEGER & INTEGER & INTEGER & INTEGER & INTEGER & INTEGER & TEXT & TEXT & INTEGER \\
			Constraints & Default(0) & Default(1) & Default(0) & Default(0) & Default(0) & Default(3) & & & PK, Default(0) \\
			Checks & \(\in\{0,1\}\) & & \(\in\{0,1\}\) & \(\in\{0,1\}\) & \(\in\{0,1\}\) & \(\in\href{https://git.cs.upb.de/stulp.imt/database/blob/master/src/main/java/de/upb/stulp/database/enums/Ort.java}{Ort}\) & & & \(\in\{0,1\}\) \\\hline
			& 0 & 1 & 1 & 1 & 1 & 0 & & & 0 \\
			& 0 & 1 & 1 & 1 & 0 & 0 & & & 0 \\\hline
		\end{tabular}
		\caption{Schema of Plan relation with sample records}
		\label{schema:plan}
	\end{table}
	%
	\begin{table}
		\centering
		\begin{tabular}{|c|c|c|c|c|c|}
			\hline
			& KursID & Veranstaltungsteil & StudiengangsID & Pflichtfach & Alternativgruppe \\\hline\hline
			Data Type & TEXT & TEXT & TEXT & Text & TEXT \\
			Constraints & NN, PK & NN, PK & NN, PK, FK(Studiengaenge) & NN & NN \\
			Checks & & & & \(\in Pflichtfach\) & \(\in Alternativgruppe\) \\\hline
			& MOD-1 & 1.2 & InfM-1BPO4 & Ja & - \\
			& ML2-1 & 1 & InfoOhne-2MPO3 & Nein & - \\\hline
		\end{tabular}
		\caption{Schema of Pflicht relation with sample records}
		\label{schema:pflicht}
	\end{table}
	\begin{table}
		\centering
		\begin{tabular}{|c|c|c|c|c|c|c|c|}
			\hline
			& KursID1 & KursID2 & Veranstaltungsteil1 & Veranstaltungsteil2 & Type & Heavy & StudiengangsID \\\hline\hline
			Data Type & TEXT & TEXT & TEXT & Text & TEXT & INTEGER & TEXT \\
			Constraints & NN, PK & NN, PK & NN, PK & NN, PK & NN, PK & NN, PK & NN,PK \\
			Checks & & & & & & \(\in\{0,1\}\) & \\\hline
			& GP-1 & MOD-1 & 2.3 & 2.7 & STUDIENGANG\_CONFLICT & 0 & InfM-1BPO4 \\
			& MOD-1 & CA & 1.1 & 1 & PROFESSOR\_CONFLICT & 1 & \\\hline
		\end{tabular}
		\caption{Schema of IgnoredConflict relation with sample records}
		\label{schema:ignoredConflict}
	\end{table}

    \begin{table}
		\centering
		\begin{tabular}{|c|c|c|}
			\hline
			& ConflictType & Priority  \\\hline\hline
			Data Type & TEXT & INTEGER  \\
			Constraints & NN, PK, U & NN  \\
			Checks & & \\\hline
			& Raum & 1  \\
			& Professor & 2 \\
            & StudiengangLeicht & 3 \\
            & StudiengangSchwer & 1  \\
            & WunschOrt & 5  \\\hline
		\end{tabular}
		\caption{Schema of ConflictPriorities relation with sample records}
		\label{schema:ConflictPrios}
	\end{table}
     \begin{table}
    		\centering
    		\begin{tabular}{|c|c|c|c|c|}
    			\hline
    			& ConflictType & Query & Active  \\\hline\hline
    			Data Type & TEXT & TEXT & INTEGER  \\
    			Constraints & NN, PK, U, FK(ConflictPriorities) & NN & NN  \\
    			Checks & & & \(\in\{0,1\}\)  \\\hline
    			& WunschOrt & SELECT * FROM Plan WHERE Raumnr NOT LIKE 'F\%' AND Ort = 1 AND &  \\ 
                & & Raumnr NOT LIKE 'kein Raum' & 1  \\
    			& RaumZuKlein & SELECT * FROM Plan JOIN Raum ON Plan.Raumnr = Raum.Raumnr WHERE & \\ 
                & & Plan.Raumnr NOT LIKE 'kein Raum' AND Raum.Sitzplatzzahl < Plan.Hoererzahl & 0  \\\hline
    		\end{tabular}
    		\caption{Schema of CustomConflicts relation with sample records}
    		\label{schema:customConflict}
    	\end{table}
     \begin{table}
		\centering
		\begin{tabular}{|c|c|c|c|c|}
			\hline
			& Timeslot & Weekday & Time & Duration \\\hline\hline
			Data Type & INTEGER & TEXT & INTEGER & INTEGER \\
			Constraints & NN, PK, U & NN & NN & NN \\
			Checks & & \(\in\{Montag, Dienstag, ...\}\) & &  \\\hline
			& 1 & Montag & 8 & 3 \\
			& 2 & Montag & 11 & 2 \\\hline
		\end{tabular}
		\caption{Schema of Timeslot relation with sample records}
		\label{schema:timeslots}
	\end{table}
\end{landscape}

\paragraph{Further explanations of the Database scheme} 
As some notations might be unclear we provide some further explanations what they refer to.

\textbf{Plan Table} - The columns from the p2tool database that contain information about the \textit{Bemerkung} fields are currently not parsed by the local program database. The flags \textit{Plan.Zweiwoechentlich} until \textit{Plan.Ort} are parsed from the fields \textit{BemerkungAnkuendigung} and \textit{BemerkungPlanung}. The value of \textit{Plan.Ort} describes where the course should be planned (\textit{1} for Fürstenallee, \textit{2} for campus and \textit{3} for "don't care"). 

In the p2tool exist two flags \textit{Zweiwoechentlich} and \textit{Einstuendig}. In the local planning tool only the flag \textit{Plan.Zweiwoechentlich} is used. If an entry has \textit{Dauer=1} and the flag \textit{Plan.Zweiwoechentlich} it is automatically set to \textit{Dauer=2}. This makes the flag \textit{Einstuendig} redundant.

The value of \textit{Plan.Fremdkurs} indicates whether the \textit{Plan} entry was loaded from PAUL data. If so, the value is set to \textit{1}. 

\textbf{Pflicht Tabelle} - Throughout the database boolean values are stored as Interger values (0 or 1). However, for some unknown reason, \textit{Pflicht.Pflichtfach} is stored as a String with the values \textit{Ja} or \textit{Nein}. We did not consider changing it because it worked. \textit{Pflicht.Alternativgruppe} is set to \textit{ÜB} if there are multiple exercises to one lecture. It indicates that you have at least one alternative available for this exercise (at least two in total). 

\textbf{Timeslot Table} - In order to display \textit{Wunschzeiten} correctly, the timeslot table stores the time values for the given integer values. The timeslot \textit{1} refers to the time \textit{Monday, 8 o'clock, 3 hours}. This means that if \textit{Plan.Wunschzeiten} contains \textit{1}, the desired time would be Monday from 8 to 11 o'clock. 

\textbf{Missing foreign key constraints} - The database is currently missing foreign key constraints. 
\begin{enumerate}
    \item \textbf{Pflicht.KursID} references \textbf{Kurs}
    \item  \textbf{Plan.KursID} references \textbf{Kurs}
    \item  \textbf{Plan.ProfID} references \textbf{Professor}
    \item  \textbf{Plan.Raumnr} references \textbf{Raum}
    \item  \textbf{IgnoredConflict.KursID1} references \textbf{Kurs}
    \item  \textbf{IgnoredConflict.KursID2} references \textbf{Kurs}
    \item  \textbf{IgnoredConflict.StudiengangsID} references \textbf{Studiengaenge}
\end{enumerate}


\paragraph{Implementing the interface}
To implement the \href{https://git.cs.uni-paderborn.de/stulp/sven_database/-/blob/main/src/main/java/de/upb/stulp/database/localProgram/LocalProgramDatabase.java}{LocalProgramDatabase} interface, do as follows and implement the methods you need to overwrite:

\begin{lstlisting}[language=Java]
public class MyLocalProgramDatabase 
implements LocalProgramDatabase {
...
}
\end{lstlisting}
\noindent
An instance of this class will then be used by the local program to query
the database. When implementing MyLocalProgramDatabase, you will also
have to return an instance of a class implementing the
\href{https://git.cs.uni-paderborn.de/stulp/sven_database/-/blob/main/src/main/java/de/upb/stulp/database/localProgram/LocalRecordFactory.java}{LocalRecordFactory}
interface as this is the only way for users of your interface to
instantiate objects representing records in the database. These are needed
to pass filter information to queries provided in the interface. Thus, you will need to create the following class:

\begin{lstlisting}[language=Java]
public class MyLocalRecordFactoy 
implements LocalRecordFactory {
...
}

\end{lstlisting}
\noindent
The database interface only offers one insert, update, delete, and getRecords operation, yet the database has different tables. This is because these methods 
delegate the operation to the different table implementations. To take
advantage of this the
\href{https://git.cs.uni-paderborn.de/stulp/sven_database/-/blob/main/src/main/java/de/upb/stulp/database/Record.java}{Record}
interface has been introduced. This interface enables you to call
the corresponding method of the record interface for delegation.

\paragraph{How are Records of database relations implemented in the
	interface}

Every relation in your database has a corresponding entity class that
stores exactly one record of a relation,
e.g.~\href{https://git.cs.uni-paderborn.de/stulp/sven_database/-/blob/main/src/main/java/de/upb/stulp/database/localProgram/Professor.java}{Professor}.
Thus, when querying the database, you wrap every record in its
corresponding interface class and return this.

\paragraph{How to implement a record class}

To implement a record class, you inherit from the corresponding
interface class and override the missing methods. For example:

\begin{lstlisting}[language=Java]
public class MyProfessor extends Professor {
//make sure to make the constructors protected 
//such that the user is forced to use the 
//RecordFactory for instantiation
...
}
\end{lstlisting}
\noindent
The interface demands the implementation of methods for inserting, updating, deleting, and retrieving a record from the database. To allow different database implementations, the creation of a table itself has not to be implemented in the corresponding entity class. Instead, this can be done anywhere, depending on your needs. For each entity class its factory methods have to be implemented in MyLocalRecordFactory:

\begin{lstlisting}[language=Java]
public Plan createMyProfessor() {
return new MyProfessor(/*professor data here*/);
}

public Plan createMyProfessor() {
return new MyProfessor(/*professor data here*/);
}

\end{lstlisting}

\paragraph{Adding new functionality without database schema change}

For adding new functionality to the database without changing the schema, only the new required method has to be added to the \href{https://git.cs.uni-paderborn.de/stulp/sven_database/-/blob/main/src/main/java/de/upb/stulp/database/localProgram/LocalProgramDatabase.java}{LocalProgramDatabase} interface and implemented in MyLocalProgramDatabase or \href{https://git.cs.uni-paderborn.de/stulp/sven_database/-/blob/main/src/main/java/de/upb/stulp/database/impl/localProgram/SQLiteLocalProgramDatabase.java}{SQLiteLocalProgramDatabase}.

\paragraph{Changing existing relations}

Changing an existing relation requires the following steps: \\
\\
1. Update the abstract entity class corresponding to the relation to be changed, e.g. \href{https://git.cs.uni-paderborn.de/stulp/sven_database/-/blob/main/src/main/java/de/upb/stulp/database/localProgram/Professor.java}{Professor}. \\
\\
2. Update its implementation, e.g. MyProfessor or \href{https://git.cs.uni-paderborn.de/stulp/sven_database/-/blob/main/src/main/java/de/upb/stulp/database/impl/localProgram/SQLiteProfessor.java}{SQLiteProfessor}. \\
\\
3. Update the affected methods for record creation in the \href{https://git.cs.uni-paderborn.de/stulp/sven_database/-/blob/main/src/main/java/de/upb/stulp/database/localProgram/LocalRecordFactory.java}{LocalRecordFactory} interface so that the parameters fit. \\
\\
4. Update the affected methods in its implementation, e.g. MyLocalRecordFactory or \href{https://git.cs.uni-paderborn.de/stulp/sven_database/-/blob/main/src/main/java/de/upb/stulp/database/impl/localProgram/SQLiteLocalRecordFactory.java}{SQLiteLocalRecordFactory}. \\
\\
5. (Optional) Update \href{https://git.cs.uni-paderborn.de/stulp/sven_database/-/blob/main/src/test/resources/planning_tool_schema.db}{planning\_tool\_schema.db} if the implementation in \href{https://git.cs.uni-paderborn.de/stulp/sven_database/-/blob/main/src/main/java/de/upb/stulp/database/impl/localProgram/SQLiteProfessor.java}{SQLiteProfessor} is used because it is using JOOQ for automatic code generation of the corresponding classes for type safe SQL queries. \\
\\
6. Update affected queries in MyLocalProgramDatabase or \href{https://git.cs.uni-paderborn.de/stulp/sven_database/-/blob/main/src/main/java/de/upb/stulp/database/impl/localProgram/SQLiteLocalProgramDatabase.java}{SQLiteLocalProgramDatabase}. \\
\\
7. Update the method for creation of the table, if \href{https://git.cs.uni-paderborn.de/stulp/sven_database/-/blob/main/src/main/java/de/upb/stulp/database/impl/localProgram/SQLiteProfessor.java}{SQLiteProfessor} is used then update its method createTableIfNotExists.

\paragraph{Adding new relations}
Adding a new relation requires the following steps: \\
\\
1. Create the abstract entity class representing a record of the new relation:

\begin{lstlisting}[language=Java]
public abstract class NewRelation implements Record {
// use protected constructors only
// Add getter and setter methods
// overwrite equals and hashCode methods
}

\end{lstlisting}
\noindent
2. Implement it:

\begin{lstlisting}[language=Java]
public class MyNewRelation extends NewRelation {
// Implement methods
// use protected constructors only
}
\end{lstlisting}
\noindent
3. Add the factory methods for the relation record to the \href{https://git.cs.uni-paderborn.de/stulp/sven_database/-/blob/main/src/main/java/de/upb/stulp/database/localProgram/LocalRecordFactory.java}{LocalRecordFactory}:

\begin{lstlisting}[language=Java]
public interface LocalRecordFactory {
// place holder for old interface methods
NewRelation createNewRelation();
NewRelation createNewRelation(/*list of all fields to set*/);
}
\end{lstlisting}
\noindent
4. Implement the factory methods, e.g. in MyLocalRecordFactory or \href{https://git.cs.uni-paderborn.de/stulp/sven_database/-/blob/main/src/main/java/de/upb/stulp/database/impl/localProgram/SQLiteLocalRecordFactory.java}{SQLiteLocalRecordFactory}. \\
\\
5. Create a method for the creation of the table. \\
\\
6. (Optional) Create a demo table in \href{https://git.cs.uni-paderborn.de/stulp/sven_database/-/blob/main/src/test/resources/planning_tool_schema.db}{planning\_tool\_schema.db} if you are using the SQLite interface provided by this repository so that its schema is defined for JOOQ. \\

\subsubsection{Import Data From p2tool}
As the p2tool uses a database scheme different from the local program, it must be translated. This is done by opening the VLPT database exported from the p2tool. If this happens the function \texttt{downloadFromOtherDB()} in the \texttt{SQLiteLocalProgramDatabase} class is called:

\begin{lstlisting}[language=Java]
public void downloadFromOtherDB(String path, 
String localDBPath, Boolean update) {
// VLPT data is parsed from the database at path 
// and correctly inserted in database localDBPath

// if update is true
// tables are updated correctly and not overwritten
}
\end{lstlisting}

\paragraph{\color{red}Limitation} \color{black}
The adapter code to form the interface between p2tool and the local planning tool was taken from the old Monika tool. Thus, it has a different code structure. It utilizes prepared statements and could benefit from a cleaner code structure that supports changes in the p2tool database. In the table \textbf{Vorlesungsplan} in the p2tool database, columns for \textbf{Hoereranzahl}, \textbf{Tafel} etc. already exist. Yet the adapter requires the \textbf{BemerkungAnkuendigung} and \textbf{BemerkungPlanung} columns to extract this information from XML strings. This should be improved as the data is already offered by the p2tool but not currently used.


\subsection{Add own conflicts}
It is possible to add custom conflict types to the database. To be capable of doing so, it is crucial to understand the database structure first. See section \ref{sec:databasescheme} for some sample queries. Adding Queries requires some restrictions:

\begin{itemize}
    \item The query must return Plan entries.
\end{itemize}

The local planning tool's interface allows for the creation of new CustomConflicts. When a CustomConflict entry is created, an entry to ConflictPriority is automatically inserted. The default priority is 1. This functionality is tested for ConflictTypes that refer to one Plan entry. This means that, e.g., the 2nd query from \autoref{schema:customConflict} returns the Plan entries planned in rooms that are too small. The conflict does not exist between two separate entries. Theoretically, this should work as well, but it is untested.

\subsection{Further comments}
\subsubsection{Javadoc}
A detailed overview of the database and its classes can be found at \href{https://git.cs.uni-paderborn.de/stulp/sven_database/-/blob/main/javadoc}{git.cs.uni-paderborn.de/stulp/sven\_database/-/blob/main/javadoc}. The list with all classes can be viewed in the \textbf{allclasses-index.html} file.

\subsubsection{Not used components}
\begin{enumerate}
    \item \textbf{enums.Alternativgruppe}: This enum is used in the database implementation, yet the adapter code is not actually using its values. It is implemented with strings.
\end{enumerate}
