\subsection{Überblick über den Planungsprozess}
Der Planungsprozess ist in fünf Phasen unterteilt, von denen nicht alle jedes Semester durchlaufen werden.
%
\paragraph*{Phase 1}
In Phase 1 entscheidet jede Fakultät, welche Lehrveranstaltungen von ihr für welchen Studiengang angeboten werden sollen. Diese Phase wird allerdings nur alle  sieben bis acht Jahre pro Studiengang wiederholt.  Dies geschieht im Rahmen der Reakkreditierung oder wenn neue Kolleg*innen berufen werden.  Da diese nicht für jeden Studiengang im selben Jahr stattfindet, hat dies Einfluss auf die in Phase 2 erstellten Modulhandbücher, weil diese auch Lehrveranstaltungen von anderen Studiengängen enthalten, die als Nebenfach belegt werden können.
%
\paragraph*{Phase 2}
In Phase 2 erstellt  der/die für einen Studiengang zuständige  Studiengangsbeauftragte das Modulhandbuch für den Studiengang, den er verwaltet. Dazu selektiert er Lehrveranstaltungen aus der Liste der angebotenen Lehrveranstaltungen aus Phase 1 und versieht diese mit ECTS-Punkten. Aufgrund  diverser Änderungen des Lehrangebots  wird diese Phase  etwa  alle ein bis drei Jahre wiederholt.
%
\paragraph*{Phase 3}
In Phase 3 wählen die Professoren Lehrveranstaltungen aus den Modulhandbüchern ihrer Studiengänge, die sie anbieten möchten. Ein Professor kann dabei keine Lehrveranstaltung anbieten, die nicht Teil eines Modulhandbuchs ist.  Für Seminare, Proseminare und Projektgruppen gibt es jedoch folgende Aussage:  Ein Professor darf einen  Veranstaltungstitel  wählen, sofern es ihm im Modulhandbuch freigestellt ist. Diese Phase wird jedes Semester wiederholt.
%
\paragraph*{Phase 4}
In dieser Phase wird jedem Lehrveranstaltungsteil jeder Lehrveranstaltung eines Studiengangs ein Raum, Tag und eine Uhrzeit durch den Planer des Studiengangs zugewiesen, sodass möglichst keine Studiengangskonflikte, Professorkonflikte oder Raumkonflikte entstehen. Der Planer hat zudem die Möglichkeit auf diverse Wünsche der Professoren bei der Planung einzugehen. Diese Phase kann mehrfach pro Semester durchlaufen werden, falls nötig.  Dies ist der Fall, wenn Veranstaltungen nachgemeldet werden oder Meldungen geändert werden. Für diese Phase typische Ausgaben sind: Vorlesungsplan als .pdf, Semesterpläne für z.B. Informatik-Master, Informatik-Bachelor im Semester 1,3,5 oder 2,4,6. 
%
\paragraph*{Phase 5}
Phase 5 ist eine Phase der Rückmeldung für den Planer.  Es gibt das Vorlesungsverzeichnis als .pdf und .tex. Große Räume werden an die zentrale Raumverwaltung gemeldet und Pläne werden für alle Lehrenden, Studierendengruppen und Fakultätsräte erstellt und ausgehändigt. Zudem werden alle weiteren Räume am Campus und in der Fürstenallee gemeldet. Alle Beteiligten   haben nun die Möglichkeit dem Planer mitzuteilen, ob ihnen etwas an ihren Plänen unter Angabe triftiger Gründe missfällt. Sollte Nachbesserungsbedarf bestehen, werden Phase 4 und 5 wiederholt.