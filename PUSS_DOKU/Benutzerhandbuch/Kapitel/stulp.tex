\section{PUSS Tool Handbuch}
In diesem Kapitel wird die Nutzung von STULP beschrieben, welches ein lokales Stundenplanunterstützungstool ist, welches im Rahmen einer anderen Projektgruppe an der Universität Paderborn entwickelt wurde.
Wir wollen hier nur eine kurze Einführung in die Benutzung des Tools geben, da wir nicht alle Funktionen des Tools verwenden und eine ausführliche Dokumentation bereits von den Entwicklern des Tools bereitgestellt wurde.
Dies ist unter dem folgenden Link zu finden: \todo{ref}
Da dieses Tool ursprünglich für die Stundenplanerstellung an Hochschulen entwickelt wurde, gibt es einige Namensunterschiede zu unserem Anwendungsfall an Schulen.
Diese Unterscheide werden wir in einem eigenen Kapitel Unterkapitel \todo{Ref} erläutern.
Außerdem werden wir in einem weiteren Unterkapitel \todo{Ref} ein konkretes Beispiel für die Nutzung des Tools zur Erstellung eines Stundenplans an einer Schule geben. Dies ist auch in Form eines Videos verfügbar, welches unter dem folgenden Link zu finden ist: \todo{ref}
\subsection{Unterschiede zwischen Hochschulen und Schulen}
In diesem Unterkapitel wollen wir die wichtigsten Unterschiede in der Terminologie und den Konzepten zwischen Hochschulen und Schulen erläutern, um Missverständnisse bei der Nutzung des PUSS Tools zu vermeiden.
