\section{Benutzung des Import Tools}
In diesem Kapitel wird die Benutzung des Import Tools beschrieben, welches im Rahmen des PUSS Projekts entwickelt wurde.
Das Import Tool dient dazu, bestehende Stundenpläne aus dem Untis System in das PUSS Tool zu importieren, um eine nahtlose Integration und Weiterverarbeitung der Daten zu ermöglichen.
Hierbei wird bereits davon ausgegangen, dass die notwendigen DIF Dateien bereits erstellt und bereitgestellt wurden.
\subsection{Installation des Import Tools}
Das Import Tool ist als eigenständige Anwendung verfügbar und kann auf Anfrage bei den Autoren dieses Dokuments zur Verfügung gestellt werden.
Die Installation des Import Tools erfolgt durch das Ausführen der bereitgestellten Installationsdatei, welche den Installationsprozess automatisch durchführt.
Diese Installationsdatei ist für zwei verschiedene Betriebssysteme verfügbar: Windows und Linux.
Nach dem erfolgreichen Abschluss der Installation kann das Import Tool über das Startmenü (Windows) oder das Anwendungsmenü (Linux) gestartet werden.
\subsection{Benutzung des Import Tools}
Nach der Installation des Import Tools muss der Anwender die DIF Dateien, welche zuvor aus Untis exportiert wurden, in den dafür vorgesehenen Ordner im Import Tool Verzeichnis kopieren.
\todo{Beschreibung und Bild hinzufügen}
Anschließend kann das Import Tool über die Kommandzeile gestartet werden.
Dafür öffnet der Anwender ein Terminal (Linux) oder die Eingabeaufforderung (Windows) und navigiert in das Verzeichnis, in welchem das Import Tool installiert wurde.
Wie dies genau funktioniert, ist im Anhang \todo{Ref} beschrieben, da dies je nach Betriebssystem unterschiedlich ist und davon ausgegangen wird, dass der Anwender über grundlegende Kenntnisse im Umgang mit der Kommandozeile verfügt.
Sobald sich der Anwender im richtigen Verzeichnis befindet, kann das Import Tool durch Eingabe des folgenden Befehls gestartet werden:
\todo{Befehl einfügen}
Worauf hin das Import Tool gestartet wird und die DIF Dateien automatisch eingelesen und verarbeitet werden.
Nach erfolgreichem Import der Daten wird eine Bestätigungsmeldung in der Kommandozeile ausgegeben (siehe Abbildung \todo{Ref}).
Nun können die importierten Daten im PUSS Tool geöffnet und bearbeitet werden.